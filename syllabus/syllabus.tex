\documentclass{article}

\usepackage[top=0.75in,bottom=1in,left=1in,right=1in]{geometry}
\usepackage{enumerate,url,hyperref,amssymb,caption}

\newcommand{\prof}{Dr. Drew Lewis}
\newcommand{\profemail}{{\tt drewlewis@southalabama.edu}}
\newcommand{\profoffice}{306 MSPB}
\newcommand{\profhours}{TBD}
\newcommand{\course}{Math 237}
\newcommand{\semester}{Fall 2022}
\newcommand{\classtime}{TR 9:30-10:45 \\ TR 11-12:15}
\newcommand{\LMS}{Canvas}


\begin{document}



\begin{center}
{\bf \large \course{}\\
Linear Algebra} \\
\end{center}

\vspace{0.25in}


\noindent \begin{tabular}{@{}ll}
\begin{tabular}{ll}
{\bf Course information:}& \course{} -- Linear Algebra \\
& \semester \\
& Course format: Web-enhanced \\
& \\
{\bf Meeting times:} & TR 9:30-10:45 (102) \\
& TR 11:00-12:15 (103)
\ 
\end{tabular} &
\begin{tabular}{ll}
{\bf Instructor:}& \prof \\
& \profemail \\
& {\tt(251) 341-3094} \\
\textbf{Office:} & 306 MSPB \\
& \\
{\bf Student Hours:} & \profhours \\
%{\bf Location:} & ASC 1303A \\
\end{tabular}
\end{tabular}

\section*{\fontsize{12}{15}\selectfont Student Success}
I am committed to helping each and every one of you achieve your goals. I fundamentally believe that \textbf{everyone is a math person} and is capable of succeeding in this course. If there is something we could do differently that will help you succeed, please come talk to me. 

There are many services on campus to help you succeed. I enthusiastically support the mission of our Office of Student Disability Services; if you are registered with them, please come speak to me in office hours so I can ensure I am doing everything needed to enable your success.


\section*{\fontsize{12}{15}\selectfont Basic Needs Security}
Any student who has difficulty affording groceries or accessing sufficient food to eat every day, or who lacks a safe and stable place to live, and believes this may affect their performance in the course, is urged to contact the Dean of Students for support. Furthermore, please notify the professor if you are comfortable in doing so. This will enable me to try and provide information on accessing additional resources.  Information on the campus food pantry is available at \url{https://southalabama.edu/departments/sga/foodpantry.html}.


\section*{\fontsize{12}{15}\selectfont Course Format}
%Due to the ongoing pandemic, we will not meet in person. \textbf{Class sessions will take place via Zoom at the regularly scheduled times}. Links to the Zoom sessions will be available in \LMS. By meeting virtually, you will be able to collaborate with a group of classmates on our in-class activities in a way that would be impossible in a socially distanced classroom.

    This class is taught by a method called {\bf team-based learning}.  You will be assigned to a team that you will work with on various activities in class each day.  The course is divided into {\bf 5 modules}.
    \begin{itemize}
    \item Before each module begins, {\bf you will be responsible for ensuring your own readiness for the module}.  A list of learning outcomes for the readiness assurance process is available in \LMS; you should be able to do each of these things before coming to class on the first day of the module.  Some of these readiness assurance outcomes are new material, some of these readiness assurance outcomes are topics from previous classes, and some are topics from earlier in this class.  Along with these outcomes in \LMS\ are some preparation resources (videos or reading material) to help you prepare.
    \item The first day of each module will be dedicated to the {\bf Readiness Assurance Process}.  The dates for these are located in the course calendar.


    On these days, you will first take an {\bf Individual Readiness Assurance Check}.  After submitting this, working with your teammates you will retake the same problems as the {\bf Team Readiness Assurance Check}.  These are designed to measure if you are prepared for the team activities on subsequent days.

    \item On the other class days, you will work with your teammates on a series of activities designed to guide you through discovering the course material.

    
    \end{itemize}

\section*{\fontsize{12}{15}\selectfont How To Be Successful in This Course}
Students who have been successful in this course in the past have told me that the following things were key to their success:
\begin{itemize}
%\item \textbf{Attend every class meeting}. Obviously, this will sometimes not be possible, but make it a priority. Most students say that most of their learning takes place when they are working the class activities with their teammates.
\item \textbf{Work homework problems}. Just like your favorite sport, you do not learn mathematics by watching someone else do it, but by doing it yourself. You need to work practice problems to learn. Resist the temptation to just watch videos or review your notes.
\item \textbf{Use your teammates and classmates as a resource}. Ask questions on \LMS. Ask your teammates questions during class. We are here to help each other learn.
\item \textbf{Reassess early and often}. The grading system in this class offers you lots of flexibility. Don't wait until the end of the semester to try and demonstrate your learning of everything.
\item \textbf{Complete the reflections}. These reflection assignments are designed to help you make sure you are progressing towards the grade you want. 
\end{itemize}

\section*{\fontsize{12}{15}\selectfont Communication Plan}

\begin{itemize}
\item Questions about the content or the course that apply to the entire class should be posted in \LMS. This way, I can post an answer that everyone can see, helping everyone get the information they need faster. If you email me such a question, I will ask you to post it in the \LMS ; this is not because I don't want to answer it, but because I think answering it publicly will be helpful to everyone.
\item For questions that pertain to you specifically or your progress in the course, the best way to reach me is by email at {\tt drewlewis@southalabama.edu}. I generally respond to emails within one working day.  Do not expect a reply to an email sent outside of business hours until the next day.
\end{itemize}

\section*{\fontsize{12}{15}\selectfont Student Hours and Instructor Availability}

Student  hours are times I have specifically set aside to be available for you to drop-in to ask questions or talk with me about anything else. 
These times are listed at the top of this document. 
In addition, you can use \url{https://calendly.com/dr-lewis/10min} to schedule an appointment with me at other times. 
My office is 306 MSPB--this is in the main Mathematics \& Statistics Department Office (around the corner to the left).

\section*{\fontsize{12}{15}\selectfont Disaster Plan}
In the event of a campus closure (e.g. due to a pandemic or weather), we will continue to meet virtually via Zoom at our regular time as best we can. Any changes to our plans will be communicated by email and/or \LMS\ Announcement. In such an event, I ask that we all extend grace and patience to each other; we will all do the best we can in the circumstances.

\section*{\fontsize{12}{15}\selectfont Course Description}
This course provides an introduction to linear algebra. Topics include systems of linear equations, matrices, Gaussian elimination, rank, linear independence, subspaces, basis, dimension, linear transformations, determinants, eigenvalues and eigenvectors, change of basis, diagonalization, the abstract concept of a vector space, and applications. Core Course.

\section*{\fontsize{12}{15}\selectfont Learning Outcomes}
At the completion of this course, each student will be able to...
\begin{enumerate}[1)]
\item Work collaboratively on difficult mathematics problems
\item Solve systems of linear equations.
\item Determine whether or not a set with given operations is a vector space or a subspace of another vector space.
\item Determine properties of sets of vectors such as whether they are linearly independent, whether they span, and whether they are a basis of a given subspace.
\item Perform fundamental operations in the algebra of matrices, including multiplying and inverting matrices.
\item Use and apply algebraic properties of a linear transformation.
\item Determine geometric information about a linear transformation, including computing determinants, eigenvalues, and eigenvectors.
\end{enumerate}

\section*{\fontsize{12}{15}\selectfont General Education Learning Outcomes}
This course addresses the following university-wide learning outcomes in quantitative reasoning:
\begin{enumerate}[1)]
\item Students will evaluate information presented in mathematical forms (e.g., equations, graphs, diagrams, tables, words).
\item Students will convert relevant information into various mathematical forms (e.g., equations, graphs, diagrams, tables, words)
\end{enumerate}


\section*{\fontsize{12}{15}\selectfont Topics}
We will cover the topics outlined on the \textbf{Course Standards}
sheet provided to you, in the order that they appear on that sheet.


\section*{\fontsize{12}{15}\selectfont Textbook}
This course will follow the activities at \url{https://teambasedinquirylearning.github.io/linear-algebra/2022/}. Most students find that this is the only text they need.

If you prefer an expensive book from a major publisher, the bookstore will be happy to sell you ``Elementary Linear Algebra with Applications'' by Kolman and Hill.  A free alternative is ``Linear Algebra'' by Hefferon, available at \url{http://joshua.smcvt.edu/linearalgebra/book.pdf}. We will not directly use either of these, but feel free to use them as an extra reference.


\section*{\fontsize{12}{15}\selectfont Attendance Policy}
We are (still) living through a pandemic. The health and safety of you and others comes first. While in the Before Times class attendance was very highly correlated with student success in this course, I trust each of you to use your best judgment to keep you and those around you safe, and to attend our class when it makes sense to do so.  Just send me an email and let your teammates know.

\section*{\fontsize{12}{15}\selectfont Standards Based Grading}
This course is graded by a methodology called {\bf standards based grading}.  Instead of receiving one percentage grade for an assessment, you will be assessed on whether or not you demonstrated excellence on individual {\bf learning standards}.  A list of these 24 standards is available in \LMS.  Your grade in the course will be based on how many of these standards you demonstrate excellence on.  {\bf On each standard, you will demonstrate excellence on two separate occasions}.



    \subsection*{\fontsize{10}{12}\selectfont Feedback}
    On a written assessment, you will not receive a score or a percentage.  For each standard on that assessment, you will be scored into one of three categories.
\begin{itemize}
    \item \textbf{Demonstrated Excellence}: You successfully demonstrated {\bf Excellence} of that standard.  Great job!  Check off another box on your progress sheet.
    \item \textbf{Minor Revision Needed}: You have a minor mistake, unrelated to the standard being assessed.  These usually fall into two main categories: arithmetic mistakes, and poor presentation (which are both usually symptoms of artificial time constraints rather than a lack of content knowledge).  You should {\bf write out a complete solution} to the problem and resubmit the assessment in Canvas (you can leave the other questions blank). \textbf{You must do this within a week of the original assessment}.    
    \item \textbf{Reassessment Needed}: You'll need to work a new problem. See `Reassessment' below for options.
    \end{itemize}

You can track your progress through the `Learning Mastery' part of the gradebook in \LMS. 

    \subsection*{\fontsize{10}{12}\selectfont Reassessment}
    You will have multiple opportunities to demonstrate your learning of each standard.  
    \begin{enumerate}[1)]
    \item Each week we will have either a short assessment (``quiz''), covering  up to 4 standards, or a long assessment (``exam''), covering all of the standards discussed so far.  {\bf A detailed schedule listing exactly which day each standard appears on quizzes is posted in \LMS.}
    \item Additionally, if you receive a ``Reassessment needed'' on a standard, you can work an on-demand reassessment through Canvas.  There are a few caveats to on-demand reassessments:
    \begin{itemize}
    \item In order to reassess a single standard, you must have completed additional practice problems. Fill out a `Reassessment Request Form' (linked in \LMS).
    \item Once you have done this, I will take a quick look at your practice problems; if there are no issues, I will assign a one question reassessment in \LMS.
    \item Note that there will be a small lag, as I want to look at your reassessment request and manually approve it. I'll try to do this each day, though if many people submit these (especially towards the end of the semester) I will only do this as fast as I can grade the reassessments.
\end{itemize}
    \end{enumerate}



\section*{\fontsize{12}{15}\selectfont Assignments}
There are two broad categories of assignments in this class. These will all be submitted through \LMS.
\begin{itemize}
\item \textbf{Learning Assessments: } As mentioned above, at the end of most weeks will be either a quiz or exam; barring unforeseen events (e.g. weather days) there will be 8 quizzes and 4 exams.   The dates for each, including which standards will appear on which quizzes, is available in the Course Calendar in \LMS. 
\item \textbf{Reflections: } Along with each exam you will complete a self-assessment, in which you are asked to reflect on your learning in the course. In addition, at mid-term and final you will complete a more extensive reflection assignment.
\end{itemize}

\section*{\fontsize{12}{15}\selectfont Calculator Policy}

Calculators of any sort may be used on quizzes or exams, if desired. However, most students find the SageCells embedded into the assessments in Canvas more than sufficient for their needs.

\section*{\fontsize{12}{15}\selectfont Missed Coursework}
Each assignment will have a ``due date'' scheduled in Canvas.  Typically, assignments will also be open for some time afterwards (during this period Canvas will mark them as `late').  If you need a different due date for an assignment, fill out the `Due Date Change Request Form' linked in Canvas.  

Because of the registrar's deadline for (and requirement that I submit) final grades, all work must be submitted by May 5.  If you need more time than that, come talk to me about an incomplete.

\noindent \begin{minipage}{\textwidth}
\section*{\fontsize{12}{15}\selectfont Grading}
At the end of the semester, you will fill out a final reflection on your final that we will use to determine your course grade.  Here are some general guidelines for what you should aim for over the course of the semester: \\

\begin{center}
 \begin{tabular}{l|l} 
To earn a  ... & ... you should \\
\hline
A & Demonstrate excellence (twice) on 22 standards;  \\
\hline

B & Demonstrate excellence (twice) on 20 standards; \\
\hline

C 	& Demonstrate excellence (twice) on 17 standards;\\
\hline

D & Demonstrate excellence (twice) on 15 standards;\\
\hline

F 	& Not fit in the above categories. \\
\hline
\end{tabular}
\end{center}
\end{minipage}


\section*{\fontsize{12}{15}\selectfont Homework}
The only way to learn mathematics is to do mathematics; thus, our class time is centered around students doing mathematics.  Additionally, you will require practice outside of class; a list of suggested exercises corresponding to each standard is available in \LMS. 

I will neither collect nor grade homework, as past experience has not shown this to be a good use of instructor or student time.  Instead, you should work as many problems as you need to. If you need feedback on your homework problems, bring them to my office hours and I will be happy to discuss them with you.


\section*{\fontsize{12}{15}\selectfont Student Academic Conduct Policy}
All students are expected to adhere to the Student Academic Conduct Policy, which you can view at
{\tt http://www.southalabama.edu/bulletin/current/student-affairs/conduct.html}.  Students violating this policy will be given one or more of the following penalties based on the severity of the offense:  1) Loss of all demonstrated excellence marks on all standards affected by the misconduct; 2) Reduction in final course grade by a letter grade; 3) Automatic course failure.


\section*{\fontsize{12}{15}\selectfont Syllabus Changes}
While I try hard to stick to the plans laid out here, this syllabus is subject to change (due to pandemics, weather events like hurricanes, etc.). Any changes made will reflect the spirit of this original syllabus, and will be updated on \LMS.




\end{document}
