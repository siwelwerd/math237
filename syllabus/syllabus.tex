\documentclass{article}

\usepackage[top=0.75in,bottom=1in,left=1in,right=1in]{geometry}
\usepackage{enumerate,hyperref,amssymb,caption}

\newcommand{\prof}{Dr. Drew Lewis}
\newcommand{\profemail}{{\tt drewlewis@southalabama.edu}}
\newcommand{\profoffice}{306 MSPB}
\newcommand{\profhours}{TBD}
\newcommand{\course}{Math 237}
\newcommand{\semester}{Fall 2022}
\newcommand{\classtime}{TR 9:30-10:45 \\ TR 11-12:15}
\newcommand{\LMS}{Canvas}


\begin{document}



\begin{center}
{\bf \large \course{}\\
Linear Algebra} \\
\end{center}

\vspace{0.25in}


\noindent \begin{tabular}{@{}ll}
\begin{tabular}{ll}
{\bf Course information:}& \course{} -- Linear Algebra \\
& \semester \\
& Course format: Web-enhanced \\
& \\
{\bf Meeting times:} & (101) MWF 9:05-9:55 \\
& (102) MWF 10:10-11:00\\
\end{tabular} &
\begin{tabular}{ll}
{\bf Instructor:}& \prof \\
& \profemail \\
& {\tt(251) 219-8163} \\
& \\
{\bf Office hours:} & by appointment
\end{tabular}
\end{tabular}

\section*{\fontsize{12}{15}\selectfont Student Success}
I am committed to helping each and every one of you achieve your goals. I fundamentally believe that \textbf{everyone is a math person} and is capable of succeeding in this course. If there is something we could do differently that will help you succeed, please come talk to me. 

There are many services on campus to help you succeed. I enthusiastically support the mission of our Office of Student Disability Services; if you are registered with them, please come speak to me in office hours so I can ensure I am doing everything needed to enable your success.


\section*{\fontsize{12}{15}\selectfont Basic Needs Security}
Any student who has difficulty affording groceries or accessing sufficient food to eat every day, or who lacks a safe and stable place to live, and believes this may affect their performance in the course, is urged to contact the Dean of Students for support. Furthermore, please notify the professor if you are comfortable in doing so. This will enable me to try and provide information on accessing additional resources.  Information on the campus food pantry is available at \url{https://southalabama.edu/departments/sga/foodpantry.html}.


\section*{\fontsize{12}{15}\selectfont Course Format}
Due to the on-going pandemic, we will not meet in person. \textbf{Class sessions will take place via Zoom at the regularly scheduled times}. Links to the Zoom sessions will be available in \LMS. By meeting virtually, you will be able to collaborate with a group of classmates on our in-class activities.

    This class is taught by a method called {\bf team-based learning}.  You will be assigned to a team that you will work with on various activities in class each day.  The course is divided into {\bf 5 modules}.
    \begin{itemize}
    \item Before each module begins, {\bf you will be responsible for ensuring your own readiness for the module}.  A list of learning outcomes for the readiness assurance process is available in \LMS; you should be able to do each of these things before coming to class on the first day of the module.  Some of these readiness assurance outcames are new material, some of these readiness assurance outcomes are topics from previous classes, and some are topics from earlier in this class.  Along with these outcomes in \LMS\ are some preparation resources (videos or reading material) to help you prepare.
    \item The first day of each module will be dedicated to the {\bf Readiness Assurance Process}.  The dates for the RATs are located in the course calendar.


    On these days, you will first take an {\bf Individual Readiness Assurance Test (iRAT)}.  After submitting this, working with your teammates you will retake the same test as the {\bf Team Readiness Assurance Test (tRAT)}.  These are not ``tests'' in the traditional sense; they are designed to measure if you are prepared for the team activities on subsequent days.

    \item On the other class days, you will work with your teammates on a series of activities designed to guide you through discovering the course material.

    \item To help your team improve in its collaboration throughout the semester, we will use a {\bf peer evaluation} process.  Several times during the semester, you will fill out a  survey (through the TEAMMATES website) evaluating your peers.  
    You will also receive written feedback given by your peers on these surveys (in an anonymized form), to help you maximize your future contributions to the team.

    \end{itemize}

\section*{\fontsize{12}{15}\selectfont How To Be Successful in This Course}
Students who have been successful in this course in the past have told me that the following things were key to their success:
\begin{itemize}
\item \textbf{Attend every class meeting}. Obviously, this will sometimes not be possible, but make it a priority. Most students say that most of their learning takes place when they are working the class activities with their teammates.
\item \textbf{Work homework problems}. Just like your favorite sport, you do not learn mathematics by watching someone else do it, but by doing it yourself. You need to work practice problems to learn. Resist the temptation to just watch videos or review your notes.
\item \textbf{Use your teammates and classmates as a resource}. Ask questions on \LMS. Ask your teammates questions during class. We are here to help each other learn.
\item \textbf{Reassess early and often}. The grading system in this class offers you lots of flexibility. Don't wait until the end of the semester to try and demonstrate mastery of everything.
\item \textbf{Complete a self assessment and participate in the weekly discussion each week}. The self assessments are designed to help you make sure you are progressing towards the grade you want. The weekly discussions will not usually apply directly to our content, but rather help us think about mathematics and our course material a little more broadly.
\end{itemize}

\section*{\fontsize{12}{15}\selectfont Communication Plan}

\begin{itemize}
\item Questions about the content or the course that apply to the entire class should be posted in \LMS. This way, I can post an answer that everyone can see, helping everyone get the information they need faster.
\item For questions that pertain to you specifically or your progress in the course, the best way to reach me is by email at {\tt drewlewis@southalabama.edu}. I generally respond to emails within one working day.  Do not expect a reply to an email sent outside of business hours until the next day.
\item Additionally, you can text or call me at {\tt(251) 219-8163}. This is my cell phone, not an office number.
\end{itemize}

\section*{\fontsize{12}{15}\selectfont Office Hours and Instructor Availability}

Office hours are times I have specifically set aside to be available for you to drop-in to ask questions, attempt reassessments, or talk with me about anything else. They would probably be better named \textbf{Student Hours}. Because of social distancing requirments, all office hours will be done virtually and by appointment; {\bf send me an email to schedule a time}. The Zoom link for office hours is posted in \LMS.

\section*{\fontsize{12}{15}\selectfont Disaster Plan}
In the event of a campus closure (e.g. due to a pandemic or weather), we will continue to meet virtually via Zoom at our regular time as best we can. In such an event, I ask that we all extend grace and patience to each other; we will all do the best we can in the circumstances.

\section*{\fontsize{12}{15}\selectfont Instructional Tools}
We will use the following tools in this course.
\begin{itemize}
\item We will use \textbf{Zoom}, particularly the \textbf{Zoom breakout groups}, for class meetings. The links to these can be found in \LMS.
\item We will use \textbf{Google Jamboards} to collaborate during class time. A link to the Jamboard for each class day will be posted in \LMS.
\item Homework problems will be available through a website called \textbf{MasterIt}. A link can be found in \LMS.
\item We will make use of \textbf{Discussion boards} in \LMS. 
\item Assessments (e.g. quizzes and tests) will be submitted through \LMS as well.
\end{itemize}


\section*{\fontsize{12}{15}\selectfont Course Description}
This course provides an introduction to linear algebra. Topics include systems of linear equations, matrices, Gaussian elimination, rank, linear independence, subspaces, basis, dimension, linear transformations, determinants, eigenvalues and eigenvectors, change of basis, diagonalization, the abstract concept of a vector space, and applications. Core Course.

\section*{\fontsize{12}{15}\selectfont Learning Outcomes}
At the completion of this course, each student will be able to...
\begin{enumerate}[1)]
\item Work collaboratively on difficult mathematics problems
\item Solve systems of linear equations.
\item Determine whether or not a set with given operations is a vector space or a subspace of another vector space.
\item Determine properties of sets of vectors such as whether they are linearly independent, whether they span, and whether they are a basis of a given subspace.
\item Perform fundamental operations in the algebra of matrices, including multiplying and inverting matrices.
\item Use and apply algebraic properties of a linear transformation.
\item Determine geometric information about a linear transformation, including computing determinants, eigenvalues, and eigenvectors.
\end{enumerate}

\section*{\fontsize{12}{15}\selectfont General Education Learning Outcomes}
This course addresses the following university-wide learning outcomes in quantitative reasoning:
\begin{enumerate}[1)]
\item Students will evaluate information presented in mathematical forms (e.g., equations, graphs, diagrams, tables, words).
\item Students will convert relevant information into various mathematical forms (e.g., equations, graphs, diagrams, tables, words)
\end{enumerate}


\section*{\fontsize{12}{15}\selectfont Topics}
We will cover the topics outlined on the \textbf{Course Standards}
sheet provided to you, in the order that they appear on that sheet.
These topics are taken from the first seven chapters of the textbook,
but are arranged in a more logical order.

\section*{\fontsize{12}{15}\selectfont Textbook}
The suggested textbook is ``Elementary Linear Algebra with Applications'' by Kolman and Hill.  This will serve as a supplement to the instructor-provided notes I post in \LMS.  A free alternative is ``Linear Algebra'' by Hefferon, available at \url{http://joshua.smcvt.edu/linearalgebra/book.pdf}.


\section*{\fontsize{12}{15}\selectfont Attendance Policy}
Attendance is required to be successful in this course, and will be tracked each day.
``Perfect'' attendance is considered anything greater than 80\%
to allow for a small number of short term absences for any reason.
If you find yourself missing more class than this, come talk to me.


\section*{\fontsize{12}{15}\selectfont Standards Based Grading}
This course is graded by a methodology called {\bf standards based grading}.  Instead of receiving one percentage grade for an assessment, you will be assessed on whether or not you mastered individual {\bf learning standards}.  A list of these 24 standards is available in \LMS.  Your grade in the course will be based on how many of these standards you demonstrate mastery of.  {\bf On each standard, you will have the opportunity to earn up to two checkmarks; the total number of checkmarks you earn will determine your grade} (see below).



    \subsection*{\fontsize{10}{12}\selectfont Feedback}
    On a written assessment, you will not receive a score or a percentage.  There will be a list of the standards that were covered on that assessment, and a letter or symbol next to each one.
    \begin{itemize}
    \item An {\bf M} means you successfully demonstrated {\bf Mastery} of that standard.  Great job!  Check off another box on your progress sheet.
    \item A {\bf *} means you have a minor mistake, unrelated to the standard being assessed.  For example, if you make a single arithmetic error while row reducing a matrix but do everything else correctly, you will earn a *.  If you receive a *, I won't give you any other feedback on that problem.  You should determine your mistake on your own, and either submit a video explanation or come to my office hours and explain your error and how to fix it; I will then modify the * into a M.  {\bf This must be done in the week following the assessment}.  After a week, the * will be treated the same as an R.
        \item A {\bf W} means your work was correct, but your exposition was lacking in some way--most often this is poor or ambiguous notation, or a missing step.  You need to {\bf write out a complete solution} to the problem on a W form (available in \LMS), and submit this through \LMS along with the original assessment.  Like a *, this must be done within a week of the assessment, or the W will revert to an R.
    \item A {\bf R} means you are eligible to {\bf Reassess in my office hours}.  You will earn this mark if you made a good faith attempt and demonstrated partial understanding, but did not demonstrate full mastery of that standard on this assessment.  You should work some more practice problems, and come by my office hours if you still do not understand.  You can either wait for the next quiz, or reassess during my office hours (see ``Reassessments'' below).

    \item A {\bf N} means there was {\bf No Significant Evidence} of understanding, and you are {\bf Not Eligible} for an office hours reassessment on this standard.  Your next attempt must come on an in-class assessment.  This is the most rarely used mark.

    \end{itemize}
    Unfortunately, this grading system is far too sophisticated for the gradebook in USAOnline to handle.  So you will periodically receive an automatic email from me detailing your current progress in the course.  If you have any questions about how to interpret where you stand, come to my office hours to discuss.


    \subsection*{\fontsize{10}{12}\selectfont Reassessment}
    You will have multiple opportunities to demonstrate mastery of each standard.  
    \begin{enumerate}[1)]
    \item Each week we will have either a short assessment (``quiz''), covering  up to 4 standards, or a long assessment (``exam''), covering all of the standards discussed so far.  {\bf A detailed schedule listing exactly which day each standard appears on quizzes is posted in \LMS.}
    \item There will be a final exam, which will be your final opportunity to demonstrate mastery. 
    \item Additionally, if you receive an R on a standard, you can come to my office to reassess the standard.  There are a few caveats to office hour reassessments:
    \begin{itemize}
    \item In order to reassess a standard in my office, you must have completed additional practice problems and {\bf fill out a reassessment form} (blank forms are available in \LMS).  If you come to my office without a reassessment form, you must complete one and come back later.
    \item \textbf{Office hour reassessments are given at the discretion of the instructor, they are not guaranteed opportunities.}   
    \item \textbf{You can only reassess one standard per day}.  Past experience has shown that students have a higher success rate by focusing on one standard for a reassessment, then moving to another after they have mastered it.  
    \item \textbf{There will be no office hours reassessments the same day as an exam.}  I ask you to instead wait and demonstrate mastery on the exam, so my office hour time can be devoted to answering questions helping everyone understand the material.
    \item If you come in for help on a standard, you should come back at a later time to reassess it after you have practiced it some more on your own.  But you may certainly ask for help on one standard, and then demonstrate mastery of a different one during the same visit.
    \item I am trying to gauge how many people read this far, and when. Please email me a picture of your favorite dinosaur when you read this. Also, please don't mention it to your classmates.
    \item \textbf{Reassessment opportunities may be limited by practical considerations like time limitations and fairness to your classmates, particularly towards the end of the semester.}  I do my best to accommodate everyone, but this is especially difficult at the end of the semester.  {\bf Students from previous semesters say the best thing you can do is to start reassessing early in the semester}.
%    \item I will try to schedule extra office hours during finals week, but \textbf{the last day for office hour reassessments will be two days before your final exam}.  As with the midterm, I want to be sure my office hours the day before the final are available to answer any questions or clear up any last difficulties.
    \item You can certainly demonstrate mastery of a standard for a second time in my office hours; however, to qualify it must occur in a subsequent week after you first mastered the standard.
    \end{itemize}
	\item Instead of coming to my virtual office hours, you can reassess by submitting a written or video explanation of a problem through \LMS.
    \end{enumerate}




\section*{\fontsize{12}{15}\selectfont Assessments}
There will be two kinds of in class assessments:
\begin{itemize}
\item As mentioned above, at the end of most weeks will be either a quiz or exam; barring unforseen events (e.g. weather days) there will be 10 quizzes and 4 exams.   The dates for each, including which standards will appear on which quizzes, is available in the Course Calendar in \LMS.
\item The {\bf final exam} times are listed on the University website and the course calendar. These times are set by the university and cannot be altered by individual instructors.
\end{itemize}

\section*{\fontsize{12}{15}\selectfont Calculator Policy}

Calculators of any sort may be used on exams provided that the calculator cannot make phone calls, send text messages, access the internet, or otherwise communicate with other devices.  A calculator that can perform row reduction of large matrices is highly recommended.

\section*{\fontsize{12}{15}\selectfont Missed Classes, Exams and Coursework}

As mentioned above, infrequent absence (e.g. due to intermittent illness) will not affect your course grade.  However, it is generally considered polite to send your instructor a short email when you know you will miss class.  

The midterm and final exams can only be made up in the event of illness (with a doctor's note), or other emergent situation (with appropriate documentation).  The definition of ``emergent'' is at the discretion of the instructor. Quizzes can only be made up if several in a row are missed due to an acceptable excuse as defined above.  Office hours reassessments can be used for additional opportunities subject to the restrictions detailed above.

Readiness Assurance Tests will not be made up. An unexcused absence will result in a 0 for both the iRAT and tRAT scores; otherwise the missed iRAT will be dropped and your team's tRAT will be counted for you.   

The above applies to the typical case of infrequent, intermittent absences. Should you have to be absent for an extended period of time for any reason, please come discuss with me as soon as possible so we can make alternate arrangements.

\noindent \begin{minipage}{\textwidth}
\section*{\fontsize{12}{15}\selectfont Grading}
At the end of the semester, your grade will be computed in the following manner.  \\

\begin{center}
 \begin{tabular}{l|l} 
To earn a  ... & ... you should \\
\hline
A & Earn 45 mastery checkmarks;  \\
\hline

B &  Earn 40 mastery checkmarks; \\
\hline

C 	&Earn 35 mastery checkmarks;\\
\hline

D & Earn 30 mastery checkmarks;\\
\hline

F 	& Not fit in the above categories. \\
\hline
\end{tabular}
\end{center}
\end{minipage}


\section*{\fontsize{12}{15}\selectfont Homework}
The only way to learn mathematics is to do mathematics; thus, our class time is centered around students doing mathematics.  Additionally, you will require practice outside of class; a list of suggested exercises corresponding to each standard is available in \LMS.  Your textbook also contains many more exercises.

I will neither collect nor grade homework, as past experience has not shown this to be a good use of instructor or student time.  Instead, you should work as many problems as you need to.  On your weekly self-assessment report (see below), you should indicate which problems you practiced. If you need feedback on your homework problems, bring them to my office hours and I will be happy to discuss them with you.


\section*{\fontsize{12}{15}\selectfont Student Academic Conduct Policy}
All students are expected to adhere to the Student Academic Conduct Policy, which you can view at
{\tt http://www.southalabama.edu/bulletin/current/student-affairs/conduct.html}.  Students violating this policy will be given one or more of the following penalties based on the severity of the offense:  1) Loss of all mastery checkmarks on all standards affected by the misconduct; 2) Reduction in final course grade by a letter grade; 3) Automatic course failure.



\end{document}
