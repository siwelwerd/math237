\documentclass[aspectratio=1610]{beamer}

\usetheme[hideothersubsections]{Hannover}
\usecolortheme{rose}
%Center frame titles
\setbeamertemplate{frametitle}[default][center]


%%%New Commands/Math operators%%%
\newcommand{\IR}{\mathbb{R}}
\newcommand{\IC}{\mathbb{C}}
\renewcommand{\P}{\mathcal{P}}
\renewcommand{\Im}{\operatorname{Im}}
\newcommand{\RREF}{\operatorname{RREF}}
\newcommand{\vspan}{\operatorname{span}}
\newcommand{\setList}[1]{\left\{#1\right\}}
\newcommand{\setBuilder}[2]{\left\{#1\,\middle|\,#2\right\}}




\begin{document}

\begin{frame}
\begin{center}
\Large Review
\end{center}
\end{frame}

\begin{frame}
Draw a flowchart connecting related standards in the class.
\end{frame}

\begin{frame}\frametitle{GT3}
Explain how to find the eigenvalues of the matrix \(\left[\begin{array}{cc}
7 & 1 \\
-18 & -4
\end{array}\right]\).
\end{frame}

\begin{frame}\frametitle{GT4}
Find a basis for the eigenspace associated to the eigenvalue $3$ in the matrix \[\begin{bmatrix} -7 & -8 & 2 \\ 8 & 9 & -1 \\ \frac{13}{2} & 5 & 2 \end{bmatrix}.\]
\end{frame}


\begin{frame}\frametitle{VS4}
Consider the following two sets of Euclidean vectors.
\[
  W = \setBuilder{\begin{bmatrix} x \\ y \\ z \end{bmatrix}}{x+y=z}
\hspace{3em}
  U = \setBuilder{\begin{bmatrix} x \\ y \\ z \end{bmatrix}}{x+y=z^2}
\]
Show that one of these sets is a subspace of \(\IR^3\), and
that one of the sets is not.

\end{frame}

\begin{frame}\frametitle{AT1}
Consider the following maps of polynomials \(S: \P \rightarrow \P\)
and \(T:\P\rightarrow\P\) defined by
\[S(f(x))= f(x)-3x \text{ and }T(f(x)) = f(x)-3f'(x).\]
Show that one of these maps is a linear transformation, and that the other
map is not.
\end{frame}


\begin{frame}
\frametitle{VS1}
Let \(V\) be the set of all pairs \((x,y)\) of real numbers together with the following operations: 

 \[(x_1,y_1)\oplus (x_2,y_2)=\left(x_{1} x_{2},\,y_{1} y_{2}\right)\]\[c \odot (x,y) =\left(x^{c},\,y^{c}\right).\] 


\begin{enumerate}[(a)]
\item Show that there exists an additive identity element, that is: 

 \[
      \text{There exists }(w,z)\in V\text{ such that }(x,y)\oplus(w,z)=(x,y).
    \] 
holds.
\item Show why \(V\) is not a vector space.
\end{enumerate}

\end{frame}


\begin{frame}\frametitle{GT2}
Find the determinant of the matrix
\[
  A
    =
  \begin{bmatrix}
    1 & 3 & 0 & -1 \\
    1 & 1 & 2 & 4 \\
    1 & 1 & 1 & 3 \\
    -3 & 1 & 2 & -5
  \end{bmatrix}
\]
\end{frame}

\begin{frame}\frametitle{MX2}
Let \(A\) be a \(4 \times 4\) matrix.

 

\begin{enumerate}[(a)]
\item Give a \(4 \times 4\) matrix \(C\) that may be used to perform the row operation \(R_4 \to R_4 + 5R_2\).
\item Give a \(4 \times 4\) matrix \(P\) that may be used to perform the row operation \(R_4 \to 3R_4\).
\item Use matrix multiplication to describe the matrix obtained by applying \(R_4 \to 3R_4\) and then \(R_4 \to R_4 + 5R_2\) to \(A\) (note the order). 
\end{enumerate}
\end{frame}

\begin{frame}\frametitle{GT1}
Let \(A\) be a \(4 \times 4\) matrix with determinant \(4\).

 

\begin{enumerate}[(a)]
\item Let \(M\) be the matrix obtained from \(A\) by applying the row operation \(R_3 \leftrightarrow R_2\). What is \(\mathrm{det}\,M\)?
\item Let \(P\) be the matrix obtained from \(A\) by applying the row operation \(R_1 \to 5R_1\). What is \(\mathrm{det}\,P\)?
\item Let \(C\) be the matrix obtained from \(A\) by applying the row operation \(R_1 \to R_1 + 5R_4\). What is \(\mathrm{det}\,C\)?
\end{enumerate}
\end{frame}

\begin{frame}\frametitle{VS8}

\begin{enumerate}[(a)]
\item  

 Given the set \[\left\{ x^{3} + 3 \, x^{2} + 1 , -x^{3} + x^{2} + x , -3 \, x^{3} - 2 \, x^{2} + 3 \, x - 2 , -6 \, x^{3} - 9 \, x^{2} + 6 \, x - 6 \right\}\] write a statement involving a polynomial equation that's equivalent to each claim below. 

 

\begin{itemize}
\item  

 The set of polynomials \textbf{spans} \(\mathcal{P}_3\) 

 
\item  

 The set of polynomials does \textbf{not} span \(\mathcal{P}_3\) 

 
\end{itemize}

     
\item  

 Explain how to determine which of these statements is true. 

 
\end{enumerate}
\end{frame}



\begin{frame}\frametitle{AT3}
Let \(T: \IR^4 \rightarrow \IR^3\) be the linear transformation given by
\[
  T\left(\begin{bmatrix}x\\y\\z\\w\end{bmatrix} \right) =
  \begin{bmatrix}
    x+3y+2z-3w \\
    2x+4y+6z-10w \\
    x+6y-z+3w
  \end{bmatrix}
\]
Compute a basis for the kernel and a basis for the image of \(T\).
\end{frame}




\end{document}