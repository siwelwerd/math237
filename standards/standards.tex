\documentclass{article}


\usepackage[left=0.75in,right=0.75in,top=1in,bottom=1in]{geometry}
\usepackage{enumerate,amssymb}

\newcommand{\module}{?}
\newcommand{\setModule}[1]{\renewcommand{\module}{#1}}
\newcommand{\moduleQuestion}[1]{
  \noindent\textbf{Module \module:} #1
}
\newenvironment{moduleStandards}{
  \begin{enumerate}[\bf\(\Box\) \(\Box\) \module{}1.]
  \setlength{\itemsep}{-0.02in}
}{
  \end{enumerate}
}
\newcommand{\standard}[3]{
  \item \textbf{#1.}
        I can#3
}

\newcommand{\IR}{\mathbb{R}}

\begin{document}
\pagestyle{empty}
\vspace{0.3in}
\hrule
\begin{center}\large \textbf{Linear Algebra Standards}\end{center}
\hrule
\vspace{0.1in}

\setModule{LE}
\begin{moduleStandards}
  \standard{Systems as matrices}{SysMat}{
    translate back and forth between a system of linear equations, 
	a vector equation, and the corresponding augmented matrix.
  }
  \standard{Row reduction}{Rref}{
    explain why a matrix isn't in reduced row echelon form, and put a matrix in reduced row echelon form.
  }
  \standard{Counting solutions of of linear systems}{SlvSys}{
    determine the number of solutions for a system of linear equations or a vector equation.
  }
    \standard{Solution sets of linear systems}{SlvSys}{
    compute the solution set for a system of linear equations or a vector equation with infinitely many solutions.
  }
\end{moduleStandards}
\setModule{VS}
\begin{moduleStandards}
  \standard{Vector spaces}{VecSp}{
    explain why a given set with defined addition and scalar multiplication does satisfy a given vector space property, but nonetheless isn't a vector space.
  }
  \standard{Linear combinations}{LinCmb}{
    determine if a Euclidean vector can be written as a linear combination of a given set of Euclidean vectors by solving an appropriate vector equation.
  }
  \standard{Spanning sets}{Span}{
    determine if a set of Euclidean vectors spans \(\IR^n\) by solving appropriate vector equations.
  }
  \standard{Subspaces}{Subsp}{
    determine if a subset of \(\IR^n\) is a subspace or not.
  }
  \standard{Linear independence}{LinInd}{
    determine if a set of Euclidean vectors is linearly dependent or
    independent by solving an appropriate vector equation.
  }
  \standard{Basis verification}{BasVer}{
    explain why a set of Euclidean vectors is or is not a basis of \(\IR^n\).
  }
  \standard{Basis computation}{BasCmp}{
    compute a basis for the subspace spanned by a given set of Euclidean
    vectors, and determine the dimension of the subspace.
  }
  \standard{Polynomial and Matrix computation}{PolyBasis}{
     answer questions about vector spaces of polynomials or matrices.
  }
  \standard{Basis of solution space}{BasSol}{
    find a basis for the solution set of a homogeneous system of equations.
  }
\end{moduleStandards}
\setModule{AT}
\begin{moduleStandards}
  \standard{Linear map verification}{LinVer}{
    determine if a map between vector spaces of polynomials is linear or not.
  }
  \standard{Linear maps and matrices}{LinMat}{
    translate back and forth between a
    linear transformation of Euclidean spaces and its standard matrix, and
    perform related computations.
  }
  \standard{Kernel and Image}{KerImg}{
    compute a basis for the kernel and a basis for the image of a linear map, and verify that the rank-nullity theorem holds for a given linear map.
  }
  \standard{Injectivity and surjectivity}{InjSrj}{
    determine if a given linear map is injective and/or surjective.
  }
\end{moduleStandards}
\setModule{MX}
\begin{moduleStandards}
  \standard{Matrix Multiplication}{MatMlt}{
    multiply matrices.
  }
  \standard{Row operations as matrix multiplication}{RowMult}{
    can express row operations through matrix multiplication.
  }
  \standard{Invertible Matrices}{InvVer}{
    determine if a square matrix is invertible or not, and if so, compute its inverse.
  }
\end{moduleStandards}
\setModule{GT}
\begin{moduleStandards}
  \standard{Row operations and Determinants}{RowOp}{
    describe how a row operation affects the determinant of a matrix. }
  \standard{Determinants}{Det}{
    compute the determinant of a \(4\times 4\) matrix.
  }
  \standard{Eigenvalues}{EigVal}{
    find the eigenvalues of a \(2\times 2\) matrix.
  }
  \standard{Eigenvectors}{EigVec}{
    find a basis for the eigenspace of a \(4\times 4\)  
    matrix associated with a given eigenvalue.
  }
\end{moduleStandards}
\end{document}