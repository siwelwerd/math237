\documentclass[aspectration=1610]{beamer}

\usetheme{Hannover}
\usecolortheme{rose}
\definecolor{JagBlue}{HTML}{00205B}
\definecolor{JagRed}{HTML}{BF0D3E}

\usepackage{tikz,pgfplots}

%Center frame titles
\setbeamertemplate{frametitle}[default][center]

\title{Linear Algebra}

\date{Spring 2020}


\begin{document}



\begin{frame}\frametitle{Interpreting Feedback}
On each assessment, for each standard you will receive one of the following marks.
\begin{itemize}
\item {\bf Demonstrated Excellence} means you demonstrated \textbf{Excellence} of that standard.
      Great job!  Check off another box on your progress sheet.
\item {\bf Minor Revision Needed} means you have demonstrated understanding, but made a minor mistake (e.g. arithmetic) or not written a complete solution.  Often, notation is poor or ambiguous.  You can upload a complete re-working of the same problem (fixing any errors) as a comment on the assessment in Canvas to demonstrate excellence.
\item {\bf Reassessment Needed} means you made a good faith effort and demonstrated
      partial understanding, but not complete understanding. You will need to attempt a new problem on a different assessment. 
\end{itemize}

\vspace{0.2in}

they don't hurt you either.
\end{frame}

\begin{frame}\frametitle{Assessment Opportunities}
Checkmarks may be earned as follows.
\begin{itemize}
\item {\bf Quizzes}: Most weeks, one day at the end of class we will have a quiz. 
\item {\bf Exams}: Periodically we will have longer assessments (usually on Friday).
\item {\bf On-Demand Reassessments}: One standard reassessments through Canvas
\end{itemize}

\pause

\vspace{0.2in}

The assessment method (quiz/exam/etc.) you used to earn a checkmark
isn't important: \textbf{I only care that you
learn the material and demonstrate that learning to me before the end of the
semester!}
\end{frame}




\begin{frame}\frametitle{Course Grades}

\begin{center}
\begin{tabular}{ll} \hline
A & Demonstrate excellence twice on 22 standards.\\ \hline
B &  Demonstrate excellence twice on 20 standards. \\ \hline
C &  Demonstrate excellence twice on 17 standards. \\ \hline
D &   Demonstrate excellence twice on 15 standards. \\ \hline
\end{tabular}
\end{center}

\end{frame}



\end{document}
