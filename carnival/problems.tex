
\begin{exercise}{VS1}{Vector spaces}{0003} 
\begin{exerciseStatement} 

 Let \(V\) be the set of all pairs \((x,y)\) of real numbers together with the following operations: 

 \[(x_1,y_1)\oplus (x_2,y_2)=\left(x_{1} + x_{2},\,y_{1} + y_{2} - 5\right)\]\[c \odot (x,y) =\left(c x,\,c y\right).\] 

 (a) Show that vector addition is associative, that is: 

 \[
      \left((x_1,y_1)\oplus(x_2,y_2)\right)\oplus(x_3,y_3)=(x_1,y_1)\oplus\left((x_2,y_2)\oplus(x_3,y_3)\right).
    \] 

 (b) Explain why \(V\) nonetheless is not a vector space. 

 \end{exerciseStatement}
 \begin{exerciseAnswer} 

 \(V\) is not a vector space, which may be shown by demonstrating that any one of the following properties do not hold: 

 

\begin{itemize}
\item scalar multiplication does not distribute over vector addition
\item scalar multiplication does not distribute over scalar addition
\end{itemize}

     \end{exerciseAnswer}
 \end{exercise}



\begin{exercise}{VS1}{Vector spaces}{0209} 
\begin{exerciseStatement} 

 Let \(V\) be the set of all pairs \((x,y)\) of real numbers together with the following operations: 

 \[(x_1,y_1)\oplus (x_2,y_2)=\left(4 \, x_{1} x_{2},\,y_{1} + 2 \, y_{2}\right)\]\[c \odot (x,y) =\left(c x,\,0\right).\] 

 (a) Show that there exists an additive identity element, that is: 

 \[
      \text{There exists }(w,z)\in V\text{ such that }(x,y)\oplus(w,z)=(x,y).
    \] 

 (b) Explain why \(V\) nonetheless is not a vector space. 

 \end{exerciseStatement}
 \begin{exerciseAnswer} 

 \(V\) is not a vector space, which may be shown by demonstrating that any one of the following properties do not hold: 

 

\begin{itemize}
\item vector addition is not associative
\item vector addition is not commutative
\item additive inverses do not always exist
\item 1 is not a scalar multiplication identity
\item scalar multiplication does not distribute over vector addition
\item scalar multiplication does not distribute over scalar addition
\end{itemize}

     \end{exerciseAnswer}
 \end{exercise}


\newpage




\begin{exercise}{VS1}{Vector spaces}{0123} 
\begin{exerciseStatement} 

 Let \(V\) be the set of all pairs \((x,y)\) of real numbers together with the following operations: 

 \[(x_1,y_1)\oplus (x_2,y_2)=\left(x_{1} + x_{2} + 3,\,y_{1} + y_{2}\right)\]\[c \odot (x,y) =\left(c x,\,y^{c}\right).\] 

 (a) Show that vector addition is associative, that is: 

 \[
      \left((x_1,y_1)\oplus(x_2,y_2)\right)\oplus(x_3,y_3)=(x_1,y_1)\oplus\left((x_2,y_2)\oplus(x_3,y_3)\right).
    \] 

 (b) Explain why \(V\) nonetheless is not a vector space. 

 \end{exerciseStatement}
 \begin{exerciseAnswer} 

 \(V\) is not a vector space, which may be shown by demonstrating that any one of the following properties do not hold: 

 

\begin{itemize}
\item scalar multiplication does not distribute over vector addition
\item scalar multiplication does not distribute over scalar addition
\end{itemize}

     \end{exerciseAnswer}
 \end{exercise}



\begin{exercise}{VS1}{Vector spaces}{0264} 
\begin{exerciseStatement} 

 Let \(V\) be the set of all pairs \((x,y)\) of real numbers together with the following operations: 

 \[(x_1,y_1)\oplus (x_2,y_2)=\left(3 \, x_{1} + x_{2},\,y_{1} + y_{2}\right)\]\[c \odot (x,y) =\left(c x,\,c y\right).\] 

 (a) Show that scalar multiplication distributes over vector addition, that is: 

 \[
      c\odot \left((x_1,y_1)\oplus(x_2,y_2)\right)=c\odot(x_1,y_1)\oplus c\odot(x_2,y_2).
    \] 

 (b) Explain why \(V\) nonetheless is not a vector space. 

 \end{exerciseStatement}
 \begin{exerciseAnswer} 

 \(V\) is not a vector space, which may be shown by demonstrating that any one of the following properties do not hold: 

 

\begin{itemize}
\item vector addition is not associative
\item vector addition is not commutative
\item scalar multiplication does not distribute over scalar addition
\end{itemize}

     \end{exerciseAnswer}
 \end{exercise}


\newpage




\begin{exercise}{VS1}{Vector spaces}{0203} 
\begin{exerciseStatement} 

 Let \(V\) be the set of all pairs \((x,y)\) of real numbers together with the following operations: 

 \[(x_1,y_1)\oplus (x_2,y_2)=\left(x_{1} + x_{2},\,y_{1} + y_{2}\right)\]\[c \odot (x,y) =\left(c x,\,c y - 7 \, c + 7\right).\] 

 (a) Show that scalar multiplication is associative, that is: 

 \[
      a\odot(b\odot (x,y))=(ab)\odot(x,y).
    \] 

 (b) Explain why \(V\) nonetheless is not a vector space. 

 \end{exerciseStatement}
 \begin{exerciseAnswer} 

 \(V\) is not a vector space, which may be shown by demonstrating that any one of the following properties do not hold: 

 

\begin{itemize}
\item scalar multiplication does not distribute over vector addition
\item scalar multiplication does not distribute over scalar addition
\end{itemize}

     \end{exerciseAnswer}
 \end{exercise}



\begin{exercise}{VS1}{Vector spaces}{0140} 
\begin{exerciseStatement} 

 Let \(V\) be the set of all pairs \((x,y)\) of real numbers together with the following operations: 

 \[(x_1,y_1)\oplus (x_2,y_2)=\left(2 \, x_{1} + 2 \, x_{2},\,5 \, y_{1} + 5 \, y_{2}\right)\]\[c \odot (x,y) =\left(c x,\,c y\right).\] 

 (a) Show that scalar multiplication distributes over vector addition, that is: 

 \[
      c\odot \left((x_1,y_1)\oplus(x_2,y_2)\right)=c\odot(x_1,y_1)\oplus c\odot(x_2,y_2).
    \] 

 (b) Explain why \(V\) nonetheless is not a vector space. 

 \end{exerciseStatement}
 \begin{exerciseAnswer} 

 \(V\) is not a vector space, which may be shown by demonstrating that any one of the following properties do not hold: 

 

\begin{itemize}
\item vector addition is not associative
\item scalar multiplication does not distribute over scalar addition
\end{itemize}

     \end{exerciseAnswer}
 \end{exercise}


\newpage




\begin{exercise}{VS1}{Vector spaces}{0270} 
\begin{exerciseStatement} 

 Let \(V\) be the set of all pairs \((x,y)\) of real numbers together with the following operations: 

 \[(x_1,y_1)\oplus (x_2,y_2)=\left(x_{1} + x_{2},\,y_{1} + y_{2} - 6\right)\]\[c \odot (x,y) =\left(c x,\,c y\right).\] 

 (a) Show that vector addition is associative, that is: 

 \[
      \left((x_1,y_1)\oplus(x_2,y_2)\right)\oplus(x_3,y_3)=(x_1,y_1)\oplus\left((x_2,y_2)\oplus(x_3,y_3)\right).
    \] 

 (b) Explain why \(V\) nonetheless is not a vector space. 

 \end{exerciseStatement}
 \begin{exerciseAnswer} 

 \(V\) is not a vector space, which may be shown by demonstrating that any one of the following properties do not hold: 

 

\begin{itemize}
\item scalar multiplication does not distribute over vector addition
\item scalar multiplication does not distribute over scalar addition
\end{itemize}

     \end{exerciseAnswer}
 \end{exercise}



\begin{exercise}{VS1}{Vector spaces}{0036} 
\begin{exerciseStatement} 

 Let \(V\) be the set of all pairs \((x,y)\) of real numbers together with the following operations: 

 \[(x_1,y_1)\oplus (x_2,y_2)=\left(x_{1} + x_{2},\,y_{1} + y_{2}\right)\]\[c \odot (x,y) =\left(4 \, c x,\,3 \, c y\right).\] 

 (a) Show that scalar multiplication distributes over scalar addition, that is: 

 \[
      (c+d)\odot(x,y)=c\odot(x,y)\oplus d\odot (x,y).
    \] 

 (b) Explain why \(V\) nonetheless is not a vector space. 

 \end{exerciseStatement}
 \begin{exerciseAnswer} 

 \(V\) is not a vector space, which may be shown by demonstrating that any one of the following properties do not hold: 

 

\begin{itemize}
\item scalar multiplication is not associative
\item 1 is not a scalar multiplication identity
\item scalar multiplication does not distribute over vector addition
\end{itemize}

     \end{exerciseAnswer}
 \end{exercise}


\newpage




\begin{exercise}{VS1}{Vector spaces}{0152} 
\begin{exerciseStatement} 

 Let \(V\) be the set of all pairs \((x,y)\) of real numbers together with the following operations: 

 \[(x_1,y_1)\oplus (x_2,y_2)=\left(x_{1} + x_{2},\,y_{1} + y_{2}\right)\]\[c \odot (x,y) =\left(c x,\,c y - 6 \, c + 6\right).\] 

 (a) Show that scalar multiplication is associative, that is: 

 \[
      a\odot(b\odot (x,y))=(ab)\odot(x,y).
    \] 

 (b) Explain why \(V\) nonetheless is not a vector space. 

 \end{exerciseStatement}
 \begin{exerciseAnswer} 

 \(V\) is not a vector space, which may be shown by demonstrating that any one of the following properties do not hold: 

 

\begin{itemize}
\item scalar multiplication does not distribute over vector addition
\item scalar multiplication does not distribute over scalar addition
\end{itemize}

     \end{exerciseAnswer}
 \end{exercise}



\begin{exercise}{VS1}{Vector spaces}{0135} 
\begin{exerciseStatement} 

 Let \(V\) be the set of all pairs \((x,y)\) of real numbers together with the following operations: 

 \[(x_1,y_1)\oplus (x_2,y_2)=\left(x_{1} + x_{2},\,y_{1} + y_{2}\right)\]\[c \odot (x,y) =\left(3 \, c x,\,4 \, c y\right).\] 

 (a) Show that scalar multiplication distributes over scalar addition, that is: 

 \[
      (c+d)\odot(x,y)=c\odot(x,y)\oplus d\odot (x,y).
    \] 

 (b) Explain why \(V\) nonetheless is not a vector space. 

 \end{exerciseStatement}
 \begin{exerciseAnswer} 

 \(V\) is not a vector space, which may be shown by demonstrating that any one of the following properties do not hold: 

 

\begin{itemize}
\item scalar multiplication is not associative
\item 1 is not a scalar multiplication identity
\item scalar multiplication does not distribute over vector addition
\end{itemize}

     \end{exerciseAnswer}
 \end{exercise}


\newpage




\begin{exercise}{VS2}{Linear combinations}{0278} 
\begin{exerciseStatement} 

\begin{enumerate}[(a)]
\item  

 Write a statement involving the solutions of a vector equation that's equivalent to each claim below. 

 

\begin{itemize}
\item  

 \(\left[\begin{array}{c}
1 \\
-2 \\
-1
\end{array}\right]\)is a linear combination of the vectors \(\left[\begin{array}{c}
1 \\
-1 \\
2
\end{array}\right] , \left[\begin{array}{c}
-2 \\
2 \\
-4
\end{array}\right] , \left[\begin{array}{c}
4 \\
-3 \\
4
\end{array}\right] , \text{ and } \left[\begin{array}{c}
5 \\
-4 \\
6
\end{array}\right]\). 

 
\item  

 \(\left[\begin{array}{c}
2 \\
-1 \\
0
\end{array}\right]\)is a linear combination of the vectors \(\left[\begin{array}{c}
1 \\
-1 \\
2
\end{array}\right] , \left[\begin{array}{c}
-2 \\
2 \\
-4
\end{array}\right] , \left[\begin{array}{c}
4 \\
-3 \\
4
\end{array}\right] , \text{ and } \left[\begin{array}{c}
5 \\
-4 \\
6
\end{array}\right]\). 

 
\end{itemize}

     
\item  

 Use these statements to determine if each vector is or is not a linear combination. If it is, give an example of such a linear combination. 

 
\end{enumerate}

     \end{exerciseStatement}
 \begin{exerciseAnswer} 

\begin{itemize}
\item  

 \(
\mathrm{RREF}\, \left[\begin{array}{cccc|c}
1 & -2 & 4 & 5 & 1 \\
-1 & 2 & -3 & -4 & -2 \\
2 & -4 & 4 & 6 & -1
\end{array}\right] = \left[\begin{array}{cccc|c}
1 & -2 & 0 & 1 & 0 \\
0 & 0 & 1 & 1 & 0 \\
0 & 0 & 0 & 0 & 1
\end{array}\right]
                        \) 

 

 \(\left[\begin{array}{c}
1 \\
-2 \\
-1
\end{array}\right]\) is not a linear combination. 

 
\item  

 \(
\mathrm{RREF}\, \left[\begin{array}{cccc|c}
1 & -2 & 4 & 5 & 2 \\
-1 & 2 & -3 & -4 & -1 \\
2 & -4 & 4 & 6 & 0
\end{array}\right] = \left[\begin{array}{cccc|c}
1 & -2 & 0 & 1 & -2 \\
0 & 0 & 1 & 1 & 1 \\
0 & 0 & 0 & 0 & 0
\end{array}\right]
                        \) 

 

 \(\left[\begin{array}{c}
2 \\
-1 \\
0
\end{array}\right]\) is a linear combination, for example: \(
-2 \left[\begin{array}{c}
1 \\
-1 \\
2
\end{array}\right] + 1 \left[\begin{array}{c}
4 \\
-3 \\
4
\end{array}\right] = \left[\begin{array}{c}
2 \\
-1 \\
0
\end{array}\right]
                            \) 

 
\end{itemize}

     \end{exerciseAnswer}
 \end{exercise}



\begin{exercise}{VS2}{Linear combinations}{0254} 
\begin{exerciseStatement} 

\begin{enumerate}[(a)]
\item  

 Write a statement involving the solutions of a vector equation that's equivalent to each claim below. 

 

\begin{itemize}
\item  

 \(\left[\begin{array}{c}
13 \\
-4 \\
13 \\
-19
\end{array}\right]\)is a linear combination of the vectors \(\left[\begin{array}{c}
-4 \\
1 \\
-4 \\
5
\end{array}\right] , \left[\begin{array}{c}
-2 \\
2 \\
0 \\
3
\end{array}\right] , \text{ and } \left[\begin{array}{c}
4 \\
2 \\
8 \\
-4
\end{array}\right]\). 

 
\item  

 \(\left[\begin{array}{c}
14 \\
-5 \\
12 \\
-18
\end{array}\right]\)is a linear combination of the vectors \(\left[\begin{array}{c}
-4 \\
1 \\
-4 \\
5
\end{array}\right] , \left[\begin{array}{c}
-2 \\
2 \\
0 \\
3
\end{array}\right] , \text{ and } \left[\begin{array}{c}
4 \\
2 \\
8 \\
-4
\end{array}\right]\). 

 
\end{itemize}

     
\item  

 Use these statements to determine if each vector is or is not a linear combination. If it is, give an example of such a linear combination. 

 
\end{enumerate}

     \end{exerciseStatement}
 \begin{exerciseAnswer} 

\begin{itemize}
\item  

 \(
\mathrm{RREF}\, \left[\begin{array}{ccc|c}
-4 & -2 & 4 & 13 \\
1 & 2 & 2 & -4 \\
-4 & 0 & 8 & 13 \\
5 & 3 & -4 & -19
\end{array}\right] = \left[\begin{array}{ccc|c}
1 & 0 & -2 & 0 \\
0 & 1 & 2 & 0 \\
0 & 0 & 0 & 1 \\
0 & 0 & 0 & 0
\end{array}\right]
                        \) 

 

 \(\left[\begin{array}{c}
13 \\
-4 \\
13 \\
-19
\end{array}\right]\) is not a linear combination. 

 
\item  

 \(
\mathrm{RREF}\, \left[\begin{array}{ccc|c}
-4 & -2 & 4 & 14 \\
1 & 2 & 2 & -5 \\
-4 & 0 & 8 & 12 \\
5 & 3 & -4 & -18
\end{array}\right] = \left[\begin{array}{ccc|c}
1 & 0 & -2 & -3 \\
0 & 1 & 2 & -1 \\
0 & 0 & 0 & 0 \\
0 & 0 & 0 & 0
\end{array}\right]
                        \) 

 

 \(\left[\begin{array}{c}
14 \\
-5 \\
12 \\
-18
\end{array}\right]\) is a linear combination, for example: \(
-3 \left[\begin{array}{c}
-4 \\
1 \\
-4 \\
5
\end{array}\right] + -1 \left[\begin{array}{c}
-2 \\
2 \\
0 \\
3
\end{array}\right] = \left[\begin{array}{c}
14 \\
-5 \\
12 \\
-18
\end{array}\right]
                            \) 

 
\end{itemize}

     \end{exerciseAnswer}
 \end{exercise}


\newpage




\begin{exercise}{VS2}{Linear combinations}{0098} 
\begin{exerciseStatement} 

\begin{enumerate}[(a)]
\item  

 Write a statement involving the solutions of a vector equation that's equivalent to each claim below. 

 

\begin{itemize}
\item  

 \(\left[\begin{array}{c}
0 \\
0 \\
3
\end{array}\right]\)is a linear combination of the vectors \(\left[\begin{array}{c}
1 \\
2 \\
-4
\end{array}\right] , \left[\begin{array}{c}
-3 \\
-6 \\
12
\end{array}\right] , \left[\begin{array}{c}
1 \\
3 \\
-4
\end{array}\right] , \text{ and } \left[\begin{array}{c}
2 \\
5 \\
-8
\end{array}\right]\). 

 
\item  

 \(\left[\begin{array}{c}
-1 \\
-1 \\
4
\end{array}\right]\)is a linear combination of the vectors \(\left[\begin{array}{c}
1 \\
2 \\
-4
\end{array}\right] , \left[\begin{array}{c}
-3 \\
-6 \\
12
\end{array}\right] , \left[\begin{array}{c}
1 \\
3 \\
-4
\end{array}\right] , \text{ and } \left[\begin{array}{c}
2 \\
5 \\
-8
\end{array}\right]\). 

 
\end{itemize}

     
\item  

 Use these statements to determine if each vector is or is not a linear combination. If it is, give an example of such a linear combination. 

 
\end{enumerate}

     \end{exerciseStatement}
 \begin{exerciseAnswer} 

\begin{itemize}
\item  

 \(
\mathrm{RREF}\, \left[\begin{array}{cccc|c}
1 & -3 & 1 & 2 & 0 \\
2 & -6 & 3 & 5 & 0 \\
-4 & 12 & -4 & -8 & 3
\end{array}\right] = \left[\begin{array}{cccc|c}
1 & -3 & 0 & 1 & 0 \\
0 & 0 & 1 & 1 & 0 \\
0 & 0 & 0 & 0 & 1
\end{array}\right]
                        \) 

 

 \(\left[\begin{array}{c}
0 \\
0 \\
3
\end{array}\right]\) is not a linear combination. 

 
\item  

 \(
\mathrm{RREF}\, \left[\begin{array}{cccc|c}
1 & -3 & 1 & 2 & -1 \\
2 & -6 & 3 & 5 & -1 \\
-4 & 12 & -4 & -8 & 4
\end{array}\right] = \left[\begin{array}{cccc|c}
1 & -3 & 0 & 1 & -2 \\
0 & 0 & 1 & 1 & 1 \\
0 & 0 & 0 & 0 & 0
\end{array}\right]
                        \) 

 

 \(\left[\begin{array}{c}
-1 \\
-1 \\
4
\end{array}\right]\) is a linear combination, for example: \(
-2 \left[\begin{array}{c}
1 \\
2 \\
-4
\end{array}\right] + 1 \left[\begin{array}{c}
1 \\
3 \\
-4
\end{array}\right] = \left[\begin{array}{c}
-1 \\
-1 \\
4
\end{array}\right]
                            \) 

 
\end{itemize}

     \end{exerciseAnswer}
 \end{exercise}



\begin{exercise}{VS2}{Linear combinations}{0257} 
\begin{exerciseStatement} 

\begin{enumerate}[(a)]
\item  

 Write a statement involving the solutions of a vector equation that's equivalent to each claim below. 

 

\begin{itemize}
\item  

 \(\left[\begin{array}{c}
-1 \\
2 \\
-5
\end{array}\right]\)is a linear combination of the vectors \(\left[\begin{array}{c}
-2 \\
3 \\
0
\end{array}\right] , \left[\begin{array}{c}
3 \\
-5 \\
5
\end{array}\right] , \left[\begin{array}{c}
0 \\
-1 \\
10
\end{array}\right] , \text{ and } \left[\begin{array}{c}
5 \\
-9 \\
15
\end{array}\right]\). 

 
\item  

 \(\left[\begin{array}{c}
0 \\
1 \\
-4
\end{array}\right]\)is a linear combination of the vectors \(\left[\begin{array}{c}
-2 \\
3 \\
0
\end{array}\right] , \left[\begin{array}{c}
3 \\
-5 \\
5
\end{array}\right] , \left[\begin{array}{c}
0 \\
-1 \\
10
\end{array}\right] , \text{ and } \left[\begin{array}{c}
5 \\
-9 \\
15
\end{array}\right]\). 

 
\end{itemize}

     
\item  

 Use these statements to determine if each vector is or is not a linear combination. If it is, give an example of such a linear combination. 

 
\end{enumerate}

     \end{exerciseStatement}
 \begin{exerciseAnswer} 

\begin{itemize}
\item  

 \(
\mathrm{RREF}\, \left[\begin{array}{cccc|c}
-2 & 3 & 0 & 5 & -1 \\
3 & -5 & -1 & -9 & 2 \\
0 & 5 & 10 & 15 & -5
\end{array}\right] = \left[\begin{array}{cccc|c}
1 & 0 & 3 & 2 & -1 \\
0 & 1 & 2 & 3 & -1 \\
0 & 0 & 0 & 0 & 0
\end{array}\right]
                        \) 

 

 \(\left[\begin{array}{c}
-1 \\
2 \\
-5
\end{array}\right]\) is a linear combination, for example: \(
-1 \left[\begin{array}{c}
-2 \\
3 \\
0
\end{array}\right] + -1 \left[\begin{array}{c}
3 \\
-5 \\
5
\end{array}\right] = \left[\begin{array}{c}
-1 \\
2 \\
-5
\end{array}\right]
                            \) 

 
\item  

 \(
\mathrm{RREF}\, \left[\begin{array}{cccc|c}
-2 & 3 & 0 & 5 & 0 \\
3 & -5 & -1 & -9 & 1 \\
0 & 5 & 10 & 15 & -4
\end{array}\right] = \left[\begin{array}{cccc|c}
1 & 0 & 3 & 2 & 0 \\
0 & 1 & 2 & 3 & 0 \\
0 & 0 & 0 & 0 & 1
\end{array}\right]
                        \) 

 

 \(\left[\begin{array}{c}
0 \\
1 \\
-4
\end{array}\right]\) is not a linear combination. 

 
\end{itemize}

     \end{exerciseAnswer}
 \end{exercise}


\newpage




\begin{exercise}{VS3}{Spanning sets}{0011} 
\begin{exerciseStatement} 

\begin{enumerate}[(a)]
\item  

 Write a statement involving the solutions of a vector equation that's equivalent to each claim below. 

 

\begin{itemize}
\item  

 The set of vectors \(\left\{ \left[\begin{array}{c}
1 \\
0 \\
0 \\
5
\end{array}\right] , \left[\begin{array}{c}
0 \\
1 \\
-2 \\
1
\end{array}\right] , \left[\begin{array}{c}
-1 \\
0 \\
1 \\
-4
\end{array}\right] , \left[\begin{array}{c}
0 \\
-3 \\
3 \\
-5
\end{array}\right] \right\}\) \textbf{spans} \(\mathbb R^4\). 

 
\item  

 The set of vectors \(\left\{ \left[\begin{array}{c}
1 \\
0 \\
0 \\
5
\end{array}\right] , \left[\begin{array}{c}
0 \\
1 \\
-2 \\
1
\end{array}\right] , \left[\begin{array}{c}
-1 \\
0 \\
1 \\
-4
\end{array}\right] , \left[\begin{array}{c}
0 \\
-3 \\
3 \\
-5
\end{array}\right] \right\}\) does \textbf{not} span \(\mathbb R^4\). 

 
\end{itemize}

     
\item  

 Explain how to determine which of these statements is true. 

 
\end{enumerate}

     \end{exerciseStatement}
 \begin{exerciseAnswer} 

 \[
\mathrm{RREF}\, \left[\begin{array}{cccc}
1 & 0 & -1 & 0 \\
0 & 1 & 0 & -3 \\
0 & -2 & 1 & 3 \\
5 & 1 & -4 & -5
\end{array}\right] = \left[\begin{array}{cccc}
1 & 0 & 0 & 0 \\
0 & 1 & 0 & 0 \\
0 & 0 & 1 & 0 \\
0 & 0 & 0 & 1
\end{array}\right]
            \] 

 

 The set of vectors \(\left\{ \left[\begin{array}{c}
1 \\
0 \\
0 \\
5
\end{array}\right] , \left[\begin{array}{c}
0 \\
1 \\
-2 \\
1
\end{array}\right] , \left[\begin{array}{c}
-1 \\
0 \\
1 \\
-4
\end{array}\right] , \left[\begin{array}{c}
0 \\
-3 \\
3 \\
-5
\end{array}\right] \right\}\) \textbf{spans} \(\mathbb{R}^4\). 

 \end{exerciseAnswer}
 \end{exercise}



\begin{exercise}{VS3}{Spanning sets}{0260} 
\begin{exerciseStatement} 

\begin{enumerate}[(a)]
\item  

 Write a statement involving the solutions of a vector equation that's equivalent to each claim below. 

 

\begin{itemize}
\item  

 The set of vectors \(\left\{ \left[\begin{array}{c}
-3 \\
-5 \\
4 \\
-4
\end{array}\right] , \left[\begin{array}{c}
2 \\
1 \\
-2 \\
-2
\end{array}\right] , \left[\begin{array}{c}
15 \\
18 \\
-18 \\
6
\end{array}\right] , \left[\begin{array}{c}
0 \\
-7 \\
2 \\
-14
\end{array}\right] , \left[\begin{array}{c}
-8 \\
-11 \\
10 \\
-6
\end{array}\right] \right\}\) \textbf{spans} \(\mathbb R^4\). 

 
\item  

 The set of vectors \(\left\{ \left[\begin{array}{c}
-3 \\
-5 \\
4 \\
-4
\end{array}\right] , \left[\begin{array}{c}
2 \\
1 \\
-2 \\
-2
\end{array}\right] , \left[\begin{array}{c}
15 \\
18 \\
-18 \\
6
\end{array}\right] , \left[\begin{array}{c}
0 \\
-7 \\
2 \\
-14
\end{array}\right] , \left[\begin{array}{c}
-8 \\
-11 \\
10 \\
-6
\end{array}\right] \right\}\) does \textbf{not} span \(\mathbb R^4\). 

 
\end{itemize}

     
\item  

 Explain how to determine which of these statements is true. 

 
\end{enumerate}

     \end{exerciseStatement}
 \begin{exerciseAnswer} 

 \[
\mathrm{RREF}\, \left[\begin{array}{ccccc}
-3 & 2 & 15 & 0 & -8 \\
-5 & 1 & 18 & -7 & -11 \\
4 & -2 & -18 & 2 & 10 \\
-4 & -2 & 6 & -14 & -6
\end{array}\right] = \left[\begin{array}{ccccc}
1 & 0 & -3 & 2 & 2 \\
0 & 1 & 3 & 3 & -1 \\
0 & 0 & 0 & 0 & 0 \\
0 & 0 & 0 & 0 & 0
\end{array}\right]
            \] 

 

 The set of vectors \(\left\{ \left[\begin{array}{c}
-3 \\
-5 \\
4 \\
-4
\end{array}\right] , \left[\begin{array}{c}
2 \\
1 \\
-2 \\
-2
\end{array}\right] , \left[\begin{array}{c}
15 \\
18 \\
-18 \\
6
\end{array}\right] , \left[\begin{array}{c}
0 \\
-7 \\
2 \\
-14
\end{array}\right] , \left[\begin{array}{c}
-8 \\
-11 \\
10 \\
-6
\end{array}\right] \right\}\) does \textbf{not} span \(\mathbb{R}^4\). 

 \end{exerciseAnswer}
 \end{exercise}


\newpage




\begin{exercise}{VS3}{Spanning sets}{0158} 
\begin{exerciseStatement} 

\begin{enumerate}[(a)]
\item  

 Write a statement involving the solutions of a vector equation that's equivalent to each claim below. 

 

\begin{itemize}
\item  

 The set of vectors \(\left\{ \left[\begin{array}{c}
5 \\
2 \\
4 \\
0
\end{array}\right] , \left[\begin{array}{c}
2 \\
1 \\
1 \\
1
\end{array}\right] , \left[\begin{array}{c}
5 \\
2 \\
5 \\
1
\end{array}\right] , \left[\begin{array}{c}
5 \\
3 \\
-1 \\
4
\end{array}\right] \right\}\) \textbf{spans} \(\mathbb R^4\). 

 
\item  

 The set of vectors \(\left\{ \left[\begin{array}{c}
5 \\
2 \\
4 \\
0
\end{array}\right] , \left[\begin{array}{c}
2 \\
1 \\
1 \\
1
\end{array}\right] , \left[\begin{array}{c}
5 \\
2 \\
5 \\
1
\end{array}\right] , \left[\begin{array}{c}
5 \\
3 \\
-1 \\
4
\end{array}\right] \right\}\) does \textbf{not} span \(\mathbb R^4\). 

 
\end{itemize}

     
\item  

 Explain how to determine which of these statements is true. 

 
\end{enumerate}

     \end{exerciseStatement}
 \begin{exerciseAnswer} 

 \[
\mathrm{RREF}\, \left[\begin{array}{cccc}
5 & 2 & 5 & 5 \\
2 & 1 & 2 & 3 \\
4 & 1 & 5 & -1 \\
0 & 1 & 1 & 4
\end{array}\right] = \left[\begin{array}{cccc}
1 & 0 & 0 & 0 \\
0 & 1 & 0 & 0 \\
0 & 0 & 1 & 0 \\
0 & 0 & 0 & 1
\end{array}\right]
            \] 

 

 The set of vectors \(\left\{ \left[\begin{array}{c}
5 \\
2 \\
4 \\
0
\end{array}\right] , \left[\begin{array}{c}
2 \\
1 \\
1 \\
1
\end{array}\right] , \left[\begin{array}{c}
5 \\
2 \\
5 \\
1
\end{array}\right] , \left[\begin{array}{c}
5 \\
3 \\
-1 \\
4
\end{array}\right] \right\}\) \textbf{spans} \(\mathbb{R}^4\). 

 \end{exerciseAnswer}
 \end{exercise}



\begin{exercise}{VS3}{Spanning sets}{0173} 
\begin{exerciseStatement} 

\begin{enumerate}[(a)]
\item  

 Write a statement involving the solutions of a vector equation that's equivalent to each claim below. 

 

\begin{itemize}
\item  

 The set of vectors \(\left\{ \left[\begin{array}{c}
1 \\
1 \\
0 \\
0
\end{array}\right] , \left[\begin{array}{c}
-1 \\
0 \\
2 \\
1
\end{array}\right] , \left[\begin{array}{c}
-5 \\
-4 \\
3 \\
0
\end{array}\right] , \left[\begin{array}{c}
2 \\
4 \\
3 \\
4
\end{array}\right] , \left[\begin{array}{c}
-4 \\
-11 \\
-11 \\
-13
\end{array}\right] \right\}\) \textbf{spans} \(\mathbb R^4\). 

 
\item  

 The set of vectors \(\left\{ \left[\begin{array}{c}
1 \\
1 \\
0 \\
0
\end{array}\right] , \left[\begin{array}{c}
-1 \\
0 \\
2 \\
1
\end{array}\right] , \left[\begin{array}{c}
-5 \\
-4 \\
3 \\
0
\end{array}\right] , \left[\begin{array}{c}
2 \\
4 \\
3 \\
4
\end{array}\right] , \left[\begin{array}{c}
-4 \\
-11 \\
-11 \\
-13
\end{array}\right] \right\}\) does \textbf{not} span \(\mathbb R^4\). 

 
\end{itemize}

     
\item  

 Explain how to determine which of these statements is true. 

 
\end{enumerate}

     \end{exerciseStatement}
 \begin{exerciseAnswer} 

 \[
\mathrm{RREF}\, \left[\begin{array}{ccccc}
1 & -1 & -5 & 2 & -4 \\
1 & 0 & -4 & 4 & -11 \\
0 & 2 & 3 & 3 & -11 \\
0 & 1 & 0 & 4 & -13
\end{array}\right] = \left[\begin{array}{ccccc}
1 & 0 & 0 & 0 & 1 \\
0 & 1 & 0 & 0 & -1 \\
0 & 0 & 1 & 0 & 0 \\
0 & 0 & 0 & 1 & -3
\end{array}\right]
            \] 

 

 The set of vectors \(\left\{ \left[\begin{array}{c}
1 \\
1 \\
0 \\
0
\end{array}\right] , \left[\begin{array}{c}
-1 \\
0 \\
2 \\
1
\end{array}\right] , \left[\begin{array}{c}
-5 \\
-4 \\
3 \\
0
\end{array}\right] , \left[\begin{array}{c}
2 \\
4 \\
3 \\
4
\end{array}\right] , \left[\begin{array}{c}
-4 \\
-11 \\
-11 \\
-13
\end{array}\right] \right\}\) \textbf{spans} \(\mathbb{R}^4\). 

 \end{exerciseAnswer}
 \end{exercise}


\newpage




\begin{exercise}{VS3}{Spanning sets}{0085} 
\begin{exerciseStatement} 

\begin{enumerate}[(a)]
\item  

 Write a statement involving the solutions of a vector equation that's equivalent to each claim below. 

 

\begin{itemize}
\item  

 The set of vectors \(\left\{ \left[\begin{array}{c}
-1 \\
0 \\
1 \\
0
\end{array}\right] , \left[\begin{array}{c}
3 \\
1 \\
-2 \\
0
\end{array}\right] , \left[\begin{array}{c}
-2 \\
0 \\
1 \\
0
\end{array}\right] , \left[\begin{array}{c}
3 \\
5 \\
-3 \\
1
\end{array}\right] \right\}\) \textbf{spans} \(\mathbb R^4\). 

 
\item  

 The set of vectors \(\left\{ \left[\begin{array}{c}
-1 \\
0 \\
1 \\
0
\end{array}\right] , \left[\begin{array}{c}
3 \\
1 \\
-2 \\
0
\end{array}\right] , \left[\begin{array}{c}
-2 \\
0 \\
1 \\
0
\end{array}\right] , \left[\begin{array}{c}
3 \\
5 \\
-3 \\
1
\end{array}\right] \right\}\) does \textbf{not} span \(\mathbb R^4\). 

 
\end{itemize}

     
\item  

 Explain how to determine which of these statements is true. 

 
\end{enumerate}

     \end{exerciseStatement}
 \begin{exerciseAnswer} 

 \[
\mathrm{RREF}\, \left[\begin{array}{cccc}
-1 & 3 & -2 & 3 \\
0 & 1 & 0 & 5 \\
1 & -2 & 1 & -3 \\
0 & 0 & 0 & 1
\end{array}\right] = \left[\begin{array}{cccc}
1 & 0 & 0 & 0 \\
0 & 1 & 0 & 0 \\
0 & 0 & 1 & 0 \\
0 & 0 & 0 & 1
\end{array}\right]
            \] 

 

 The set of vectors \(\left\{ \left[\begin{array}{c}
-1 \\
0 \\
1 \\
0
\end{array}\right] , \left[\begin{array}{c}
3 \\
1 \\
-2 \\
0
\end{array}\right] , \left[\begin{array}{c}
-2 \\
0 \\
1 \\
0
\end{array}\right] , \left[\begin{array}{c}
3 \\
5 \\
-3 \\
1
\end{array}\right] \right\}\) \textbf{spans} \(\mathbb{R}^4\). 

 \end{exerciseAnswer}
 \end{exercise}



\begin{exercise}{VS4}{Subspaces}{0092} 
\begin{exerciseStatement} 

Consider the following two sets of Euclidean vectors: 

 \[
          U=\left\{ \left[\begin{array}{c}
x \\
y
\end{array}\right] \middle|\,7 \, x + 6 \, y = 0\right\} \hspace{2em}  W=\left\{ \left[\begin{array}{c}
x \\
y
\end{array}\right] \middle|\,7 \, x y^{2} = 0\right\}
    \] 

 Explain why one of these sets is a subspace of \(\mathbb{R}^2\) and one is not. 

 \end{exerciseStatement}
 \begin{exerciseAnswer} 

\(U\) is a subspace of \(\mathbb{R}^2\) and \(W\) is not.

 \end{exerciseAnswer}
 \end{exercise}


\newpage




\begin{exercise}{VS4}{Subspaces}{0136} 
\begin{exerciseStatement} 

Consider the following two sets of Euclidean vectors: 

 \[
          U=\left\{ \left[\begin{array}{c}
x \\
y \\
z \\
w
\end{array}\right] \middle|\,2 \, w + 3 \, x = 5 \, z\right\} \hspace{2em}  W=\left\{ \left[\begin{array}{c}
x \\
y \\
z \\
w
\end{array}\right] \middle|\,5 \, x = z^{2} - 2 \, w + 2 \, y\right\}
    \] 

 Explain why one of these sets is a subspace of \(\mathbb{R}^4\) and one is not. 

 \end{exerciseStatement}
 \begin{exerciseAnswer} 

\(U\) is a subspace of \(\mathbb{R}^4\) and \(W\) is not.

 \end{exerciseAnswer}
 \end{exercise}



\begin{exercise}{VS4}{Subspaces}{0098} 
\begin{exerciseStatement} 

Consider the following two sets of Euclidean vectors: 

 \[
          U=\left\{ \left[\begin{array}{c}
x \\
y
\end{array}\right] \middle|\,5 \, x = 3 \, y\right\} \hspace{2em}  W=\left\{ \left[\begin{array}{c}
x \\
y
\end{array}\right] \middle|\,x^{2} + 5 \, y = 0\right\}
    \] 

 Explain why one of these sets is a subspace of \(\mathbb{R}^2\) and one is not. 

 \end{exerciseStatement}
 \begin{exerciseAnswer} 

\(U\) is a subspace of \(\mathbb{R}^2\) and \(W\) is not.

 \end{exerciseAnswer}
 \end{exercise}


\newpage




\begin{exercise}{VS4}{Subspaces}{0280} 
\begin{exerciseStatement} 

Consider the following two sets of Euclidean vectors: 

 \[
          U=\left\{ \left[\begin{array}{c}
x \\
y \\
z
\end{array}\right] \middle|\,x^{3} + 3 \, y + 5 \, z = 0\right\} \hspace{2em}  W=\left\{ \left[\begin{array}{c}
x \\
y \\
z
\end{array}\right] \middle|\,4 \, x + 5 \, y = 5 \, z\right\}
    \] 

 Explain why one of these sets is a subspace of \(\mathbb{R}^3\) and one is not. 

 \end{exerciseStatement}
 \begin{exerciseAnswer} 

\(W\) is a subspace of \(\mathbb{R}^3\) and \(U\) is not.

 \end{exerciseAnswer}
 \end{exercise}



\begin{exercise}{VS4}{Subspaces}{0078} 
\begin{exerciseStatement} 

Consider the following two sets of Euclidean vectors: 

 \[
          U=\left\{ \left[\begin{array}{c}
x \\
y
\end{array}\right] \middle|\,3 \, x y^{2} = 0\right\} \hspace{2em}  W=\left\{ \left[\begin{array}{c}
x \\
y
\end{array}\right] \middle|\,5 \, x + 4 \, y = 0\right\}
    \] 

 Explain why one of these sets is a subspace of \(\mathbb{R}^2\) and one is not. 

 \end{exerciseStatement}
 \begin{exerciseAnswer} 

\(W\) is a subspace of \(\mathbb{R}^2\) and \(U\) is not.

 \end{exerciseAnswer}
 \end{exercise}


\newpage




\begin{exercise}{VS4}{Subspaces}{0283} 
\begin{exerciseStatement} 

Consider the following two sets of Euclidean vectors: 

 \[
          U=\left\{ \left[\begin{array}{c}
x \\
y
\end{array}\right] \middle|\,3 \, x = 3 \, y\right\} \hspace{2em}  W=\left\{ \left[\begin{array}{c}
x \\
y
\end{array}\right] \middle|\,6 \, x y^{2} = 0\right\}
    \] 

 Explain why one of these sets is a subspace of \(\mathbb{R}^2\) and one is not. 

 \end{exerciseStatement}
 \begin{exerciseAnswer} 

\(U\) is a subspace of \(\mathbb{R}^2\) and \(W\) is not.

 \end{exerciseAnswer}
 \end{exercise}



\begin{exercise}{VS4}{Subspaces}{0152} 
\begin{exerciseStatement} 

Consider the following two sets of Euclidean vectors: 

 \[
          U=\left\{ \left[\begin{array}{c}
x \\
y \\
z \\
w
\end{array}\right] \middle|\,-6 \, w + 6 \, y = 4 \, x - 4 \, z\right\} \hspace{2em}  W=\left\{ \left[\begin{array}{c}
x \\
y \\
z \\
w
\end{array}\right] \middle|\,x^{3} + 4 \, y + 4 \, z = 2 \, w\right\}
    \] 

 Explain why one of these sets is a subspace of \(\mathbb{R}^4\) and one is not. 

 \end{exerciseStatement}
 \begin{exerciseAnswer} 

\(U\) is a subspace of \(\mathbb{R}^4\) and \(W\) is not.

 \end{exerciseAnswer}
 \end{exercise}


\newpage




\begin{exercise}{VS4}{Subspaces}{0175} 
\begin{exerciseStatement} 

Consider the following two sets of Euclidean vectors: 

 \[
          U=\left\{ \left[\begin{array}{c}
x \\
y \\
z \\
w
\end{array}\right] \middle|\,x + 4 \, y + 2 \, z = 5 \, w\right\} \hspace{2em}  W=\left\{ \left[\begin{array}{c}
x \\
y \\
z \\
w
\end{array}\right] \middle|\,6 \, x = z^{2} - 5 \, w + 5 \, y\right\}
    \] 

 Explain why one of these sets is a subspace of \(\mathbb{R}^4\) and one is not. 

 \end{exerciseStatement}
 \begin{exerciseAnswer} 

\(U\) is a subspace of \(\mathbb{R}^4\) and \(W\) is not.

 \end{exerciseAnswer}
 \end{exercise}



\begin{exercise}{VS4}{Subspaces}{0278} 
\begin{exerciseStatement} 

Consider the following two sets of Euclidean vectors: 

 \[
          U=\left\{ \left[\begin{array}{c}
x \\
y
\end{array}\right] \middle|\,5 \, x y^{2} = 0\right\} \hspace{2em}  W=\left\{ \left[\begin{array}{c}
x \\
y
\end{array}\right] \middle|\,7 \, x + 6 \, y = 0\right\}
    \] 

 Explain why one of these sets is a subspace of \(\mathbb{R}^2\) and one is not. 

 \end{exerciseStatement}
 \begin{exerciseAnswer} 

\(W\) is a subspace of \(\mathbb{R}^2\) and \(U\) is not.

 \end{exerciseAnswer}
 \end{exercise}


\newpage




\begin{exercise}{VS4}{Subspaces}{0168} 
\begin{exerciseStatement} 

Consider the following two sets of Euclidean vectors: 

 \[
          U=\left\{ \left[\begin{array}{c}
x \\
y \\
z
\end{array}\right] \middle|\,4 \, x^{3} y + 5 \, z = 0\right\} \hspace{2em}  W=\left\{ \left[\begin{array}{c}
x \\
y \\
z
\end{array}\right] \middle|\,2 \, x + y = 4 \, z\right\}
    \] 

 Explain why one of these sets is a subspace of \(\mathbb{R}^3\) and one is not. 

 \end{exerciseStatement}
 \begin{exerciseAnswer} 

\(W\) is a subspace of \(\mathbb{R}^3\) and \(U\) is not.

 \end{exerciseAnswer}
 \end{exercise}



\begin{exercise}{VS5}{Linear independence}{0259} 
\begin{exerciseStatement} 

\begin{enumerate}[(a)]
\item  

 Write a statement involving the solutions of a vector equation that's equivalent to each claim below. 

 

\begin{itemize}
\item  

 The set of vectors \(\left\{ \left[\begin{array}{c}
0 \\
1 \\
0 \\
1
\end{array}\right] , \left[\begin{array}{c}
-1 \\
5 \\
-1 \\
0
\end{array}\right] , \left[\begin{array}{c}
2 \\
-11 \\
2 \\
-1
\end{array}\right] \right\}\) is linearly \textbf{independent}. 

 
\item  

 The set of vectors \(\left\{ \left[\begin{array}{c}
0 \\
1 \\
0 \\
1
\end{array}\right] , \left[\begin{array}{c}
-1 \\
5 \\
-1 \\
0
\end{array}\right] , \left[\begin{array}{c}
2 \\
-11 \\
2 \\
-1
\end{array}\right] \right\}\) is linearly \textbf{dependent}. 

 
\end{itemize}

     
\item  

 Explain how to determine which of these statements is true. 

 
\end{enumerate}

     \end{exerciseStatement}
 \begin{exerciseAnswer} 

 \[
\mathrm{RREF}\, \left[\begin{array}{ccc}
0 & -1 & 2 \\
1 & 5 & -11 \\
0 & -1 & 2 \\
1 & 0 & -1
\end{array}\right] = \left[\begin{array}{ccc}
1 & 0 & -1 \\
0 & 1 & -2 \\
0 & 0 & 0 \\
0 & 0 & 0
\end{array}\right]
            \] 

 

 The set of vectors \(\left\{ \left[\begin{array}{c}
0 \\
1 \\
0 \\
1
\end{array}\right] , \left[\begin{array}{c}
-1 \\
5 \\
-1 \\
0
\end{array}\right] , \left[\begin{array}{c}
2 \\
-11 \\
2 \\
-1
\end{array}\right] \right\}\) is linearly \textbf{dependent}. 

 \end{exerciseAnswer}
 \end{exercise}


\newpage




\begin{exercise}{VS5}{Linear independence}{0038} 
\begin{exerciseStatement} 

\begin{enumerate}[(a)]
\item  

 Write a statement involving the solutions of a vector equation that's equivalent to each claim below. 

 

\begin{itemize}
\item  

 The set of vectors \(\left\{ \left[\begin{array}{c}
-3 \\
1 \\
3 \\
-2
\end{array}\right] , \left[\begin{array}{c}
-5 \\
4 \\
5 \\
-4
\end{array}\right] , \left[\begin{array}{c}
-8 \\
5 \\
8 \\
-6
\end{array}\right] \right\}\) is linearly \textbf{independent}. 

 
\item  

 The set of vectors \(\left\{ \left[\begin{array}{c}
-3 \\
1 \\
3 \\
-2
\end{array}\right] , \left[\begin{array}{c}
-5 \\
4 \\
5 \\
-4
\end{array}\right] , \left[\begin{array}{c}
-8 \\
5 \\
8 \\
-6
\end{array}\right] \right\}\) is linearly \textbf{dependent}. 

 
\end{itemize}

     
\item  

 Explain how to determine which of these statements is true. 

 
\end{enumerate}

     \end{exerciseStatement}
 \begin{exerciseAnswer} 

 \[
\mathrm{RREF}\, \left[\begin{array}{ccc}
-3 & -5 & -8 \\
1 & 4 & 5 \\
3 & 5 & 8 \\
-2 & -4 & -6
\end{array}\right] = \left[\begin{array}{ccc}
1 & 0 & 1 \\
0 & 1 & 1 \\
0 & 0 & 0 \\
0 & 0 & 0
\end{array}\right]
            \] 

 

 The set of vectors \(\left\{ \left[\begin{array}{c}
-3 \\
1 \\
3 \\
-2
\end{array}\right] , \left[\begin{array}{c}
-5 \\
4 \\
5 \\
-4
\end{array}\right] , \left[\begin{array}{c}
-8 \\
5 \\
8 \\
-6
\end{array}\right] \right\}\) is linearly \textbf{dependent}. 

 \end{exerciseAnswer}
 \end{exercise}



\begin{exercise}{VS6}{Basis identification}{0005} 
\begin{exerciseStatement} 

\begin{enumerate}[(a)]
\item  

 Write a statement involving spanning and independence properties that's equivalent to each claim below. 

 

\begin{itemize}
\item  

 The set of vectors \(\left\{ \left[\begin{array}{c}
0 \\
-3 \\
-1 \\
-2
\end{array}\right] , \left[\begin{array}{c}
0 \\
-9 \\
-3 \\
-6
\end{array}\right] , \left[\begin{array}{c}
-2 \\
1 \\
-2 \\
3
\end{array}\right] , \left[\begin{array}{c}
4 \\
4 \\
6 \\
-2
\end{array}\right] \right\}\) is a \textbf{basis} of \(\mathbb{R}^4\). 

 
\item  

 The set of vectors \(\left\{ \left[\begin{array}{c}
0 \\
-3 \\
-1 \\
-2
\end{array}\right] , \left[\begin{array}{c}
0 \\
-9 \\
-3 \\
-6
\end{array}\right] , \left[\begin{array}{c}
-2 \\
1 \\
-2 \\
3
\end{array}\right] , \left[\begin{array}{c}
4 \\
4 \\
6 \\
-2
\end{array}\right] \right\}\) is \textbf{not} a basis of \(\mathbb{R}^4\). 

 
\end{itemize}

     
\item  

 Explain how to determine which of these statements is true. 

 
\end{enumerate}

     \end{exerciseStatement}
 \begin{exerciseAnswer} 

 \[
\mathrm{RREF}\, \left[\begin{array}{cccc}
0 & 0 & -2 & 4 \\
-3 & -9 & 1 & 4 \\
-1 & -3 & -2 & 6 \\
-2 & -6 & 3 & -2
\end{array}\right] = \left[\begin{array}{cccc}
1 & 3 & 0 & -2 \\
0 & 0 & 1 & -2 \\
0 & 0 & 0 & 0 \\
0 & 0 & 0 & 0
\end{array}\right]
            \] 

 

 The set of vectors \(\left\{ \left[\begin{array}{c}
0 \\
-3 \\
-1 \\
-2
\end{array}\right] , \left[\begin{array}{c}
0 \\
-9 \\
-3 \\
-6
\end{array}\right] , \left[\begin{array}{c}
-2 \\
1 \\
-2 \\
3
\end{array}\right] , \left[\begin{array}{c}
4 \\
4 \\
6 \\
-2
\end{array}\right] \right\}\) is \textbf{not} a basis. 

 \end{exerciseAnswer}
 \end{exercise}


\newpage




\begin{exercise}{VS6}{Basis identification}{0220} 
\begin{exerciseStatement} 

\begin{enumerate}[(a)]
\item  

 Write a statement involving spanning and independence properties that's equivalent to each claim below. 

 

\begin{itemize}
\item  

 The set of vectors \(\left\{ \left[\begin{array}{c}
0 \\
-1 \\
1 \\
-1
\end{array}\right] , \left[\begin{array}{c}
1 \\
-1 \\
3 \\
0
\end{array}\right] , \left[\begin{array}{c}
-2 \\
-1 \\
-2 \\
-3
\end{array}\right] , \left[\begin{array}{c}
-1 \\
-6 \\
6 \\
-7
\end{array}\right] \right\}\) is a \textbf{basis} of \(\mathbb{R}^4\). 

 
\item  

 The set of vectors \(\left\{ \left[\begin{array}{c}
0 \\
-1 \\
1 \\
-1
\end{array}\right] , \left[\begin{array}{c}
1 \\
-1 \\
3 \\
0
\end{array}\right] , \left[\begin{array}{c}
-2 \\
-1 \\
-2 \\
-3
\end{array}\right] , \left[\begin{array}{c}
-1 \\
-6 \\
6 \\
-7
\end{array}\right] \right\}\) is \textbf{not} a basis of \(\mathbb{R}^4\). 

 
\end{itemize}

     
\item  

 Explain how to determine which of these statements is true. 

 
\end{enumerate}

     \end{exerciseStatement}
 \begin{exerciseAnswer} 

 \[
\mathrm{RREF}\, \left[\begin{array}{cccc}
0 & 1 & -2 & -1 \\
-1 & -1 & -1 & -6 \\
1 & 3 & -2 & 6 \\
-1 & 0 & -3 & -7
\end{array}\right] = \left[\begin{array}{cccc}
1 & 0 & 0 & 1 \\
0 & 1 & 0 & 3 \\
0 & 0 & 1 & 2 \\
0 & 0 & 0 & 0
\end{array}\right]
            \] 

 

 The set of vectors \(\left\{ \left[\begin{array}{c}
0 \\
-1 \\
1 \\
-1
\end{array}\right] , \left[\begin{array}{c}
1 \\
-1 \\
3 \\
0
\end{array}\right] , \left[\begin{array}{c}
-2 \\
-1 \\
-2 \\
-3
\end{array}\right] , \left[\begin{array}{c}
-1 \\
-6 \\
6 \\
-7
\end{array}\right] \right\}\) is \textbf{not} a basis. 

 \end{exerciseAnswer}
 \end{exercise}



\begin{exercise}{VS7}{Basis of a subspace}{0297} 
\begin{exerciseStatement} 

 Consider the following subspace \(W\) of \(\mathbb R^4\): \[W=\mathrm{span}\,\left\{ \left[\begin{array}{c}
-2 \\
-1 \\
-3 \\
3
\end{array}\right] , \left[\begin{array}{c}
5 \\
2 \\
1 \\
-2
\end{array}\right] , \left[\begin{array}{c}
-6 \\
-2 \\
4 \\
-2
\end{array}\right] \right\}.\] 

 

\begin{enumerate}[(a)]
\item 

Explain how to find a basis of \(W\).


\item 

Explain how to find the dimension of \(W\).


\end{enumerate}

     \end{exerciseStatement}
 \begin{exerciseAnswer} 

\[\mathrm{RREF}\,\left[\begin{array}{ccc}
-2 & 5 & -6 \\
-1 & 2 & -2 \\
-3 & 1 & 4 \\
3 & -2 & -2
\end{array}\right]=\left[\begin{array}{ccc}
1 & 0 & -2 \\
0 & 1 & -2 \\
0 & 0 & 0 \\
0 & 0 & 0
\end{array}\right]\]

 

\begin{enumerate}[(a)]
\item 

A basis of \(W\) is \(\left\{ \left[\begin{array}{c}
-2 \\
-1 \\
-3 \\
3
\end{array}\right] , \left[\begin{array}{c}
5 \\
2 \\
1 \\
-2
\end{array}\right] \right\}\).


\item 

The dimension of \(W\) is \(2\).


\end{enumerate}

     \end{exerciseAnswer}
 \end{exercise}


\newpage




\begin{exercise}{VS7}{Basis of a subspace}{0229} 
\begin{exerciseStatement} 

 Consider the following subspace \(W\) of \(\mathbb R^4\): \[W=\mathrm{span}\,\left\{ \left[\begin{array}{c}
0 \\
2 \\
2 \\
-1
\end{array}\right] , \left[\begin{array}{c}
1 \\
-5 \\
-2 \\
4
\end{array}\right] , \left[\begin{array}{c}
0 \\
-6 \\
-6 \\
3
\end{array}\right] , \left[\begin{array}{c}
3 \\
5 \\
3 \\
0
\end{array}\right] , \left[\begin{array}{c}
6 \\
-14 \\
-7 \\
14
\end{array}\right] \right\}.\] 

 

\begin{enumerate}[(a)]
\item 

Explain how to find a basis of \(W\).


\item 

Explain how to find the dimension of \(W\).


\end{enumerate}

     \end{exerciseStatement}
 \begin{exerciseAnswer} 

\[\mathrm{RREF}\,\left[\begin{array}{ccccc}
0 & 1 & 0 & 3 & 6 \\
2 & -5 & -6 & 5 & -14 \\
2 & -2 & -6 & 3 & -7 \\
-1 & 4 & 3 & 0 & 14
\end{array}\right]=\left[\begin{array}{ccccc}
1 & 0 & -3 & 0 & -2 \\
0 & 1 & 0 & 0 & 3 \\
0 & 0 & 0 & 1 & 1 \\
0 & 0 & 0 & 0 & 0
\end{array}\right]\]

 

\begin{enumerate}[(a)]
\item 

A basis of \(W\) is \(\left\{ \left[\begin{array}{c}
0 \\
2 \\
2 \\
-1
\end{array}\right] , \left[\begin{array}{c}
1 \\
-5 \\
-2 \\
4
\end{array}\right] , \left[\begin{array}{c}
3 \\
5 \\
3 \\
0
\end{array}\right] \right\}\).


\item 

The dimension of \(W\) is \(3\).


\end{enumerate}

     \end{exerciseAnswer}
 \end{exercise}



\begin{exercise}{VS8}{Polynomial and matrix spaces}{0299} 
\begin{exerciseStatement} 

\begin{enumerate}[(a)]
\item  

 Given the set \[\left\{ x^{3} - x^{2} - x - 1 , -x^{3} - 2 \, x^{2} - 3 \, x - 2 , 4 \, x^{3} + 5 \, x^{2} + 8 \, x + 5 \right\}\] write a statement involving the solutions to a polynomial equation that's equivalent to each claim below. 

 

\begin{itemize}
\item  

 The set of polynomials is linearly \textbf{independent}. 

 
\item  

 The set of polynomials is linearly \textbf{dependent}. 

 
\end{itemize}

     
\item  

 Explain how to determine which of these statements is true. 

 
\end{enumerate}

     \end{exerciseStatement}
 \begin{exerciseAnswer} 

 \[
\mathrm{RREF}\, \left[\begin{array}{ccc}
-1 & -2 & 5 \\
-1 & -3 & 8 \\
-1 & -2 & 5 \\
1 & -1 & 4
\end{array}\right] = \left[\begin{array}{ccc}
1 & 0 & 1 \\
0 & 1 & -3 \\
0 & 0 & 0 \\
0 & 0 & 0
\end{array}\right]
            \] 

 

 The set is linearly \textbf{dependent}. 

 \end{exerciseAnswer}
 \end{exercise}


\newpage




\begin{exercise}{VS8}{Polynomial and matrix spaces}{0059} 
\begin{exerciseStatement} 

\begin{enumerate}[(a)]
\item  

 Given the set \[\left\{ -2 \, x^{2} + 1 , 6 \, x^{2} - 3 , 2 \, x^{2} - 1 , -4 \, x^{2} + 2 , -2 \, x^{3} + 3 \, x^{2} + x - 5 \right\}\] write a statement involving the solutions to a polynomial equation that's equivalent to each claim below. 

 

\begin{itemize}
\item  

 The set of polynomials \textbf{spans} \(\mathcal{P}_3\) 

 
\item  

 The set of polynomials does \textbf{not} span \(\mathcal{P}_3\) 

 
\end{itemize}

     
\item  

 Explain how to determine which of these statements is true. 

 
\end{enumerate}

     \end{exerciseStatement}
 \begin{exerciseAnswer} 

 \[
\mathrm{RREF}\, \left[\begin{array}{ccccc}
1 & -3 & -1 & 2 & -5 \\
0 & 0 & 0 & 0 & 1 \\
-2 & 6 & 2 & -4 & 3 \\
0 & 0 & 0 & 0 & -2
\end{array}\right] = \left[\begin{array}{ccccc}
1 & -3 & -1 & 2 & 0 \\
0 & 0 & 0 & 0 & 1 \\
0 & 0 & 0 & 0 & 0 \\
0 & 0 & 0 & 0 & 0
\end{array}\right]
            \] 

 

 The set does \textbf{not} span. 

 \end{exerciseAnswer}
 \end{exercise}



\begin{exercise}{VS8}{Polynomial and matrix spaces}{0129} 
\begin{exerciseStatement} 

\begin{enumerate}[(a)]
\item  

 Given the set \[\left\{ -2 \, x^{3} - 3 \, x^{2} - 4 \, x + 4 , 2 \, x^{3} + 3 \, x^{2} + x - 3 , 12 \, x^{3} + 18 \, x^{2} + 15 \, x - 21 \right\}\] write a statement involving the solutions to a polynomial equation that's equivalent to each claim below. 

 

\begin{itemize}
\item  

 The set of polynomials is linearly \textbf{independent}. 

 
\item  

 The set of polynomials is linearly \textbf{dependent}. 

 
\end{itemize}

     
\item  

 Explain how to determine which of these statements is true. 

 
\end{enumerate}

     \end{exerciseStatement}
 \begin{exerciseAnswer} 

 \[
\mathrm{RREF}\, \left[\begin{array}{ccc}
4 & -3 & -21 \\
-4 & 1 & 15 \\
-3 & 3 & 18 \\
-2 & 2 & 12
\end{array}\right] = \left[\begin{array}{ccc}
1 & 0 & -3 \\
0 & 1 & 3 \\
0 & 0 & 0 \\
0 & 0 & 0
\end{array}\right]
            \] 

 

 The set is linearly \textbf{dependent}. 

 \end{exerciseAnswer}
 \end{exercise}


\newpage




\begin{exercise}{VS8}{Polynomial and matrix spaces}{0000} 
\begin{exerciseStatement} 

\begin{enumerate}[(a)]
\item  

 Given the set \[\left\{ x^{3} - 2 \, x^{2} + x + 2 , 2 \, x^{2} - 1 , -x^{3} + 3 \, x^{2} + 3 \, x - 2 , x^{3} - 6 \, x^{2} + 9 \, x + 5 \right\}\] write a statement involving the solutions to a polynomial equation that's equivalent to each claim below. 

 

\begin{itemize}
\item  

 The set of polynomials is linearly \textbf{independent}. 

 
\item  

 The set of polynomials is linearly \textbf{dependent}. 

 
\end{itemize}

     
\item  

 Explain how to determine which of these statements is true. 

 
\end{enumerate}

     \end{exerciseStatement}
 \begin{exerciseAnswer} 

 \[
\mathrm{RREF}\, \left[\begin{array}{cccc}
2 & -1 & -2 & 5 \\
1 & 0 & 3 & 9 \\
-2 & 2 & 3 & -6 \\
1 & 0 & -1 & 1
\end{array}\right] = \left[\begin{array}{cccc}
1 & 0 & 0 & 3 \\
0 & 1 & 0 & -3 \\
0 & 0 & 1 & 2 \\
0 & 0 & 0 & 0
\end{array}\right]
            \] 

 

 The set is linearly \textbf{dependent}. 

 \end{exerciseAnswer}
 \end{exercise}



\begin{exercise}{VS8}{Polynomial and matrix spaces}{0216} 
\begin{exerciseStatement} 

\begin{enumerate}[(a)]
\item  

 Given the set \[\left\{ -x + 1 , -x + 1 , 5 \, x^{3} + 4 \, x^{2} - 4 \, x + 5 , 15 \, x^{3} + 12 \, x^{2} - 13 \, x + 16 \right\}\] write a statement involving the solutions to a polynomial equation that's equivalent to each claim below. 

 

\begin{itemize}
\item  

 The set of polynomials \textbf{spans} \(\mathcal{P}_3\) 

 
\item  

 The set of polynomials does \textbf{not} span \(\mathcal{P}_3\) 

 
\end{itemize}

     
\item  

 Explain how to determine which of these statements is true. 

 
\end{enumerate}

     \end{exerciseStatement}
 \begin{exerciseAnswer} 

 \[
\mathrm{RREF}\, \left[\begin{array}{cccc}
1 & 1 & 5 & 16 \\
-1 & -1 & -4 & -13 \\
0 & 0 & 4 & 12 \\
0 & 0 & 5 & 15
\end{array}\right] = \left[\begin{array}{cccc}
1 & 1 & 0 & 1 \\
0 & 0 & 1 & 3 \\
0 & 0 & 0 & 0 \\
0 & 0 & 0 & 0
\end{array}\right]
            \] 

 

 The set does \textbf{not} span. 

 \end{exerciseAnswer}
 \end{exercise}


\newpage




\begin{exercise}{VS8}{Polynomial and matrix spaces}{0000} 
\begin{exerciseStatement} 

\begin{enumerate}[(a)]
\item  

 Given the set \[\left\{ x^{3} - 2 \, x^{2} + x + 2 , 2 \, x^{2} - 1 , -x^{3} + 3 \, x^{2} + 3 \, x - 2 , x^{3} - 6 \, x^{2} + 9 \, x + 5 \right\}\] write a statement involving the solutions to a polynomial equation that's equivalent to each claim below. 

 

\begin{itemize}
\item  

 The set of polynomials is linearly \textbf{independent}. 

 
\item  

 The set of polynomials is linearly \textbf{dependent}. 

 
\end{itemize}

     
\item  

 Explain how to determine which of these statements is true. 

 
\end{enumerate}

     \end{exerciseStatement}
 \begin{exerciseAnswer} 

 \[
\mathrm{RREF}\, \left[\begin{array}{cccc}
2 & -1 & -2 & 5 \\
1 & 0 & 3 & 9 \\
-2 & 2 & 3 & -6 \\
1 & 0 & -1 & 1
\end{array}\right] = \left[\begin{array}{cccc}
1 & 0 & 0 & 3 \\
0 & 1 & 0 & -3 \\
0 & 0 & 1 & 2 \\
0 & 0 & 0 & 0
\end{array}\right]
            \] 

 

 The set is linearly \textbf{dependent}. 

 \end{exerciseAnswer}
 \end{exercise}



\begin{exercise}{VS8}{Polynomial and matrix spaces}{0188} 
\begin{exerciseStatement} 

\begin{enumerate}[(a)]
\item  

 Given the set \[\left\{ -3 \, x^{3} + 2 \, x^{2} + 2 \, x + 3 , -x^{3} + x^{2} + x + 1 , -2 \, x^{3} + 2 \, x^{2} + x , 4 \, x^{3} - x^{2} - 2 \, x - 6 \right\}\] write a statement involving the solutions to a polynomial equation that's equivalent to each claim below. 

 

\begin{itemize}
\item  

 The set of polynomials is linearly \textbf{independent}. 

 
\item  

 The set of polynomials is linearly \textbf{dependent}. 

 
\end{itemize}

     
\item  

 Explain how to determine which of these statements is true. 

 
\end{enumerate}

     \end{exerciseStatement}
 \begin{exerciseAnswer} 

 \[
\mathrm{RREF}\, \left[\begin{array}{cccc}
3 & 1 & 0 & -6 \\
2 & 1 & 1 & -2 \\
2 & 1 & 2 & -1 \\
-3 & -1 & -2 & 4
\end{array}\right] = \left[\begin{array}{cccc}
1 & 0 & 0 & -3 \\
0 & 1 & 0 & 3 \\
0 & 0 & 1 & 1 \\
0 & 0 & 0 & 0
\end{array}\right]
            \] 

 

 The set is linearly \textbf{dependent}. 

 \end{exerciseAnswer}
 \end{exercise}


\newpage




\begin{exercise}{VS8}{Polynomial and matrix spaces}{0146} 
\begin{exerciseStatement} 

\begin{enumerate}[(a)]
\item  

 Given the set \[\left\{ \left[\begin{array}{cc}
-3 & -2 \\
-2 & -4
\end{array}\right] , \left[\begin{array}{cc}
-3 & -2 \\
-2 & -4
\end{array}\right] , \left[\begin{array}{cc}
-1 & -3 \\
-1 & -3
\end{array}\right] , \left[\begin{array}{cc}
3 & 2 \\
2 & 4
\end{array}\right] , \left[\begin{array}{cc}
11 & 12 \\
8 & 18
\end{array}\right] \right\}\] write a statement involving the solutions to a matrix equation that's equivalent to each claim below. 

 

\begin{itemize}
\item  

 The set of matrices \textbf{spans} \(\mathrm{M}_{2,2}\) 

 
\item  

 The set of matrices does \textbf{not} span \(\mathrm{M}_{2,2}\) 

 
\end{itemize}

     
\item  

 Explain how to determine which of these statements is true. 

 
\end{enumerate}

     \end{exerciseStatement}
 \begin{exerciseAnswer} 

 \[
\mathrm{RREF}\, \left[\begin{array}{ccccc}
-3 & -3 & -1 & 3 & 11 \\
-2 & -2 & -3 & 2 & 12 \\
-2 & -2 & -1 & 2 & 8 \\
-4 & -4 & -3 & 4 & 18
\end{array}\right] = \left[\begin{array}{ccccc}
1 & 1 & 0 & -1 & -3 \\
0 & 0 & 1 & 0 & -2 \\
0 & 0 & 0 & 0 & 0 \\
0 & 0 & 0 & 0 & 0
\end{array}\right]
            \] 

 

 The set does \textbf{not} span. 

 \end{exerciseAnswer}
 \end{exercise}



\begin{exercise}{VS8}{Polynomial and matrix spaces}{0156} 
\begin{exerciseStatement} 

\begin{enumerate}[(a)]
\item  

 Given the set \[\left\{ \left[\begin{array}{cc}
-2 & -4 \\
3 & 0
\end{array}\right] , \left[\begin{array}{cc}
0 & 1 \\
0 & -1
\end{array}\right] , \left[\begin{array}{cc}
3 & 4 \\
-5 & 2
\end{array}\right] , \left[\begin{array}{cc}
6 & 13 \\
-9 & -1
\end{array}\right] \right\}\] write a statement involving the solutions to a matrix equation that's equivalent to each claim below. 

 

\begin{itemize}
\item  

 The set of matrices is linearly \textbf{independent}. 

 
\item  

 The set of matrices is linearly \textbf{dependent}. 

 
\end{itemize}

     
\item  

 Explain how to determine which of these statements is true. 

 
\end{enumerate}

     \end{exerciseStatement}
 \begin{exerciseAnswer} 

 \[
\mathrm{RREF}\, \left[\begin{array}{cccc}
-2 & 0 & 3 & 6 \\
-4 & 1 & 4 & 13 \\
3 & 0 & -5 & -9 \\
0 & -1 & 2 & -1
\end{array}\right] = \left[\begin{array}{cccc}
1 & 0 & 0 & -3 \\
0 & 1 & 0 & 1 \\
0 & 0 & 1 & 0 \\
0 & 0 & 0 & 0
\end{array}\right]
            \] 

 

 The set is linearly \textbf{dependent}. 

 \end{exerciseAnswer}
 \end{exercise}


\newpage




\begin{exercise}{VS8}{Polynomial and matrix spaces}{0008} 
\begin{exerciseStatement} 

\begin{enumerate}[(a)]
\item  

 Given the set \[\left\{ \left[\begin{array}{cc}
1 & -1 \\
0 & 0
\end{array}\right] , \left[\begin{array}{cc}
2 & -1 \\
0 & -3
\end{array}\right] , \left[\begin{array}{cc}
4 & -3 \\
1 & -3
\end{array}\right] , \left[\begin{array}{cc}
-2 & 3 \\
-4 & -2
\end{array}\right] \right\}\] write a statement involving the solutions to a matrix equation that's equivalent to each claim below. 

 

\begin{itemize}
\item  

 The set of matrices is linearly \textbf{independent}. 

 
\item  

 The set of matrices is linearly \textbf{dependent}. 

 
\end{itemize}

     
\item  

 Explain how to determine which of these statements is true. 

 
\end{enumerate}

     \end{exerciseStatement}
 \begin{exerciseAnswer} 

 \[
\mathrm{RREF}\, \left[\begin{array}{cccc}
1 & 2 & 4 & -2 \\
-1 & -1 & -3 & 3 \\
0 & 0 & 1 & -4 \\
0 & -3 & -3 & -2
\end{array}\right] = \left[\begin{array}{cccc}
1 & 0 & 0 & 0 \\
0 & 1 & 0 & 0 \\
0 & 0 & 1 & 0 \\
0 & 0 & 0 & 1
\end{array}\right]
            \] 

 

 The set is linearly \textbf{independent}. 

 \end{exerciseAnswer}
 \end{exercise}



\begin{exercise}{VS9}{Homogeneous systems}{0127} 
\begin{exerciseStatement} 

Consider the following homogeneous system of equations.

 \[\begin{matrix}
 x_{1} &  +  & 4 \, x_{2} &  -  & 2 \, x_{3} & = & 0 \\
 &  & x_{2} &  &  & = & 0 \\
 &  & 2 \, x_{2} &  &  & = & 0 \\
 x_{1} &  +  & 5 \, x_{2} &  -  & 2 \, x_{3} & = & 0 \\
 &  -  & 3 \, x_{2} &  &  & = & 0 \\
 \end{matrix}\] 

\begin{enumerate}[(a)]
\item  Find the solution space of this system.
\item  Find a basis of the solution space.
\end{enumerate}

     \end{exerciseStatement}
 \begin{exerciseAnswer} 

\[\mathrm{RREF}\,\left[\begin{array}{ccc|c}
1 & 4 & -2 & 0 \\
0 & 1 & 0 & 0 \\
0 & 2 & 0 & 0 \\
1 & 5 & -2 & 0 \\
0 & -3 & 0 & 0
\end{array}\right]=\left[\begin{array}{ccc|c}
1 & 0 & -2 & 0 \\
0 & 1 & 0 & 0 \\
0 & 0 & 0 & 0 \\
0 & 0 & 0 & 0 \\
0 & 0 & 0 & 0
\end{array}\right]\]

 

\begin{enumerate}[(a)]
\item The solution space is \( \left\{ \left[\begin{array}{c}
2 \, a \\
0 \\
a
\end{array}\right] \,\middle|\, a \in\mathbb R \right\} \) 
\item A basis of the solution space is \(\left\{ \left[\begin{array}{c}
2 \\
0 \\
1
\end{array}\right] \right\}\).
\end{enumerate}

     \end{exerciseAnswer}
 \end{exercise}


\newpage




\begin{exercise}{VS9}{Homogeneous systems}{0029} 
\begin{exerciseStatement} 

Consider the following homogeneous system of equations.

 \[\begin{matrix}
 &  -  & x_{2} &  -  & x_{3} & = & 0 \\
 x_{1} &  +  & 5 \, x_{2} &  +  & 7 \, x_{3} & = & 0 \\
 -x_{1} &  -  & 4 \, x_{2} &  -  & 6 \, x_{3} & = & 0 \\
 -2 \, x_{1} &  -  & 4 \, x_{2} &  -  & 8 \, x_{3} & = & 0 \\
 x_{1} &  +  & 2 \, x_{2} &  +  & 4 \, x_{3} & = & 0 \\
 \end{matrix}\] 

\begin{enumerate}[(a)]
\item  Find the solution space of this system.
\item  Find a basis of the solution space.
\end{enumerate}

     \end{exerciseStatement}
 \begin{exerciseAnswer} 

\[\mathrm{RREF}\,\left[\begin{array}{ccc|c}
0 & -1 & -1 & 0 \\
1 & 5 & 7 & 0 \\
-1 & -4 & -6 & 0 \\
-2 & -4 & -8 & 0 \\
1 & 2 & 4 & 0
\end{array}\right]=\left[\begin{array}{ccc|c}
1 & 0 & 2 & 0 \\
0 & 1 & 1 & 0 \\
0 & 0 & 0 & 0 \\
0 & 0 & 0 & 0 \\
0 & 0 & 0 & 0
\end{array}\right]\]

 

\begin{enumerate}[(a)]
\item The solution space is \( \left\{ \left[\begin{array}{c}
-2 \, a \\
-a \\
a
\end{array}\right] \,\middle|\, a \in\mathbb R \right\} \) 
\item A basis of the solution space is \(\left\{ \left[\begin{array}{c}
-2 \\
-1 \\
1
\end{array}\right] \right\}\).
\end{enumerate}

     \end{exerciseAnswer}
 \end{exercise}



\begin{exercise}{VS9}{Homogeneous systems}{0182} 
\begin{exerciseStatement} 

Consider the following homogeneous system of equations.

 \[\begin{matrix}
 -x_{1} &  -  & x_{2} &  +  & 3 \, x_{3} & = & 0 \\
 &  &  &  & x_{3} & = & 0 \\
 -x_{1} &  -  & x_{2} &  -  & 3 \, x_{3} & = & 0 \\
 -x_{1} &  -  & x_{2} &  -  & 5 \, x_{3} & = & 0 \\
 x_{1} &  +  & x_{2} &  +  & x_{3} & = & 0 \\
 \end{matrix}\] 

\begin{enumerate}[(a)]
\item  Find the solution space of this system.
\item  Find a basis of the solution space.
\end{enumerate}

     \end{exerciseStatement}
 \begin{exerciseAnswer} 

\[\mathrm{RREF}\,\left[\begin{array}{ccc|c}
-1 & -1 & 3 & 0 \\
0 & 0 & 1 & 0 \\
-1 & -1 & -3 & 0 \\
-1 & -1 & -5 & 0 \\
1 & 1 & 1 & 0
\end{array}\right]=\left[\begin{array}{ccc|c}
1 & 1 & 0 & 0 \\
0 & 0 & 1 & 0 \\
0 & 0 & 0 & 0 \\
0 & 0 & 0 & 0 \\
0 & 0 & 0 & 0
\end{array}\right]\]

 

\begin{enumerate}[(a)]
\item The solution space is \( \left\{ \left[\begin{array}{c}
-a \\
a \\
0
\end{array}\right] \,\middle|\, a \in\mathbb R \right\} \) 
\item A basis of the solution space is \(\left\{ \left[\begin{array}{c}
-1 \\
1 \\
0
\end{array}\right] \right\}\).
\end{enumerate}

     \end{exerciseAnswer}
 \end{exercise}


\newpage




\begin{exercise}{VS9}{Homogeneous systems}{0053} 
\begin{exerciseStatement} 

Consider the following homogeneous system of equations.

 \[\begin{matrix}
 -3 \, x_{1} &  -  & 4 \, x_{2} &  -  & 2 \, x_{3} & = & 0 \\
 -x_{1} &  -  & 3 \, x_{2} &  -  & 4 \, x_{3} & = & 0 \\
 -x_{1} &  &  &  +  & 2 \, x_{3} & = & 0 \\
 2 \, x_{1} &  +  & 4 \, x_{2} &  +  & 4 \, x_{3} & = & 0 \\
 x_{1} &  &  &  -  & 2 \, x_{3} & = & 0 \\
 \end{matrix}\] 

\begin{enumerate}[(a)]
\item  Find the solution space of this system.
\item  Find a basis of the solution space.
\end{enumerate}

     \end{exerciseStatement}
 \begin{exerciseAnswer} 

\[\mathrm{RREF}\,\left[\begin{array}{ccc|c}
-3 & -4 & -2 & 0 \\
-1 & -3 & -4 & 0 \\
-1 & 0 & 2 & 0 \\
2 & 4 & 4 & 0 \\
1 & 0 & -2 & 0
\end{array}\right]=\left[\begin{array}{ccc|c}
1 & 0 & -2 & 0 \\
0 & 1 & 2 & 0 \\
0 & 0 & 0 & 0 \\
0 & 0 & 0 & 0 \\
0 & 0 & 0 & 0
\end{array}\right]\]

 

\begin{enumerate}[(a)]
\item The solution space is \( \left\{ \left[\begin{array}{c}
2 \, a \\
-2 \, a \\
a
\end{array}\right] \,\middle|\, a \in\mathbb R \right\} \) 
\item A basis of the solution space is \(\left\{ \left[\begin{array}{c}
2 \\
-2 \\
1
\end{array}\right] \right\}\).
\end{enumerate}

     \end{exerciseAnswer}
 \end{exercise}



\begin{exercise}{VS9}{Homogeneous systems}{0051} 
\begin{exerciseStatement} 

Consider the following homogeneous system of equations.

 \[\begin{matrix}
 x_{1} &  -  & 2 \, x_{2} &  -  & x_{3} &  &  & = & 0 \\
 -2 \, x_{1} &  +  & 4 \, x_{2} &  +  & 3 \, x_{3} &  +  & 4 \, x_{4} & = & 0 \\
 -3 \, x_{1} &  +  & 6 \, x_{2} &  +  & 2 \, x_{3} &  -  & 3 \, x_{4} & = & 0 \\
 -3 \, x_{1} &  +  & 6 \, x_{2} &  +  & 2 \, x_{3} &  -  & 2 \, x_{4} & = & 0 \\
 \end{matrix}\] 

\begin{enumerate}[(a)]
\item  Find the solution space of this system.
\item  Find a basis of the solution space.
\end{enumerate}

     \end{exerciseStatement}
 \begin{exerciseAnswer} 

\[\mathrm{RREF}\,\left[\begin{array}{cccc|c}
1 & -2 & -1 & 0 & 0 \\
-2 & 4 & 3 & 4 & 0 \\
-3 & 6 & 2 & -3 & 0 \\
-3 & 6 & 2 & -2 & 0
\end{array}\right]=\left[\begin{array}{cccc|c}
1 & -2 & 0 & 0 & 0 \\
0 & 0 & 1 & 0 & 0 \\
0 & 0 & 0 & 1 & 0 \\
0 & 0 & 0 & 0 & 0
\end{array}\right]\]

 

\begin{enumerate}[(a)]
\item The solution space is \( \left\{ \left[\begin{array}{c}
2 \, a \\
a \\
0 \\
0
\end{array}\right] \,\middle|\, a \in\mathbb R \right\} \) 
\item A basis of the solution space is \(\left\{ \left[\begin{array}{c}
2 \\
1 \\
0 \\
0
\end{array}\right] \right\}\).
\end{enumerate}

     \end{exerciseAnswer}
 \end{exercise}


\newpage




\begin{exercise}{AT1}{Linear maps}{0186} 
\begin{exerciseStatement} 

 Consider the following maps of polynomials \(S:\mathcal{P}\rightarrow\mathcal{P}\) and \(T:\mathcal{P}\rightarrow\mathcal{P}\) defined by \[
            S(h(x))=
                    2 \, h\left(x\right) - 2 \, h'\left(4\right)
                \hspace{1em} \text{and} \hspace{1em}
            T(h(x))=
                    2 \, h\left(x^{3}\right) - 4
        \] Explain why one these maps is a linear transformation and why the other map is not. 

 \end{exerciseStatement}
 \begin{exerciseAnswer} 

\(S\) is linear and \(T\) is not linear.

 \end{exerciseAnswer}
 \end{exercise}



\begin{exercise}{AT1}{Linear maps}{0273} 
\begin{exerciseStatement} 

 Consider the following maps of polynomials \(S:\mathcal{P}\rightarrow\mathcal{P}\) and \(T:\mathcal{P}\rightarrow\mathcal{P}\) defined by \[
            S(g(x))=
                    3 \, g\left(x\right)^{2} - 2 \, g'\left(-4\right)
                \hspace{1em} \text{and} \hspace{1em}
            T(g(x))=
                    g\left(-4\right) + 4 \, g\left(x^{3}\right)
        \] Explain why one these maps is a linear transformation and why the other map is not. 

 \end{exerciseStatement}
 \begin{exerciseAnswer} 

\(S\) is not linear and \(T\) is linear.

 \end{exerciseAnswer}
 \end{exercise}


\newpage




\begin{exercise}{AT1}{Linear maps}{0299} 
\begin{exerciseStatement} 

 Consider the following maps of polynomials \(S:\mathcal{P}\rightarrow\mathcal{P}\) and \(T:\mathcal{P}\rightarrow\mathcal{P}\) defined by \[
            S(h(x))=
                    -4 \, h\left(-1\right) - h\left(x^{3}\right)
                \hspace{1em} \text{and} \hspace{1em}
            T(h(x))=
                    -3 \, h\left(x\right)^{2} - 4 \, h\left(x\right)
        \] Explain why one these maps is a linear transformation and why the other map is not. 

 \end{exerciseStatement}
 \begin{exerciseAnswer} 

\(S\) is linear and \(T\) is not linear.

 \end{exerciseAnswer}
 \end{exercise}



\begin{exercise}{AT1}{Linear maps}{0038} 
\begin{exerciseStatement} 

 Consider the following maps of polynomials \(S:\mathcal{P}\rightarrow\mathcal{P}\) and \(T:\mathcal{P}\rightarrow\mathcal{P}\) defined by \[
            S(f(x))=
                    -2 \, f\left(-4\right) + 5 \, f'\left(x\right)
                \hspace{1em} \text{and} \hspace{1em}
            T(f(x))=
                    -f\left(x\right)^{2} - 4 \, f\left(x\right)
        \] Explain why one these maps is a linear transformation and why the other map is not. 

 \end{exerciseStatement}
 \begin{exerciseAnswer} 

\(S\) is linear and \(T\) is not linear.

 \end{exerciseAnswer}
 \end{exercise}


\newpage




\begin{exercise}{AT1}{Linear maps}{0269} 
\begin{exerciseStatement} 

 Consider the following maps of polynomials \(S:\mathcal{P}\rightarrow\mathcal{P}\) and \(T:\mathcal{P}\rightarrow\mathcal{P}\) defined by \[
            S(f(x))=
                    -f\left(x\right)^{2} + 2 \, f\left(x^{3}\right)
                \hspace{1em} \text{and} \hspace{1em}
            T(f(x))=
                    5 \, x^{3} f\left(x\right) - 4 \, f'\left(-4\right)
        \] Explain why one these maps is a linear transformation and why the other map is not. 

 \end{exerciseStatement}
 \begin{exerciseAnswer} 

\(S\) is not linear and \(T\) is linear.

 \end{exerciseAnswer}
 \end{exercise}



\begin{exercise}{AT1}{Linear maps}{0291} 
\begin{exerciseStatement} 

 Consider the following maps of polynomials \(S:\mathcal{P}\rightarrow\mathcal{P}\) and \(T:\mathcal{P}\rightarrow\mathcal{P}\) defined by \[
            S(h(x))=
                    -3 \, h\left(x^{3}\right) - 5 \, h'\left(x\right)
                \hspace{1em} \text{and} \hspace{1em}
            T(h(x))=
                    -2 \, h\left(x\right)^{2} - h'\left(3\right)
        \] Explain why one these maps is a linear transformation and why the other map is not. 

 \end{exerciseStatement}
 \begin{exerciseAnswer} 

\(S\) is linear and \(T\) is not linear.

 \end{exerciseAnswer}
 \end{exercise}


\newpage




\begin{exercise}{AT1}{Linear maps}{0018} 
\begin{exerciseStatement} 

 Consider the following maps of polynomials \(S:\mathcal{P}\rightarrow\mathcal{P}\) and \(T:\mathcal{P}\rightarrow\mathcal{P}\) defined by \[
            S(g(x))=
                    g'\left(x\right) + 1
                \hspace{1em} \text{and} \hspace{1em}
            T(g(x))=
                    -5 \, g\left(4\right) + 3 \, g'\left(-1\right)
        \] Explain why one these maps is a linear transformation and why the other map is not. 

 \end{exerciseStatement}
 \begin{exerciseAnswer} 

\(S\) is not linear and \(T\) is linear.

 \end{exerciseAnswer}
 \end{exercise}



\begin{exercise}{AT1}{Linear maps}{0260} 
\begin{exerciseStatement} 

 Consider the following maps of polynomials \(S:\mathcal{P}\rightarrow\mathcal{P}\) and \(T:\mathcal{P}\rightarrow\mathcal{P}\) defined by \[
            S(h(x))=
                    -5 \, h\left(-4\right) + 5 \, h'\left(x\right)
                \hspace{1em} \text{and} \hspace{1em}
            T(h(x))=
                    4 \, x^{2} - x h\left(x\right)
        \] Explain why one these maps is a linear transformation and why the other map is not. 

 \end{exerciseStatement}
 \begin{exerciseAnswer} 

\(S\) is linear and \(T\) is not linear.

 \end{exerciseAnswer}
 \end{exercise}


\newpage




\begin{exercise}{AT1}{Linear maps}{0220} 
\begin{exerciseStatement} 

 Consider the following maps of polynomials \(S:\mathcal{P}\rightarrow\mathcal{P}\) and \(T:\mathcal{P}\rightarrow\mathcal{P}\) defined by \[
            S(h(x))=
                    -3 \, h\left(x\right)^{2} + h\left(5\right)
                \hspace{1em} \text{and} \hspace{1em}
            T(h(x))=
                    2 \, h\left(x^{2}\right) - h'\left(-3\right)
        \] Explain why one these maps is a linear transformation and why the other map is not. 

 \end{exerciseStatement}
 \begin{exerciseAnswer} 

\(S\) is not linear and \(T\) is linear.

 \end{exerciseAnswer}
 \end{exercise}



\begin{exercise}{AT1}{Linear maps}{0248} 
\begin{exerciseStatement} 

 Consider the following maps of polynomials \(S:\mathcal{P}\rightarrow\mathcal{P}\) and \(T:\mathcal{P}\rightarrow\mathcal{P}\) defined by \[
            S(h(x))=
                    -2 \, x - 2 \, h\left(-5\right)
                \hspace{1em} \text{and} \hspace{1em}
            T(h(x))=
                    x^{3} h\left(x\right) + 5 \, h\left(x^{2}\right)
        \] Explain why one these maps is a linear transformation and why the other map is not. 

 \end{exerciseStatement}
 \begin{exerciseAnswer} 

\(S\) is not linear and \(T\) is linear.

 \end{exerciseAnswer}
 \end{exercise}


\newpage




\begin{exercise}{AT2}{Standard matrices}{0127} 
\begin{exerciseStatement} 

\begin{enumerate}[(a)]
\item Find the standard matrix for the linear transformation \(S:\mathbb{R}^4 \to \mathbb{R}^3\) given by \[S\left( \left[\begin{array}{c}
x \\
y \\
z \\
{w}
\end{array}\right] \right) = \left[\begin{array}{c}
-2 \, x - 3 \, y - 5 \, z - 6 \, {w} \\
y + 4 \, z - {w} \\
x + y + 4 \, {w}
\end{array}\right].\] 
\item Let \(T:\mathbb{R}^3 \to \mathbb{R}^3\) be the linear transformation given by the standard matrix \[\left[\begin{array}{ccc}
-2 & -1 & 4 \\
3 & 1 & -7 \\
-5 & -4 & 8
\end{array}\right].\] Compute \(T\left(\left[\begin{array}{c}
-4 \\
-3 \\
0
\end{array}\right]\right)\). 
\end{enumerate}

     \end{exerciseStatement}
 \begin{exerciseAnswer} 

\begin{enumerate}[(a)]
\item  \[\left[\begin{array}{cccc}
-2 & -3 & -5 & -6 \\
0 & 1 & 4 & -1 \\
1 & 1 & 0 & 4
\end{array}\right]\] 
\item  \[T\left(\left[\begin{array}{c}
-4 \\
-3 \\
0
\end{array}\right]\right)=\left[\begin{array}{c}
11 \\
-15 \\
32
\end{array}\right]\] 
\end{enumerate}

     \end{exerciseAnswer}
 \end{exercise}



\begin{exercise}{AT2}{Standard matrices}{0018} 
\begin{exerciseStatement} 

\begin{enumerate}[(a)]
\item Find the standard matrix for the linear transformation \(S:\mathbb{R}^3 \to \mathbb{R}^4\) given by \[S\left( \left[\begin{array}{c}
x \\
y \\
z
\end{array}\right] \right) = \left[\begin{array}{c}
-2 \, x + 2 \, y + 6 \, z \\
x + 6 \, y + 8 \, z \\
x + 2 \, y + z \\
x + y
\end{array}\right].\] 
\item Let \(T:\mathbb{R}^3 \to \mathbb{R}^2\) be the linear transformation given by the standard matrix \[\left[\begin{array}{ccc}
-4 & 1 & 7 \\
-5 & 1 & 8
\end{array}\right].\] Compute \(T\left(\left[\begin{array}{c}
-8 \\
-1 \\
-6
\end{array}\right]\right)\). 
\end{enumerate}

     \end{exerciseStatement}
 \begin{exerciseAnswer} 

\begin{enumerate}[(a)]
\item  \[\left[\begin{array}{ccc}
-2 & 2 & 6 \\
1 & 6 & 8 \\
1 & 2 & 1 \\
1 & 1 & 0
\end{array}\right]\] 
\item  \[T\left(\left[\begin{array}{c}
-8 \\
-1 \\
-6
\end{array}\right]\right)=\left[\begin{array}{c}
-11 \\
-9
\end{array}\right]\] 
\end{enumerate}

     \end{exerciseAnswer}
 \end{exercise}


\newpage




\begin{exercise}{AT3}{Image and kernel}{0043} 
\begin{exerciseStatement} 

 Let \(T:\mathbb{R}^4 \to \mathbb{R}^3\) be the linear transformation given by \[T\left( \left[\begin{array}{c}
x \\
y \\
z \\
{w}
\end{array}\right] \right) = \left[\begin{array}{c}
-x + y - 2 \, z + 2 \, {w} \\
x - y + z - 2 \, {w} \\
-3 \, x + 3 \, y + 2 \, z + 6 \, {w}
\end{array}\right].\] 

 

\begin{enumerate}[(a)]
\item Explain how to find the image of \(T\) and the kernel of \(T\).
\item Explain how to find a basis of the image of \(T\) and a basis of the kernel of \(T\).
\item Explain how to find the rank and nullity of \(T\), and why the rank-nullity theorem holds for \(T\).
\end{enumerate}

     \end{exerciseStatement}
 \begin{exerciseAnswer} 

\[\mathrm{RREF}\,\left[\begin{array}{cccc}
-1 & 1 & -2 & 2 \\
1 & -1 & 1 & -2 \\
-3 & 3 & 2 & 6
\end{array}\right]=\left[\begin{array}{cccc}
1 & -1 & 0 & -2 \\
0 & 0 & 1 & 0 \\
0 & 0 & 0 & 0
\end{array}\right]\]

 

\begin{enumerate}[(a)]
\item  

 \(\mathrm{Im}\,T =  \left\{ \left[\begin{array}{c}
-a - 2 \, b \\
a + b \\
-3 \, a + 2 \, b
\end{array}\right] \middle|\,a,b\in\mathbb{R}\right\}\) and \(\mathrm{ker}\,T = \left\{ \left[\begin{array}{c}
a + 2 \, b \\
a \\
0 \\
b
\end{array}\right] \middle|\,a,b\in\mathbb{R}\right\}\) 

 
\item  

 A basis of \(\mathrm{Im}\,T\) is \(\left\{ \left[\begin{array}{c}
-1 \\
1 \\
-3
\end{array}\right] , \left[\begin{array}{c}
-2 \\
1 \\
2
\end{array}\right] \right\}\). A basis of \(\mathrm{ker}\,T\) is \(\left\{ \left[\begin{array}{c}
1 \\
1 \\
0 \\
0
\end{array}\right] , \left[\begin{array}{c}
2 \\
0 \\
0 \\
1
\end{array}\right] \right\}\). 

 
\item  

 The rank of \(T\) is \(2\), the nullity of \(T\) is \(2\), and the dimension of the domain of \(T\) is \(4\). The rank-nullity theorem asserts that \(2+2=4\), which we see to be true. 

 
\end{enumerate}

     \end{exerciseAnswer}
 \end{exercise}



\begin{exercise}{AT3}{Image and kernel}{0197} 
\begin{exerciseStatement} 

 Let \(T:\mathbb{R}^4 \to \mathbb{R}^3\) be the linear transformation given by \[T\left( \left[\begin{array}{c}
x \\
y \\
z \\
{w}
\end{array}\right] \right) = \left[\begin{array}{c}
-y - 4 \, z - 7 \, {w} \\
x - y - 2 \, {w} \\
-y - 3 \, z - 6 \, {w}
\end{array}\right].\] 

 

\begin{enumerate}[(a)]
\item Explain how to find the image of \(T\) and the kernel of \(T\).
\item Explain how to find a basis of the image of \(T\) and a basis of the kernel of \(T\).
\item Explain how to find the rank and nullity of \(T\), and why the rank-nullity theorem holds for \(T\).
\end{enumerate}

     \end{exerciseStatement}
 \begin{exerciseAnswer} 

\[\mathrm{RREF}\,\left[\begin{array}{cccc}
0 & -1 & -4 & -7 \\
1 & -1 & 0 & -2 \\
0 & -1 & -3 & -6
\end{array}\right]=\left[\begin{array}{cccc}
1 & 0 & 0 & 1 \\
0 & 1 & 0 & 3 \\
0 & 0 & 1 & 1
\end{array}\right]\]

 

\begin{enumerate}[(a)]
\item  

 \(\mathrm{Im}\,T =  \left\{ \left[\begin{array}{c}
-b - 4 \, c \\
a - b \\
-b - 3 \, c
\end{array}\right] \middle|\,a,b,c\in\mathbb{R}\right\}\) and \(\mathrm{ker}\,T = \left\{ \left[\begin{array}{c}
-a \\
-3 \, a \\
-a \\
a
\end{array}\right] \middle|\,a\in\mathbb{R}\right\}\) 

 
\item  

 A basis of \(\mathrm{Im}\,T\) is \(\left\{ \left[\begin{array}{c}
0 \\
1 \\
0
\end{array}\right] , \left[\begin{array}{c}
-1 \\
-1 \\
-1
\end{array}\right] , \left[\begin{array}{c}
-4 \\
0 \\
-3
\end{array}\right] \right\}\). A basis of \(\mathrm{ker}\,T\) is \(\left\{ \left[\begin{array}{c}
-1 \\
-3 \\
-1 \\
1
\end{array}\right] \right\}\). 

 
\item  

 The rank of \(T\) is \(3\), the nullity of \(T\) is \(1\), and the dimension of the domain of \(T\) is \(4\). The rank-nullity theorem asserts that \(3+1=4\), which we see to be true. 

 
\end{enumerate}

     \end{exerciseAnswer}
 \end{exercise}


\newpage




\begin{exercise}{AT3}{Image and kernel}{0158} 
\begin{exerciseStatement} 

 Let \(T:\mathbb{R}^4 \to \mathbb{R}^3\) be the linear transformation given by \[T\left( \left[\begin{array}{c}
x_{1} \\
x_{2} \\
x_{3} \\
x_{4}
\end{array}\right] \right) = \left[\begin{array}{c}
x_{1} - x_{2} + x_{3} - 5 \, x_{4} \\
x_{2} + x_{3} + 2 \, x_{4} \\
x_{1} - x_{2} + 2 \, x_{3} - 6 \, x_{4}
\end{array}\right].\] 

 

\begin{enumerate}[(a)]
\item Explain how to find the image of \(T\) and the kernel of \(T\).
\item Explain how to find a basis of the image of \(T\) and a basis of the kernel of \(T\).
\item Explain how to find the rank and nullity of \(T\), and why the rank-nullity theorem holds for \(T\).
\end{enumerate}

     \end{exerciseStatement}
 \begin{exerciseAnswer} 

\[\mathrm{RREF}\,\left[\begin{array}{cccc}
1 & -1 & 1 & -5 \\
0 & 1 & 1 & 2 \\
1 & -1 & 2 & -6
\end{array}\right]=\left[\begin{array}{cccc}
1 & 0 & 0 & -1 \\
0 & 1 & 0 & 3 \\
0 & 0 & 1 & -1
\end{array}\right]\]

 

\begin{enumerate}[(a)]
\item  

 \(\mathrm{Im}\,T =  \left\{ \left[\begin{array}{c}
a - b + c \\
b + c \\
a - b + 2 \, c
\end{array}\right] \middle|\,a,b,c\in\mathbb{R}\right\}\) and \(\mathrm{ker}\,T = \left\{ \left[\begin{array}{c}
a \\
-3 \, a \\
a \\
a
\end{array}\right] \middle|\,a\in\mathbb{R}\right\}\) 

 
\item  

 A basis of \(\mathrm{Im}\,T\) is \(\left\{ \left[\begin{array}{c}
1 \\
0 \\
1
\end{array}\right] , \left[\begin{array}{c}
-1 \\
1 \\
-1
\end{array}\right] , \left[\begin{array}{c}
1 \\
1 \\
2
\end{array}\right] \right\}\). A basis of \(\mathrm{ker}\,T\) is \(\left\{ \left[\begin{array}{c}
1 \\
-3 \\
1 \\
1
\end{array}\right] \right\}\). 

 
\item  

 The rank of \(T\) is \(3\), the nullity of \(T\) is \(1\), and the dimension of the domain of \(T\) is \(4\). The rank-nullity theorem asserts that \(3+1=4\), which we see to be true. 

 
\end{enumerate}

     \end{exerciseAnswer}
 \end{exercise}



\begin{exercise}{AT3}{Image and kernel}{0018} 
\begin{exerciseStatement} 

 Let \(T:\mathbb{R}^4 \to \mathbb{R}^3\) be the linear transformation given by \[T\left( \left[\begin{array}{c}
x \\
y \\
z \\
{w}
\end{array}\right] \right) = \left[\begin{array}{c}
x + 2 \, y - 7 \, {w} \\
-2 \, x - 3 \, y + 2 \, z + 7 \, {w} \\
x - y - 5 \, z + 12 \, {w}
\end{array}\right].\] 

 

\begin{enumerate}[(a)]
\item Explain how to find the image of \(T\) and the kernel of \(T\).
\item Explain how to find a basis of the image of \(T\) and a basis of the kernel of \(T\).
\item Explain how to find the rank and nullity of \(T\), and why the rank-nullity theorem holds for \(T\).
\end{enumerate}

     \end{exerciseStatement}
 \begin{exerciseAnswer} 

\[\mathrm{RREF}\,\left[\begin{array}{cccc}
1 & 2 & 0 & -7 \\
-2 & -3 & 2 & 7 \\
1 & -1 & -5 & 12
\end{array}\right]=\left[\begin{array}{cccc}
1 & 0 & 0 & -1 \\
0 & 1 & 0 & -3 \\
0 & 0 & 1 & -2
\end{array}\right]\]

 

\begin{enumerate}[(a)]
\item  

 \(\mathrm{Im}\,T =  \left\{ \left[\begin{array}{c}
a + 2 \, b \\
-2 \, a - 3 \, b + 2 \, c \\
a - b - 5 \, c
\end{array}\right] \middle|\,a,b,c\in\mathbb{R}\right\}\) and \(\mathrm{ker}\,T = \left\{ \left[\begin{array}{c}
a \\
3 \, a \\
2 \, a \\
a
\end{array}\right] \middle|\,a\in\mathbb{R}\right\}\) 

 
\item  

 A basis of \(\mathrm{Im}\,T\) is \(\left\{ \left[\begin{array}{c}
1 \\
-2 \\
1
\end{array}\right] , \left[\begin{array}{c}
2 \\
-3 \\
-1
\end{array}\right] , \left[\begin{array}{c}
0 \\
2 \\
-5
\end{array}\right] \right\}\). A basis of \(\mathrm{ker}\,T\) is \(\left\{ \left[\begin{array}{c}
1 \\
3 \\
2 \\
1
\end{array}\right] \right\}\). 

 
\item  

 The rank of \(T\) is \(3\), the nullity of \(T\) is \(1\), and the dimension of the domain of \(T\) is \(4\). The rank-nullity theorem asserts that \(3+1=4\), which we see to be true. 

 
\end{enumerate}

     \end{exerciseAnswer}
 \end{exercise}


\newpage




\begin{exercise}{AT3}{Image and kernel}{0045} 
\begin{exerciseStatement} 

 Let \(T:\mathbb{R}^4 \to \mathbb{R}^3\) be the linear transformation given by \[T\left( \left[\begin{array}{c}
x_{1} \\
x_{2} \\
x_{3} \\
x_{4}
\end{array}\right] \right) = \left[\begin{array}{c}
-x_{1} + 2 \, x_{2} + 2 \, x_{4} \\
x_{1} - 2 \, x_{2} - 4 \, x_{3} + 6 \, x_{4} \\
-2 \, x_{1} + 4 \, x_{2} + 5 \, x_{3} - 6 \, x_{4}
\end{array}\right].\] 

 

\begin{enumerate}[(a)]
\item Explain how to find the image of \(T\) and the kernel of \(T\).
\item Explain how to find a basis of the image of \(T\) and a basis of the kernel of \(T\).
\item Explain how to find the rank and nullity of \(T\), and why the rank-nullity theorem holds for \(T\).
\end{enumerate}

     \end{exerciseStatement}
 \begin{exerciseAnswer} 

\[\mathrm{RREF}\,\left[\begin{array}{cccc}
-1 & 2 & 0 & 2 \\
1 & -2 & -4 & 6 \\
-2 & 4 & 5 & -6
\end{array}\right]=\left[\begin{array}{cccc}
1 & -2 & 0 & -2 \\
0 & 0 & 1 & -2 \\
0 & 0 & 0 & 0
\end{array}\right]\]

 

\begin{enumerate}[(a)]
\item  

 \(\mathrm{Im}\,T =  \left\{ \left[\begin{array}{c}
-a \\
a - 4 \, b \\
-2 \, a + 5 \, b
\end{array}\right] \middle|\,a,b\in\mathbb{R}\right\}\) and \(\mathrm{ker}\,T = \left\{ \left[\begin{array}{c}
2 \, a + 2 \, b \\
a \\
2 \, b \\
b
\end{array}\right] \middle|\,a,b\in\mathbb{R}\right\}\) 

 
\item  

 A basis of \(\mathrm{Im}\,T\) is \(\left\{ \left[\begin{array}{c}
-1 \\
1 \\
-2
\end{array}\right] , \left[\begin{array}{c}
0 \\
-4 \\
5
\end{array}\right] \right\}\). A basis of \(\mathrm{ker}\,T\) is \(\left\{ \left[\begin{array}{c}
2 \\
1 \\
0 \\
0
\end{array}\right] , \left[\begin{array}{c}
2 \\
0 \\
2 \\
1
\end{array}\right] \right\}\). 

 
\item  

 The rank of \(T\) is \(2\), the nullity of \(T\) is \(2\), and the dimension of the domain of \(T\) is \(4\). The rank-nullity theorem asserts that \(2+2=4\), which we see to be true. 

 
\end{enumerate}

     \end{exerciseAnswer}
 \end{exercise}



\begin{exercise}{AT3}{Image and kernel}{0151} 
\begin{exerciseStatement} 

 Let \(T:\mathbb{R}^4 \to \mathbb{R}^3\) be the linear transformation given by \[T\left( \left[\begin{array}{c}
x \\
y \\
z \\
{w}
\end{array}\right] \right) = \left[\begin{array}{c}
x + y + z + 3 \, {w} \\
z + {w} \\
-x - y - 5 \, z - 7 \, {w}
\end{array}\right].\] 

 

\begin{enumerate}[(a)]
\item Explain how to find the image of \(T\) and the kernel of \(T\).
\item Explain how to find a basis of the image of \(T\) and a basis of the kernel of \(T\).
\item Explain how to find the rank and nullity of \(T\), and why the rank-nullity theorem holds for \(T\).
\end{enumerate}

     \end{exerciseStatement}
 \begin{exerciseAnswer} 

\[\mathrm{RREF}\,\left[\begin{array}{cccc}
1 & 1 & 1 & 3 \\
0 & 0 & 1 & 1 \\
-1 & -1 & -5 & -7
\end{array}\right]=\left[\begin{array}{cccc}
1 & 1 & 0 & 2 \\
0 & 0 & 1 & 1 \\
0 & 0 & 0 & 0
\end{array}\right]\]

 

\begin{enumerate}[(a)]
\item  

 \(\mathrm{Im}\,T =  \left\{ \left[\begin{array}{c}
a + b \\
b \\
-a - 5 \, b
\end{array}\right] \middle|\,a,b\in\mathbb{R}\right\}\) and \(\mathrm{ker}\,T = \left\{ \left[\begin{array}{c}
-a - 2 \, b \\
a \\
-b \\
b
\end{array}\right] \middle|\,a,b\in\mathbb{R}\right\}\) 

 
\item  

 A basis of \(\mathrm{Im}\,T\) is \(\left\{ \left[\begin{array}{c}
1 \\
0 \\
-1
\end{array}\right] , \left[\begin{array}{c}
1 \\
1 \\
-5
\end{array}\right] \right\}\). A basis of \(\mathrm{ker}\,T\) is \(\left\{ \left[\begin{array}{c}
-1 \\
1 \\
0 \\
0
\end{array}\right] , \left[\begin{array}{c}
-2 \\
0 \\
-1 \\
1
\end{array}\right] \right\}\). 

 
\item  

 The rank of \(T\) is \(2\), the nullity of \(T\) is \(2\), and the dimension of the domain of \(T\) is \(4\). The rank-nullity theorem asserts that \(2+2=4\), which we see to be true. 

 
\end{enumerate}

     \end{exerciseAnswer}
 \end{exercise}


\newpage




\begin{exercise}{AT3}{Image and kernel}{0243} 
\begin{exerciseStatement} 

 Let \(T:\mathbb{R}^4 \to \mathbb{R}^3\) be the linear transformation given by \[T\left( \left[\begin{array}{c}
x \\
y \\
z \\
{w}
\end{array}\right] \right) = \left[\begin{array}{c}
x - 2 \, y - 3 \, z + 4 \, {w} \\
2 \, x - 4 \, y - 5 \, z + 6 \, {w} \\
x - 2 \, y + 2 \, z - 6 \, {w}
\end{array}\right].\] 

 

\begin{enumerate}[(a)]
\item Explain how to find the image of \(T\) and the kernel of \(T\).
\item Explain how to find a basis of the image of \(T\) and a basis of the kernel of \(T\).
\item Explain how to find the rank and nullity of \(T\), and why the rank-nullity theorem holds for \(T\).
\end{enumerate}

     \end{exerciseStatement}
 \begin{exerciseAnswer} 

\[\mathrm{RREF}\,\left[\begin{array}{cccc}
1 & -2 & -3 & 4 \\
2 & -4 & -5 & 6 \\
1 & -2 & 2 & -6
\end{array}\right]=\left[\begin{array}{cccc}
1 & -2 & 0 & -2 \\
0 & 0 & 1 & -2 \\
0 & 0 & 0 & 0
\end{array}\right]\]

 

\begin{enumerate}[(a)]
\item  

 \(\mathrm{Im}\,T =  \left\{ \left[\begin{array}{c}
a - 3 \, b \\
2 \, a - 5 \, b \\
a + 2 \, b
\end{array}\right] \middle|\,a,b\in\mathbb{R}\right\}\) and \(\mathrm{ker}\,T = \left\{ \left[\begin{array}{c}
2 \, a + 2 \, b \\
a \\
2 \, b \\
b
\end{array}\right] \middle|\,a,b\in\mathbb{R}\right\}\) 

 
\item  

 A basis of \(\mathrm{Im}\,T\) is \(\left\{ \left[\begin{array}{c}
1 \\
2 \\
1
\end{array}\right] , \left[\begin{array}{c}
-3 \\
-5 \\
2
\end{array}\right] \right\}\). A basis of \(\mathrm{ker}\,T\) is \(\left\{ \left[\begin{array}{c}
2 \\
1 \\
0 \\
0
\end{array}\right] , \left[\begin{array}{c}
2 \\
0 \\
2 \\
1
\end{array}\right] \right\}\). 

 
\item  

 The rank of \(T\) is \(2\), the nullity of \(T\) is \(2\), and the dimension of the domain of \(T\) is \(4\). The rank-nullity theorem asserts that \(2+2=4\), which we see to be true. 

 
\end{enumerate}

     \end{exerciseAnswer}
 \end{exercise}



\begin{exercise}{AT4}{Injectivity and surjectivity}{0213} 
\begin{exerciseStatement} 

 Let \(T:\mathbb{R}^3 \to \mathbb{R}^5\) be the linear transformation given by the standard matrix \(\left[\begin{array}{ccc}
0 & 0 & 3 \\
0 & 1 & 5 \\
3 & -3 & -5 \\
-1 & 1 & 5 \\
1 & -1 & 2
\end{array}\right]\). 

 

\begin{enumerate}[(a)]
\item 

Explain why \(T\) is or is not injective.


\item 

Explain why \(T\) is or is not surjective.


\end{enumerate}

     \end{exerciseStatement}
 \begin{exerciseAnswer} 

\[\mathrm{RREF}\,\left[\begin{array}{ccc}
0 & 0 & 3 \\
0 & 1 & 5 \\
3 & -3 & -5 \\
-1 & 1 & 5 \\
1 & -1 & 2
\end{array}\right]=\left[\begin{array}{ccc}
1 & 0 & 0 \\
0 & 1 & 0 \\
0 & 0 & 1 \\
0 & 0 & 0 \\
0 & 0 & 0
\end{array}\right]\]

 

\begin{enumerate}[(a)]
\item  

\(T\) is injective.

 
\item  

\(T\) is not surjective.

 
\end{enumerate}

     \end{exerciseAnswer}
 \end{exercise}


\newpage




\begin{exercise}{AT4}{Injectivity and surjectivity}{0207} 
\begin{exerciseStatement} 

 Let \(T:\mathbb{R}^4 \to \mathbb{R}^5\) be the linear transformation given by the standard matrix \(\left[\begin{array}{cccc}
1 & 0 & 2 & 2 \\
0 & 1 & 3 & 0 \\
-1 & 0 & -1 & -2 \\
0 & 1 & 3 & 0 \\
0 & -1 & -4 & 0
\end{array}\right]\). 

 

\begin{enumerate}[(a)]
\item 

Explain why \(T\) is or is not injective.


\item 

Explain why \(T\) is or is not surjective.


\end{enumerate}

     \end{exerciseStatement}
 \begin{exerciseAnswer} 

\[\mathrm{RREF}\,\left[\begin{array}{cccc}
1 & 0 & 2 & 2 \\
0 & 1 & 3 & 0 \\
-1 & 0 & -1 & -2 \\
0 & 1 & 3 & 0 \\
0 & -1 & -4 & 0
\end{array}\right]=\left[\begin{array}{cccc}
1 & 0 & 0 & 2 \\
0 & 1 & 0 & 0 \\
0 & 0 & 1 & 0 \\
0 & 0 & 0 & 0 \\
0 & 0 & 0 & 0
\end{array}\right]\]

 

\begin{enumerate}[(a)]
\item  

\(T\) is not injective.

 
\item  

\(T\) is not surjective.

 
\end{enumerate}

     \end{exerciseAnswer}
 \end{exercise}



\begin{exercise}{AT4}{Injectivity and surjectivity}{0286} 
\begin{exerciseStatement} 

 Let \(T:\mathbb{R}^3 \to \mathbb{R}^3\) be the linear transformation given by the standard matrix \(\left[\begin{array}{ccc}
-1 & 1 & 1 \\
-1 & 0 & -2 \\
2 & 0 & 5
\end{array}\right]\). 

 

\begin{enumerate}[(a)]
\item 

Explain why \(T\) is or is not injective.


\item 

Explain why \(T\) is or is not surjective.


\end{enumerate}

     \end{exerciseStatement}
 \begin{exerciseAnswer} 

\[\mathrm{RREF}\,\left[\begin{array}{ccc}
-1 & 1 & 1 \\
-1 & 0 & -2 \\
2 & 0 & 5
\end{array}\right]=\left[\begin{array}{ccc}
1 & 0 & 0 \\
0 & 1 & 0 \\
0 & 0 & 1
\end{array}\right]\]

 

\begin{enumerate}[(a)]
\item  

\(T\) is injective.

 
\item  

\(T\) is surjective.

 
\end{enumerate}

     \end{exerciseAnswer}
 \end{exercise}


\newpage




\begin{exercise}{AT4}{Injectivity and surjectivity}{0029} 
\begin{exerciseStatement} 

 Let \(T:\mathbb{R}^4 \to \mathbb{R}^3\) be the linear transformation given by the standard matrix \(\left[\begin{array}{cccc}
1 & 1 & 1 & -2 \\
4 & 5 & 1 & -6 \\
-3 & -4 & 1 & 3
\end{array}\right]\). 

 

\begin{enumerate}[(a)]
\item 

Explain why \(T\) is or is not injective.


\item 

Explain why \(T\) is or is not surjective.


\end{enumerate}

     \end{exerciseStatement}
 \begin{exerciseAnswer} 

\[\mathrm{RREF}\,\left[\begin{array}{cccc}
1 & 1 & 1 & -2 \\
4 & 5 & 1 & -6 \\
-3 & -4 & 1 & 3
\end{array}\right]=\left[\begin{array}{cccc}
1 & 0 & 0 & 0 \\
0 & 1 & 0 & -1 \\
0 & 0 & 1 & -1
\end{array}\right]\]

 

\begin{enumerate}[(a)]
\item  

\(T\) is not injective.

 
\item  

\(T\) is surjective.

 
\end{enumerate}

     \end{exerciseAnswer}
 \end{exercise}



\begin{exercise}{AT4}{Injectivity and surjectivity}{0071} 
\begin{exerciseStatement} 

 Let \(T:\mathbb{R}^4 \to \mathbb{R}^4\) be the linear transformation given by the standard matrix \(\left[\begin{array}{cccc}
-3 & 2 & 1 & 4 \\
4 & -3 & 0 & -6 \\
4 & -4 & 5 & -9 \\
0 & 2 & -4 & 0
\end{array}\right]\). 

 

\begin{enumerate}[(a)]
\item 

Explain why \(T\) is or is not injective.


\item 

Explain why \(T\) is or is not surjective.


\end{enumerate}

     \end{exerciseStatement}
 \begin{exerciseAnswer} 

\[\mathrm{RREF}\,\left[\begin{array}{cccc}
-3 & 2 & 1 & 4 \\
4 & -3 & 0 & -6 \\
4 & -4 & 5 & -9 \\
0 & 2 & -4 & 0
\end{array}\right]=\left[\begin{array}{cccc}
1 & 0 & 0 & -3 \\
0 & 1 & 0 & -2 \\
0 & 0 & 1 & -1 \\
0 & 0 & 0 & 0
\end{array}\right]\]

 

\begin{enumerate}[(a)]
\item  

\(T\) is not injective.

 
\item  

\(T\) is not surjective.

 
\end{enumerate}

     \end{exerciseAnswer}
 \end{exercise}


\newpage




\begin{exercise}{MX1}{Multiplying matrices}{0213} 
\begin{exerciseStatement} 

Of the following three matrices, only two may be multiplied. \[
          A=\left[\begin{array}{cc}
1 & 3 \\
1 & 4 \\
-1 & -4 \\
0 & -5
\end{array}\right] \hspace{1em} B=\left[\begin{array}{ccc}
-1 & 2 & -4 \\
0 & 1 & 3 \\
0 & -2 & -5 \\
1 & -1 & 4
\end{array}\right] \hspace{1em} C=\left[\begin{array}{cc}
1 & -4 \\
3 & -5 \\
0 & 1
\end{array}\right]
      \] Explain which two can be multiplied and why. Then show how to find their product.

 \end{exerciseStatement}
 \begin{exerciseAnswer} \[BC=\left[\begin{array}{cc}
5 & -10 \\
3 & -2 \\
-6 & 5 \\
-2 & 5
\end{array}\right]\] \end{exerciseAnswer}
 \end{exercise}



\begin{exercise}{MX1}{Multiplying matrices}{0082} 
\begin{exerciseStatement} 

Of the following three matrices, only two may be multiplied. \[
          A=\left[\begin{array}{cc}
1 & 2 \\
0 & 1 \\
0 & 2
\end{array}\right] \hspace{1em} B=\left[\begin{array}{cc}
3 & 1 \\
1 & -2 \\
-1 & -2 \\
-1 & 2
\end{array}\right] \hspace{1em} C=\left[\begin{array}{cccc}
1 & 0 & -2 & 3 \\
-3 & 1 & 2 & -5 \\
1 & -1 & 3 & -2
\end{array}\right]
      \] Explain which two can be multiplied and why. Then show how to find their product.

 \end{exerciseStatement}
 \begin{exerciseAnswer} \[CB=\left[\begin{array}{cc}
2 & 11 \\
-5 & -19 \\
1 & -7
\end{array}\right]\] \end{exerciseAnswer}
 \end{exercise}


\newpage




\begin{exercise}{MX1}{Multiplying matrices}{0081} 
\begin{exerciseStatement} 

Of the following three matrices, only two may be multiplied. \[
          A=\left[\begin{array}{cccc}
1 & 0 & 1 & -3 \\
-1 & 1 & -1 & 6 \\
-2 & 1 & -1 & 6
\end{array}\right] \hspace{1em} B=\left[\begin{array}{cc}
1 & 0 \\
5 & 1 \\
-3 & -5
\end{array}\right] \hspace{1em} C=\left[\begin{array}{cccc}
1 & 1 & -3 & -5 \\
-1 & 0 & 1 & 2
\end{array}\right]
      \] Explain which two can be multiplied and why. Then show how to find their product.

 \end{exerciseStatement}
 \begin{exerciseAnswer} \[BC=\left[\begin{array}{cccc}
1 & 1 & -3 & -5 \\
4 & 5 & -14 & -23 \\
2 & -3 & 4 & 5
\end{array}\right]\] \end{exerciseAnswer}
 \end{exercise}



\begin{exercise}{MX1}{Multiplying matrices}{0146} 
\begin{exerciseStatement} 

Of the following three matrices, only two may be multiplied. \[
          A=\left[\begin{array}{ccc}
-3 & 1 & -2 \\
-4 & 1 & -3
\end{array}\right] \hspace{1em} B=\left[\begin{array}{ccc}
1 & 0 & 0 \\
-5 & 1 & -3 \\
5 & 1 & -2 \\
5 & -1 & 0
\end{array}\right] \hspace{1em} C=\left[\begin{array}{cccc}
1 & 1 & 6 & -2 \\
0 & 1 & 4 & -1
\end{array}\right]
      \] Explain which two can be multiplied and why. Then show how to find their product.

 \end{exerciseStatement}
 \begin{exerciseAnswer} \[CB=\left[\begin{array}{ccc}
16 & 9 & -15 \\
10 & 6 & -11
\end{array}\right]\] \end{exerciseAnswer}
 \end{exercise}


\newpage




\begin{exercise}{MX1}{Multiplying matrices}{0172} 
\begin{exerciseStatement} 

Of the following three matrices, only two may be multiplied. \[
          A=\left[\begin{array}{cccc}
1 & 0 & 1 & 0 \\
2 & 1 & 1 & 0 \\
4 & -1 & 6 & 2
\end{array}\right] \hspace{1em} B=\left[\begin{array}{cccc}
-1 & 2 & -5 & -1 \\
-1 & 1 & -2 & 1
\end{array}\right] \hspace{1em} C=\left[\begin{array}{ccc}
1 & -6 & 1 \\
0 & 1 & 0
\end{array}\right]
      \] Explain which two can be multiplied and why. Then show how to find their product.

 \end{exerciseStatement}
 \begin{exerciseAnswer} \[CA=\left[\begin{array}{cccc}
-7 & -7 & 1 & 2 \\
2 & 1 & 1 & 0
\end{array}\right]\] \end{exerciseAnswer}
 \end{exercise}



\begin{exercise}{MX2}{Row operations as matrix multiplication}{0280} 
\begin{exerciseStatement} 

Let \(A\) be a \(4 \times 4\) matrix.

 

\begin{enumerate}[(a)]
\item Give a \(4 \times 4\) matrix \(Q\) that may be used to perform the row operation \(R_2 - 5 R_4 \to R_2\).
\item Give a \(4 \times 4\) matrix \(N\) that may be used to perform the row operation \(5 R_2 \to R_2\).
\item Use matrix multiplication to describe the matrix obtained by applying \(5 R_2 \to R_2\) and then \(R_2 - 5 R_4 \to R_2\) to \(A\) (note the order). 
\end{enumerate}

     \end{exerciseStatement}
 \begin{exerciseAnswer} 

\begin{enumerate}[(a)]
\item \(Q=\left[\begin{array}{cccc}
1 & 0 & 0 & 0 \\
0 & 1 & 0 & -5 \\
0 & 0 & 1 & 0 \\
0 & 0 & 0 & 1
\end{array}\right]\)
\item \(N=\left[\begin{array}{cccc}
1 & 0 & 0 & 0 \\
0 & 5 & 0 & 0 \\
0 & 0 & 1 & 0 \\
0 & 0 & 0 & 1
\end{array}\right]\)
\item  \(QNA\) 
\end{enumerate}

     \end{exerciseAnswer}
 \end{exercise}


\newpage




\begin{exercise}{MX2}{Row operations as matrix multiplication}{0010} 
\begin{exerciseStatement} 

Let \(A\) be a \(4 \times 4\) matrix.

 

\begin{enumerate}[(a)]
\item Give a \(4 \times 4\) matrix \(B\) that may be used to perform the row operation \(R_4 \leftrightarrow R_2\).
\item Give a \(4 \times 4\) matrix \(N\) that may be used to perform the row operation \(5 R_4 \to R_4\).
\item Use matrix multiplication to describe the matrix obtained by applying \(5 R_4 \to R_4\) and then \(R_4 \leftrightarrow R_2\) to \(A\) (note the order). 
\end{enumerate}

     \end{exerciseStatement}
 \begin{exerciseAnswer} 

\begin{enumerate}[(a)]
\item \(B=\left[\begin{array}{cccc}
1 & 0 & 0 & 0 \\
0 & 0 & 0 & 1 \\
0 & 0 & 1 & 0 \\
0 & 1 & 0 & 0
\end{array}\right]\)
\item \(N=\left[\begin{array}{cccc}
1 & 0 & 0 & 0 \\
0 & 1 & 0 & 0 \\
0 & 0 & 1 & 0 \\
0 & 0 & 0 & 5
\end{array}\right]\)
\item  \(BNA\) 
\end{enumerate}

     \end{exerciseAnswer}
 \end{exercise}



\begin{exercise}{MX2}{Row operations as matrix multiplication}{0234} 
\begin{exerciseStatement} 

Let \(A\) be a \(4 \times 4\) matrix.

 

\begin{enumerate}[(a)]
\item Give a \(4 \times 4\) matrix \(P\) that may be used to perform the row operation \(R_3 + 4 R_2 \to R_3\).
\item Give a \(4 \times 4\) matrix \(N\) that may be used to perform the row operation \(2 R_3 \to R_3\).
\item Use matrix multiplication to describe the matrix obtained by applying \(2 R_3 \to R_3\) and then \(R_3 + 4 R_2 \to R_3\) to \(A\) (note the order). 
\end{enumerate}

     \end{exerciseStatement}
 \begin{exerciseAnswer} 

\begin{enumerate}[(a)]
\item \(P=\left[\begin{array}{cccc}
1 & 0 & 0 & 0 \\
0 & 1 & 0 & 0 \\
0 & 4 & 1 & 0 \\
0 & 0 & 0 & 1
\end{array}\right]\)
\item \(N=\left[\begin{array}{cccc}
1 & 0 & 0 & 0 \\
0 & 1 & 0 & 0 \\
0 & 0 & 2 & 0 \\
0 & 0 & 0 & 1
\end{array}\right]\)
\item  \(PNA\) 
\end{enumerate}

     \end{exerciseAnswer}
 \end{exercise}


\newpage




\begin{exercise}{MX2}{Row operations as matrix multiplication}{0279} 
\begin{exerciseStatement} 

Let \(A\) be a \(4 \times 4\) matrix.

 

\begin{enumerate}[(a)]
\item Give a \(4 \times 4\) matrix \(Q\) that may be used to perform the row operation \(-5 R_1 \to R_1\).
\item Give a \(4 \times 4\) matrix \(B\) that may be used to perform the row operation \(R_1 - 4 R_3 \to R_1\).
\item Use matrix multiplication to describe the matrix obtained by applying \(-5 R_1 \to R_1\) and then \(R_1 - 4 R_3 \to R_1\) to \(A\) (note the order). 
\end{enumerate}

     \end{exerciseStatement}
 \begin{exerciseAnswer} 

\begin{enumerate}[(a)]
\item \(Q=\left[\begin{array}{cccc}
-5 & 0 & 0 & 0 \\
0 & 1 & 0 & 0 \\
0 & 0 & 1 & 0 \\
0 & 0 & 0 & 1
\end{array}\right]\)
\item \(B=\left[\begin{array}{cccc}
1 & 0 & -4 & 0 \\
0 & 1 & 0 & 0 \\
0 & 0 & 1 & 0 \\
0 & 0 & 0 & 1
\end{array}\right]\)
\item  \(BQA\) 
\end{enumerate}

     \end{exerciseAnswer}
 \end{exercise}



\begin{exercise}{MX2}{Row operations as matrix multiplication}{0212} 
\begin{exerciseStatement} 

Let \(A\) be a \(4 \times 4\) matrix.

 

\begin{enumerate}[(a)]
\item Give a \(4 \times 4\) matrix \(M\) that may be used to perform the row operation \(R_3 \leftrightarrow R_1\).
\item Give a \(4 \times 4\) matrix \(Q\) that may be used to perform the row operation \(3 R_3 \to R_3\).
\item Use matrix multiplication to describe the matrix obtained by applying \(R_3 \leftrightarrow R_1\) and then \(3 R_3 \to R_3\) to \(A\) (note the order). 
\end{enumerate}

     \end{exerciseStatement}
 \begin{exerciseAnswer} 

\begin{enumerate}[(a)]
\item \(M=\left[\begin{array}{cccc}
0 & 0 & 1 & 0 \\
0 & 1 & 0 & 0 \\
1 & 0 & 0 & 0 \\
0 & 0 & 0 & 1
\end{array}\right]\)
\item \(Q=\left[\begin{array}{cccc}
1 & 0 & 0 & 0 \\
0 & 1 & 0 & 0 \\
0 & 0 & 3 & 0 \\
0 & 0 & 0 & 1
\end{array}\right]\)
\item  \(QMA\) 
\end{enumerate}

     \end{exerciseAnswer}
 \end{exercise}


\newpage




\begin{exercise}{MX2}{Row operations as matrix multiplication}{0143} 
\begin{exerciseStatement} 

Let \(A\) be a \(4 \times 4\) matrix.

 

\begin{enumerate}[(a)]
\item Give a \(4 \times 4\) matrix \(N\) that may be used to perform the row operation \(R_4 - 2 R_1 \to R_4\).
\item Give a \(4 \times 4\) matrix \(C\) that may be used to perform the row operation \(R_2 \leftrightarrow R_4\).
\item Use matrix multiplication to describe the matrix obtained by applying \(R_2 \leftrightarrow R_4\) and then \(R_4 - 2 R_1 \to R_4\) to \(A\) (note the order). 
\end{enumerate}

     \end{exerciseStatement}
 \begin{exerciseAnswer} 

\begin{enumerate}[(a)]
\item \(N=\left[\begin{array}{cccc}
1 & 0 & 0 & 0 \\
0 & 1 & 0 & 0 \\
0 & 0 & 1 & 0 \\
-2 & 0 & 0 & 1
\end{array}\right]\)
\item \(C=\left[\begin{array}{cccc}
1 & 0 & 0 & 0 \\
0 & 0 & 0 & 1 \\
0 & 0 & 1 & 0 \\
0 & 1 & 0 & 0
\end{array}\right]\)
\item  \(NCA\) 
\end{enumerate}

     \end{exerciseAnswer}
 \end{exercise}



\begin{exercise}{MX2}{Row operations as matrix multiplication}{0147} 
\begin{exerciseStatement} 

Let \(A\) be a \(4 \times 4\) matrix.

 

\begin{enumerate}[(a)]
\item Give a \(4 \times 4\) matrix \(B\) that may be used to perform the row operation \(4 R_4 \to R_4\).
\item Give a \(4 \times 4\) matrix \(M\) that may be used to perform the row operation \(R_4 + 2 R_1 \to R_4\).
\item Use matrix multiplication to describe the matrix obtained by applying \(4 R_4 \to R_4\) and then \(R_4 + 2 R_1 \to R_4\) to \(A\) (note the order). 
\end{enumerate}

     \end{exerciseStatement}
 \begin{exerciseAnswer} 

\begin{enumerate}[(a)]
\item \(B=\left[\begin{array}{cccc}
1 & 0 & 0 & 0 \\
0 & 1 & 0 & 0 \\
0 & 0 & 1 & 0 \\
0 & 0 & 0 & 4
\end{array}\right]\)
\item \(M=\left[\begin{array}{cccc}
1 & 0 & 0 & 0 \\
0 & 1 & 0 & 0 \\
0 & 0 & 1 & 0 \\
2 & 0 & 0 & 1
\end{array}\right]\)
\item  \(MBA\) 
\end{enumerate}

     \end{exerciseAnswer}
 \end{exercise}


\newpage




\begin{exercise}{MX2}{Row operations as matrix multiplication}{0235} 
\begin{exerciseStatement} 

Let \(A\) be a \(4 \times 4\) matrix.

 

\begin{enumerate}[(a)]
\item Give a \(4 \times 4\) matrix \(M\) that may be used to perform the row operation \(R_1 \leftrightarrow R_3\).
\item Give a \(4 \times 4\) matrix \(C\) that may be used to perform the row operation \(-4 R_1 \to R_1\).
\item Use matrix multiplication to describe the matrix obtained by applying \(-4 R_1 \to R_1\) and then \(R_1 \leftrightarrow R_3\) to \(A\) (note the order). 
\end{enumerate}

     \end{exerciseStatement}
 \begin{exerciseAnswer} 

\begin{enumerate}[(a)]
\item \(M=\left[\begin{array}{cccc}
0 & 0 & 1 & 0 \\
0 & 1 & 0 & 0 \\
1 & 0 & 0 & 0 \\
0 & 0 & 0 & 1
\end{array}\right]\)
\item \(C=\left[\begin{array}{cccc}
-4 & 0 & 0 & 0 \\
0 & 1 & 0 & 0 \\
0 & 0 & 1 & 0 \\
0 & 0 & 0 & 1
\end{array}\right]\)
\item  \(MCA\) 
\end{enumerate}

     \end{exerciseAnswer}
 \end{exercise}



\begin{exercise}{MX2}{Row operations as matrix multiplication}{0103} 
\begin{exerciseStatement} 

Let \(A\) be a \(4 \times 4\) matrix.

 

\begin{enumerate}[(a)]
\item Give a \(4 \times 4\) matrix \(B\) that may be used to perform the row operation \(R_4 \leftrightarrow R_2\).
\item Give a \(4 \times 4\) matrix \(N\) that may be used to perform the row operation \(R_2 - 2 R_1 \to R_2\).
\item Use matrix multiplication to describe the matrix obtained by applying \(R_4 \leftrightarrow R_2\) and then \(R_2 - 2 R_1 \to R_2\) to \(A\) (note the order). 
\end{enumerate}

     \end{exerciseStatement}
 \begin{exerciseAnswer} 

\begin{enumerate}[(a)]
\item \(B=\left[\begin{array}{cccc}
1 & 0 & 0 & 0 \\
0 & 0 & 0 & 1 \\
0 & 0 & 1 & 0 \\
0 & 1 & 0 & 0
\end{array}\right]\)
\item \(N=\left[\begin{array}{cccc}
1 & 0 & 0 & 0 \\
-2 & 1 & 0 & 0 \\
0 & 0 & 1 & 0 \\
0 & 0 & 0 & 1
\end{array}\right]\)
\item  \(NBA\) 
\end{enumerate}

     \end{exerciseAnswer}
 \end{exercise}


\newpage




\begin{exercise}{MX3}{The inverse of a matrix}{0169} 
\begin{exerciseStatement} 

 Explain why each of the following matrices is or is not invertible by disussing its corresponding linear transformation. If the matrix is invertible, explain how to find its inverse. \[
\hspace{2em}
B = \left[\begin{array}{cccc}
1 & -2 & -3 & 1 \\
3 & -5 & -6 & 4 \\
0 & -2 & -6 & -2 \\
0 & 3 & 9 & 3
\end{array}\right]
\hspace{2em}
C = \left[\begin{array}{cccc}
-5 & 3 & 5 & -1 \\
-2 & 1 & 2 & -1 \\
-2 & -1 & 3 & -4 \\
0 & 0 & -1 & -2
\end{array}\right]
\hspace{2em}
        \] 

 \end{exerciseStatement}
 \begin{exerciseAnswer} 

 \[\mathrm{RREF}\,\left[\begin{array}{cccc}
1 & -2 & -3 & 1 \\
3 & -5 & -6 & 4 \\
0 & -2 & -6 & -2 \\
0 & 3 & 9 & 3
\end{array}\right]=\left[\begin{array}{cccc}
1 & 0 & 3 & 3 \\
0 & 1 & 3 & 1 \\
0 & 0 & 0 & 0 \\
0 & 0 & 0 & 0
\end{array}\right]\] \(B\) is not invertible. 

 

 \[\mathrm{RREF}\,\left[\begin{array}{cccc}
-5 & 3 & 5 & -1 \\
-2 & 1 & 2 & -1 \\
-2 & -1 & 3 & -4 \\
0 & 0 & -1 & -2
\end{array}\right]=\left[\begin{array}{cccc}
1 & 0 & 0 & 0 \\
0 & 1 & 0 & 0 \\
0 & 0 & 1 & 0 \\
0 & 0 & 0 & 1
\end{array}\right]\] \(C\) is invertible. Its inverse is \(\left[\begin{array}{cccc}
-15 & 41 & -4 & -5 \\
-10 & 28 & -3 & -3 \\
-8 & 22 & -2 & -3 \\
4 & -11 & 1 & 1
\end{array}\right]\). 

 \end{exerciseAnswer}
 \end{exercise}



\begin{exercise}{MX3}{The inverse of a matrix}{0160} 
\begin{exerciseStatement} 

 Explain why each of the following matrices is or is not invertible by disussing its corresponding linear transformation. If the matrix is invertible, explain how to find its inverse. \[
\hspace{2em}
Q = \left[\begin{array}{cccc}
3 & -4 & 2 & -12 \\
1 & -1 & 0 & -3 \\
-3 & 4 & -2 & 12 \\
-2 & 4 & -4 & 12
\end{array}\right]
\hspace{2em}
L = \left[\begin{array}{cccc}
0 & -1 & 1 & 0 \\
1 & -3 & 0 & -3 \\
0 & 4 & -3 & 0 \\
0 & -1 & -4 & 1
\end{array}\right]
\hspace{2em}
        \] 

 \end{exerciseStatement}
 \begin{exerciseAnswer} 

 \[\mathrm{RREF}\,\left[\begin{array}{cccc}
3 & -4 & 2 & -12 \\
1 & -1 & 0 & -3 \\
-3 & 4 & -2 & 12 \\
-2 & 4 & -4 & 12
\end{array}\right]=\left[\begin{array}{cccc}
1 & 0 & -2 & 0 \\
0 & 1 & -2 & 3 \\
0 & 0 & 0 & 0 \\
0 & 0 & 0 & 0
\end{array}\right]\] \(Q\) is not invertible. 

 

 \[\mathrm{RREF}\,\left[\begin{array}{cccc}
0 & -1 & 1 & 0 \\
1 & -3 & 0 & -3 \\
0 & 4 & -3 & 0 \\
0 & -1 & -4 & 1
\end{array}\right]=\left[\begin{array}{cccc}
1 & 0 & 0 & 0 \\
0 & 1 & 0 & 0 \\
0 & 0 & 1 & 0 \\
0 & 0 & 0 & 1
\end{array}\right]\] \(L\) is invertible. Its inverse is \(\left[\begin{array}{cccc}
66 & 1 & 18 & 3 \\
3 & 0 & 1 & 0 \\
4 & 0 & 1 & 0 \\
19 & 0 & 5 & 1
\end{array}\right]\). 

 \end{exerciseAnswer}
 \end{exercise}


\newpage




\begin{exercise}{MX3}{The inverse of a matrix}{0277} 
\begin{exerciseStatement} 

 Explain why each of the following matrices is or is not invertible by disussing its corresponding linear transformation. If the matrix is invertible, explain how to find its inverse. \[
\hspace{2em}
A = \left[\begin{array}{cccc}
1 & 1 & -2 & -4 \\
0 & 1 & 0 & 0 \\
0 & -1 & 1 & 4 \\
0 & 5 & -1 & -3
\end{array}\right]
\hspace{2em}
C = \left[\begin{array}{cccc}
1 & 1 & 0 & 3 \\
1 & 2 & -1 & 6 \\
2 & -3 & 5 & -9 \\
-4 & 0 & -4 & 0
\end{array}\right]
\hspace{2em}
        \] 

 \end{exerciseStatement}
 \begin{exerciseAnswer} 

 \[\mathrm{RREF}\,\left[\begin{array}{cccc}
1 & 1 & -2 & -4 \\
0 & 1 & 0 & 0 \\
0 & -1 & 1 & 4 \\
0 & 5 & -1 & -3
\end{array}\right]=\left[\begin{array}{cccc}
1 & 0 & 0 & 0 \\
0 & 1 & 0 & 0 \\
0 & 0 & 1 & 0 \\
0 & 0 & 0 & 1
\end{array}\right]\] \(A\) is invertible. Its inverse is \(\left[\begin{array}{cccc}
1 & 17 & -2 & -4 \\
0 & 1 & 0 & 0 \\
0 & 17 & -3 & -4 \\
0 & -4 & 1 & 1
\end{array}\right]\). 

 

 \[\mathrm{RREF}\,\left[\begin{array}{cccc}
1 & 1 & 0 & 3 \\
1 & 2 & -1 & 6 \\
2 & -3 & 5 & -9 \\
-4 & 0 & -4 & 0
\end{array}\right]=\left[\begin{array}{cccc}
1 & 0 & 1 & 0 \\
0 & 1 & -1 & 3 \\
0 & 0 & 0 & 0 \\
0 & 0 & 0 & 0
\end{array}\right]\] \(C\) is not invertible. 

 \end{exerciseAnswer}
 \end{exercise}



\begin{exercise}{MX3}{The inverse of a matrix}{0009} 
\begin{exerciseStatement} 

 Explain why each of the following matrices is or is not invertible by disussing its corresponding linear transformation. If the matrix is invertible, explain how to find its inverse. \[
\hspace{2em}
D = \left[\begin{array}{cccc}
-1 & 1 & 1 & -1 \\
-2 & 2 & 1 & -4 \\
2 & -2 & -3 & 1 \\
1 & -1 & 1 & 1
\end{array}\right]
\hspace{2em}
A = \left[\begin{array}{cccc}
-3 & 1 & 1 & 1 \\
4 & -3 & 0 & -3 \\
-4 & 2 & 1 & 3 \\
-4 & 3 & 0 & 4
\end{array}\right]
\hspace{2em}
        \] 

 \end{exerciseStatement}
 \begin{exerciseAnswer} 

 \[\mathrm{RREF}\,\left[\begin{array}{cccc}
-1 & 1 & 1 & -1 \\
-2 & 2 & 1 & -4 \\
2 & -2 & -3 & 1 \\
1 & -1 & 1 & 1
\end{array}\right]=\left[\begin{array}{cccc}
1 & -1 & 0 & 0 \\
0 & 0 & 1 & 0 \\
0 & 0 & 0 & 1 \\
0 & 0 & 0 & 0
\end{array}\right]\] \(D\) is not invertible. 

 

 \[\mathrm{RREF}\,\left[\begin{array}{cccc}
-3 & 1 & 1 & 1 \\
4 & -3 & 0 & -3 \\
-4 & 2 & 1 & 3 \\
-4 & 3 & 0 & 4
\end{array}\right]=\left[\begin{array}{cccc}
1 & 0 & 0 & 0 \\
0 & 1 & 0 & 0 \\
0 & 0 & 1 & 0 \\
0 & 0 & 0 & 1
\end{array}\right]\] \(A\) is invertible. Its inverse is \(\left[\begin{array}{cccc}
-3 & -2 & 3 & -3 \\
-4 & -4 & 4 & -5 \\
-4 & -3 & 5 & -5 \\
0 & 1 & 0 & 1
\end{array}\right]\). 

 \end{exerciseAnswer}
 \end{exercise}


\newpage




\begin{exercise}{MX3}{The inverse of a matrix}{0251} 
\begin{exerciseStatement} 

 Explain why each of the following matrices is or is not invertible by disussing its corresponding linear transformation. If the matrix is invertible, explain how to find its inverse. \[
\hspace{2em}
M = \left[\begin{array}{cccc}
0 & -1 & -2 & 2 \\
0 & 1 & 1 & -4 \\
1 & 0 & -3 & 1 \\
1 & 0 & -3 & 2
\end{array}\right]
\hspace{2em}
N = \left[\begin{array}{cccc}
0 & 0 & 0 & 0 \\
-1 & -4 & 1 & 8 \\
1 & 5 & -2 & -10 \\
-1 & -3 & 0 & 6
\end{array}\right]
\hspace{2em}
        \] 

 \end{exerciseStatement}
 \begin{exerciseAnswer} 

 \[\mathrm{RREF}\,\left[\begin{array}{cccc}
0 & -1 & -2 & 2 \\
0 & 1 & 1 & -4 \\
1 & 0 & -3 & 1 \\
1 & 0 & -3 & 2
\end{array}\right]=\left[\begin{array}{cccc}
1 & 0 & 0 & 0 \\
0 & 1 & 0 & 0 \\
0 & 0 & 1 & 0 \\
0 & 0 & 0 & 1
\end{array}\right]\] \(M\) is invertible. Its inverse is \(\left[\begin{array}{cccc}
-3 & -3 & 8 & -7 \\
1 & 2 & -6 & 6 \\
-1 & -1 & 2 & -2 \\
0 & 0 & -1 & 1
\end{array}\right]\). 

 

 \[\mathrm{RREF}\,\left[\begin{array}{cccc}
0 & 0 & 0 & 0 \\
-1 & -4 & 1 & 8 \\
1 & 5 & -2 & -10 \\
-1 & -3 & 0 & 6
\end{array}\right]=\left[\begin{array}{cccc}
1 & 0 & 3 & 0 \\
0 & 1 & -1 & -2 \\
0 & 0 & 0 & 0 \\
0 & 0 & 0 & 0
\end{array}\right]\] \(N\) is not invertible. 

 \end{exerciseAnswer}
 \end{exercise}



\begin{exercise}{GT1}{Row operations and determinants}{0100} 
\begin{exerciseStatement} 

Let \(A\) be a \(4 \times 4\) matrix with determinant \(-4\).

 

\begin{enumerate}[(a)]
\item Let \(C\) be the matrix obtained from \(A\) by applying the row operation \(R_2 \leftrightarrow R_3\). What is \(\mathrm{det}\,C\)?
\item Let \(B\) be the matrix obtained from \(A\) by applying the row operation \(-4 R_1 \to R_1\). What is \(\mathrm{det}\,B\)?
\item Let \(M\) be the matrix obtained from \(A\) by applying the row operation \(R_4 + 4 R_3 \to R_4\). What is \(\mathrm{det}\,M\)?
\end{enumerate}

     \end{exerciseStatement}
 \begin{exerciseAnswer} 

\begin{enumerate}[(a)]
\item \(\mathrm{det}\,C=4\)
\item \(\mathrm{det}\,B=16\)
\item \(\mathrm{det}\,M=-4\)
\end{enumerate}

     \end{exerciseAnswer}
 \end{exercise}


\newpage




\begin{exercise}{GT1}{Row operations and determinants}{0187} 
\begin{exerciseStatement} 

Let \(A\) be a \(4 \times 4\) matrix with determinant \(2\).

 

\begin{enumerate}[(a)]
\item Let \(P\) be the matrix obtained from \(A\) by applying the row operation \(-3 R_1 \to R_1\). What is \(\mathrm{det}\,P\)?
\item Let \(M\) be the matrix obtained from \(A\) by applying the row operation \(R_4 + 2 R_1 \to R_4\). What is \(\mathrm{det}\,M\)?
\item Let \(N\) be the matrix obtained from \(A\) by applying the row operation \(R_3 \leftrightarrow R_1\). What is \(\mathrm{det}\,N\)?
\end{enumerate}

     \end{exerciseStatement}
 \begin{exerciseAnswer} 

\begin{enumerate}[(a)]
\item \(\mathrm{det}\,P=-6\)
\item \(\mathrm{det}\,M=2\)
\item \(\mathrm{det}\,N=-2\)
\end{enumerate}

     \end{exerciseAnswer}
 \end{exercise}



\begin{exercise}{GT1}{Row operations and determinants}{0104} 
\begin{exerciseStatement} 

Let \(A\) be a \(4 \times 4\) matrix with determinant \(-7\).

 

\begin{enumerate}[(a)]
\item Let \(B\) be the matrix obtained from \(A\) by applying the row operation \(R_3 \leftrightarrow R_4\). What is \(\mathrm{det}\,B\)?
\item Let \(M\) be the matrix obtained from \(A\) by applying the row operation \(R_1 + 4 R_4 \to R_1\). What is \(\mathrm{det}\,M\)?
\item Let \(C\) be the matrix obtained from \(A\) by applying the row operation \(-4 R_3 \to R_3\). What is \(\mathrm{det}\,C\)?
\end{enumerate}

     \end{exerciseStatement}
 \begin{exerciseAnswer} 

\begin{enumerate}[(a)]
\item \(\mathrm{det}\,B=7\)
\item \(\mathrm{det}\,M=-7\)
\item \(\mathrm{det}\,C=28\)
\end{enumerate}

     \end{exerciseAnswer}
 \end{exercise}


\newpage




\begin{exercise}{GT1}{Row operations and determinants}{0057} 
\begin{exerciseStatement} 

Let \(A\) be a \(4 \times 4\) matrix with determinant \(-7\).

 

\begin{enumerate}[(a)]
\item Let \(C\) be the matrix obtained from \(A\) by applying the row operation \(2 R_3 \to R_3\). What is \(\mathrm{det}\,C\)?
\item Let \(P\) be the matrix obtained from \(A\) by applying the row operation \(R_1 \leftrightarrow R_3\). What is \(\mathrm{det}\,P\)?
\item Let \(B\) be the matrix obtained from \(A\) by applying the row operation \(R_3 + 2 R_2 \to R_3\). What is \(\mathrm{det}\,B\)?
\end{enumerate}

     \end{exerciseStatement}
 \begin{exerciseAnswer} 

\begin{enumerate}[(a)]
\item \(\mathrm{det}\,C=-14\)
\item \(\mathrm{det}\,P=7\)
\item \(\mathrm{det}\,B=-7\)
\end{enumerate}

     \end{exerciseAnswer}
 \end{exercise}



\begin{exercise}{GT1}{Row operations and determinants}{0116} 
\begin{exerciseStatement} 

Let \(A\) be a \(4 \times 4\) matrix with determinant \(-7\).

 

\begin{enumerate}[(a)]
\item Let \(M\) be the matrix obtained from \(A\) by applying the row operation \(2 R_3 \to R_3\). What is \(\mathrm{det}\,M\)?
\item Let \(Q\) be the matrix obtained from \(A\) by applying the row operation \(R_1 \leftrightarrow R_2\). What is \(\mathrm{det}\,Q\)?
\item Let \(B\) be the matrix obtained from \(A\) by applying the row operation \(R_4 + 5 R_1 \to R_4\). What is \(\mathrm{det}\,B\)?
\end{enumerate}

     \end{exerciseStatement}
 \begin{exerciseAnswer} 

\begin{enumerate}[(a)]
\item \(\mathrm{det}\,M=-14\)
\item \(\mathrm{det}\,Q=7\)
\item \(\mathrm{det}\,B=-7\)
\end{enumerate}

     \end{exerciseAnswer}
 \end{exercise}


\newpage




\begin{exercise}{GT1}{Row operations and determinants}{0273} 
\begin{exerciseStatement} 

Let \(A\) be a \(4 \times 4\) matrix with determinant \(-5\).

 

\begin{enumerate}[(a)]
\item Let \(N\) be the matrix obtained from \(A\) by applying the row operation \(-5 R_1 \to R_1\). What is \(\mathrm{det}\,N\)?
\item Let \(P\) be the matrix obtained from \(A\) by applying the row operation \(R_4 + 2 R_1 \to R_4\). What is \(\mathrm{det}\,P\)?
\item Let \(C\) be the matrix obtained from \(A\) by applying the row operation \(R_3 \leftrightarrow R_4\). What is \(\mathrm{det}\,C\)?
\end{enumerate}

     \end{exerciseStatement}
 \begin{exerciseAnswer} 

\begin{enumerate}[(a)]
\item \(\mathrm{det}\,N=25\)
\item \(\mathrm{det}\,P=-5\)
\item \(\mathrm{det}\,C=5\)
\end{enumerate}

     \end{exerciseAnswer}
 \end{exercise}



\begin{exercise}{GT1}{Row operations and determinants}{0126} 
\begin{exerciseStatement} 

Let \(A\) be a \(4 \times 4\) matrix with determinant \(-4\).

 

\begin{enumerate}[(a)]
\item Let \(B\) be the matrix obtained from \(A\) by applying the row operation \(-5 R_2 \to R_2\). What is \(\mathrm{det}\,B\)?
\item Let \(C\) be the matrix obtained from \(A\) by applying the row operation \(R_2 \leftrightarrow R_3\). What is \(\mathrm{det}\,C\)?
\item Let \(P\) be the matrix obtained from \(A\) by applying the row operation \(R_4 + 2 R_1 \to R_4\). What is \(\mathrm{det}\,P\)?
\end{enumerate}

     \end{exerciseStatement}
 \begin{exerciseAnswer} 

\begin{enumerate}[(a)]
\item \(\mathrm{det}\,B=20\)
\item \(\mathrm{det}\,C=4\)
\item \(\mathrm{det}\,P=-4\)
\end{enumerate}

     \end{exerciseAnswer}
 \end{exercise}


\newpage




\begin{exercise}{GT1}{Row operations and determinants}{0019} 
\begin{exerciseStatement} 

Let \(A\) be a \(4 \times 4\) matrix with determinant \(5\).

 

\begin{enumerate}[(a)]
\item Let \(P\) be the matrix obtained from \(A\) by applying the row operation \(R_4 \leftrightarrow R_1\). What is \(\mathrm{det}\,P\)?
\item Let \(B\) be the matrix obtained from \(A\) by applying the row operation \(R_4 - 5 R_1 \to R_4\). What is \(\mathrm{det}\,B\)?
\item Let \(M\) be the matrix obtained from \(A\) by applying the row operation \(4 R_3 \to R_3\). What is \(\mathrm{det}\,M\)?
\end{enumerate}

     \end{exerciseStatement}
 \begin{exerciseAnswer} 

\begin{enumerate}[(a)]
\item \(\mathrm{det}\,P=-5\)
\item \(\mathrm{det}\,B=5\)
\item \(\mathrm{det}\,M=20\)
\end{enumerate}

     \end{exerciseAnswer}
 \end{exercise}



\begin{exercise}{GT1}{Row operations and determinants}{0099} 
\begin{exerciseStatement} 

Let \(A\) be a \(4 \times 4\) matrix with determinant \(-2\).

 

\begin{enumerate}[(a)]
\item Let \(C\) be the matrix obtained from \(A\) by applying the row operation \(R_2 + 2 R_3 \to R_2\). What is \(\mathrm{det}\,C\)?
\item Let \(P\) be the matrix obtained from \(A\) by applying the row operation \(-5 R_3 \to R_3\). What is \(\mathrm{det}\,P\)?
\item Let \(N\) be the matrix obtained from \(A\) by applying the row operation \(R_4 \leftrightarrow R_3\). What is \(\mathrm{det}\,N\)?
\end{enumerate}

     \end{exerciseStatement}
 \begin{exerciseAnswer} 

\begin{enumerate}[(a)]
\item \(\mathrm{det}\,C=-2\)
\item \(\mathrm{det}\,P=10\)
\item \(\mathrm{det}\,N=2\)
\end{enumerate}

     \end{exerciseAnswer}
 \end{exercise}


\newpage




\begin{exercise}{GT1}{Row operations and determinants}{0121} 
\begin{exerciseStatement} 

Let \(A\) be a \(4 \times 4\) matrix with determinant \(4\).

 

\begin{enumerate}[(a)]
\item Let \(Q\) be the matrix obtained from \(A\) by applying the row operation \(R_1 \leftrightarrow R_3\). What is \(\mathrm{det}\,Q\)?
\item Let \(M\) be the matrix obtained from \(A\) by applying the row operation \(-4 R_1 \to R_1\). What is \(\mathrm{det}\,M\)?
\item Let \(B\) be the matrix obtained from \(A\) by applying the row operation \(R_2 - 2 R_3 \to R_2\). What is \(\mathrm{det}\,B\)?
\end{enumerate}

     \end{exerciseStatement}
 \begin{exerciseAnswer} 

\begin{enumerate}[(a)]
\item \(\mathrm{det}\,Q=-4\)
\item \(\mathrm{det}\,M=-16\)
\item \(\mathrm{det}\,B=4\)
\end{enumerate}

     \end{exerciseAnswer}
 \end{exercise}



\begin{exercise}{GT1}{Row operations and determinants}{0166} 
\begin{exerciseStatement} 

Let \(A\) be a \(4 \times 4\) matrix with determinant \(2\).

 

\begin{enumerate}[(a)]
\item Let \(Q\) be the matrix obtained from \(A\) by applying the row operation \(R_1 \leftrightarrow R_2\). What is \(\mathrm{det}\,Q\)?
\item Let \(M\) be the matrix obtained from \(A\) by applying the row operation \(-5 R_4 \to R_4\). What is \(\mathrm{det}\,M\)?
\item Let \(C\) be the matrix obtained from \(A\) by applying the row operation \(R_3 - 3 R_4 \to R_3\). What is \(\mathrm{det}\,C\)?
\end{enumerate}

     \end{exerciseStatement}
 \begin{exerciseAnswer} 

\begin{enumerate}[(a)]
\item \(\mathrm{det}\,Q=-2\)
\item \(\mathrm{det}\,M=-10\)
\item \(\mathrm{det}\,C=2\)
\end{enumerate}

     \end{exerciseAnswer}
 \end{exercise}


\newpage




\begin{exercise}{GT1}{Row operations and determinants}{0069} 
\begin{exerciseStatement} 

Let \(A\) be a \(4 \times 4\) matrix with determinant \(5\).

 

\begin{enumerate}[(a)]
\item Let \(C\) be the matrix obtained from \(A\) by applying the row operation \(R_2 \leftrightarrow R_1\). What is \(\mathrm{det}\,C\)?
\item Let \(N\) be the matrix obtained from \(A\) by applying the row operation \(R_1 + 2 R_4 \to R_1\). What is \(\mathrm{det}\,N\)?
\item Let \(Q\) be the matrix obtained from \(A\) by applying the row operation \(-3 R_2 \to R_2\). What is \(\mathrm{det}\,Q\)?
\end{enumerate}

     \end{exerciseStatement}
 \begin{exerciseAnswer} 

\begin{enumerate}[(a)]
\item \(\mathrm{det}\,C=-5\)
\item \(\mathrm{det}\,N=5\)
\item \(\mathrm{det}\,Q=-15\)
\end{enumerate}

     \end{exerciseAnswer}
 \end{exercise}



\begin{exercise}{GT2}{Determinants}{0104} 
\begin{exerciseStatement} 

Show how to compute the determinant of the matrix \[A=\left[\begin{array}{cccc}
-3 & 1 & 5 & 4 \\
-2 & 5 & 5 & -1 \\
5 & 0 & 2 & -3 \\
2 & 1 & 0 & 0
\end{array}\right].\]

 \end{exerciseStatement}
 \begin{exerciseAnswer} \[\mathrm{det}\,\left[\begin{array}{cccc}
-3 & 1 & 5 & 4 \\
-2 & 5 & 5 & -1 \\
5 & 0 & 2 & -3 \\
2 & 1 & 0 & 0
\end{array}\right]=-336\] \end{exerciseAnswer}
 \end{exercise}


\newpage




\begin{exercise}{GT2}{Determinants}{0121} 
\begin{exerciseStatement} 

Show how to compute the determinant of the matrix \[A=\left[\begin{array}{cccc}
3 & 0 & -1 & 2 \\
5 & 0 & 0 & 2 \\
3 & 3 & -1 & 5 \\
3 & 1 & 5 & 4
\end{array}\right].\]

 \end{exerciseStatement}
 \begin{exerciseAnswer} \[\mathrm{det}\,\left[\begin{array}{cccc}
3 & 0 & -1 & 2 \\
5 & 0 & 0 & 2 \\
3 & 3 & -1 & 5 \\
3 & 1 & 5 & 4
\end{array}\right]=-87\] \end{exerciseAnswer}
 \end{exercise}



\begin{exercise}{GT2}{Determinants}{0217} 
\begin{exerciseStatement} 

Show how to compute the determinant of the matrix \[A=\left[\begin{array}{cccc}
-5 & -4 & -3 & -1 \\
1 & 0 & 1 & 4 \\
3 & 0 & -1 & 0 \\
-5 & 3 & -1 & 1
\end{array}\right].\]

 \end{exerciseStatement}
 \begin{exerciseAnswer} \[\mathrm{det}\,\left[\begin{array}{cccc}
-5 & -4 & -3 & -1 \\
1 & 0 & 1 & 4 \\
3 & 0 & -1 & 0 \\
-5 & 3 & -1 & 1
\end{array}\right]=-300\] \end{exerciseAnswer}
 \end{exercise}


\newpage




\begin{exercise}{GT2}{Determinants}{0295} 
\begin{exerciseStatement} 

Show how to compute the determinant of the matrix \[A=\left[\begin{array}{cccc}
1 & -2 & -3 & 4 \\
-3 & 5 & 4 & -2 \\
2 & 1 & 0 & 0 \\
1 & 5 & 2 & -2
\end{array}\right].\]

 \end{exerciseStatement}
 \begin{exerciseAnswer} \[\mathrm{det}\,\left[\begin{array}{cccc}
1 & -2 & -3 & 4 \\
-3 & 5 & 4 & -2 \\
2 & 1 & 0 & 0 \\
1 & 5 & 2 & -2
\end{array}\right]=-44\] \end{exerciseAnswer}
 \end{exercise}



\begin{exercise}{GT2}{Determinants}{0034} 
\begin{exerciseStatement} 

Show how to compute the determinant of the matrix \[A=\left[\begin{array}{cccc}
5 & 2 & 3 & 0 \\
-3 & -4 & 1 & -1 \\
0 & -1 & 0 & -3 \\
3 & -4 & -1 & -4
\end{array}\right].\]

 \end{exerciseStatement}
 \begin{exerciseAnswer} \[\mathrm{det}\,\left[\begin{array}{cccc}
5 & 2 & 3 & 0 \\
-3 & -4 & 1 & -1 \\
0 & -1 & 0 & -3 \\
3 & -4 & -1 & -4
\end{array}\right]=266\] \end{exerciseAnswer}
 \end{exercise}


\newpage




\begin{exercise}{GT2}{Determinants}{0186} 
\begin{exerciseStatement} 

Show how to compute the determinant of the matrix \[A=\left[\begin{array}{cccc}
-5 & -4 & 4 & 4 \\
3 & 0 & -1 & 0 \\
-2 & 2 & -4 & 4 \\
-2 & -2 & 0 & 1
\end{array}\right].\]

 \end{exerciseStatement}
 \begin{exerciseAnswer} \[\mathrm{det}\,\left[\begin{array}{cccc}
-5 & -4 & 4 & 4 \\
3 & 0 & -1 & 0 \\
-2 & 2 & -4 & 4 \\
-2 & -2 & 0 & 1
\end{array}\right]=174\] \end{exerciseAnswer}
 \end{exercise}



\begin{exercise}{GT2}{Determinants}{0165} 
\begin{exerciseStatement} 

Show how to compute the determinant of the matrix \[A=\left[\begin{array}{cccc}
2 & 5 & 0 & 0 \\
4 & 4 & 4 & -2 \\
3 & 3 & -4 & 0 \\
-3 & -4 & -4 & -1
\end{array}\right].\]

 \end{exerciseStatement}
 \begin{exerciseAnswer} \[\mathrm{det}\,\left[\begin{array}{cccc}
2 & 5 & 0 & 0 \\
4 & 4 & 4 & -2 \\
3 & 3 & -4 & 0 \\
-3 & -4 & -4 & -1
\end{array}\right]=-212\] \end{exerciseAnswer}
 \end{exercise}


\newpage




\begin{exercise}{GT2}{Determinants}{0233} 
\begin{exerciseStatement} 

Show how to compute the determinant of the matrix \[A=\left[\begin{array}{cccc}
3 & 4 & 0 & -5 \\
-5 & 2 & -2 & -5 \\
-4 & 1 & -1 & 4 \\
-5 & 3 & 0 & 4
\end{array}\right].\]

 \end{exerciseStatement}
 \begin{exerciseAnswer} \[\mathrm{det}\,\left[\begin{array}{cccc}
3 & 4 & 0 & -5 \\
-5 & 2 & -2 & -5 \\
-4 & 1 & -1 & 4 \\
-5 & 3 & 0 & 4
\end{array}\right]=-284\] \end{exerciseAnswer}
 \end{exercise}



\begin{exercise}{GT2}{Determinants}{0008} 
\begin{exerciseStatement} 

Show how to compute the determinant of the matrix \[A=\left[\begin{array}{cccc}
2 & 3 & 5 & -4 \\
-2 & 1 & 3 & -3 \\
-5 & 2 & -1 & -2 \\
0 & 1 & 0 & 0
\end{array}\right].\]

 \end{exerciseStatement}
 \begin{exerciseAnswer} \[\mathrm{det}\,\left[\begin{array}{cccc}
2 & 3 & 5 & -4 \\
-2 & 1 & 3 & -3 \\
-5 & 2 & -1 & -2 \\
0 & 1 & 0 & 0
\end{array}\right]=-31\] \end{exerciseAnswer}
 \end{exercise}


\newpage




\begin{exercise}{GT2}{Determinants}{0054} 
\begin{exerciseStatement} 

Show how to compute the determinant of the matrix \[A=\left[\begin{array}{cccc}
-5 & 0 & -5 & -4 \\
5 & -2 & -1 & -5 \\
0 & 2 & 0 & -1 \\
-5 & 2 & -3 & 5
\end{array}\right].\]

 \end{exerciseStatement}
 \begin{exerciseAnswer} \[\mathrm{det}\,\left[\begin{array}{cccc}
-5 & 0 & -5 & -4 \\
5 & -2 & -1 & -5 \\
0 & 2 & 0 & -1 \\
-5 & 2 & -3 & 5
\end{array}\right]=-400\] \end{exerciseAnswer}
 \end{exercise}



\begin{exercise}{GT2}{Determinants}{0252} 
\begin{exerciseStatement} 

Show how to compute the determinant of the matrix \[A=\left[\begin{array}{cccc}
3 & 0 & 1 & -2 \\
2 & 4 & -5 & -1 \\
-4 & 4 & -5 & 0 \\
-3 & -3 & 3 & 0
\end{array}\right].\]

 \end{exerciseStatement}
 \begin{exerciseAnswer} \[\mathrm{det}\,\left[\begin{array}{cccc}
3 & 0 & 1 & -2 \\
2 & 4 & -5 & -1 \\
-4 & 4 & -5 & 0 \\
-3 & -3 & 3 & 0
\end{array}\right]=-51\] \end{exerciseAnswer}
 \end{exercise}


\newpage




\begin{exercise}{GT2}{Determinants}{0254} 
\begin{exerciseStatement} 

Show how to compute the determinant of the matrix \[A=\left[\begin{array}{cccc}
3 & 0 & -4 & 0 \\
3 & -1 & -1 & -1 \\
5 & 0 & 3 & -2 \\
3 & -2 & 2 & 4
\end{array}\right].\]

 \end{exerciseStatement}
 \begin{exerciseAnswer} \[\mathrm{det}\,\left[\begin{array}{cccc}
3 & 0 & -4 & 0 \\
3 & -1 & -1 & -1 \\
5 & 0 & 3 & -2 \\
3 & -2 & 2 & 4
\end{array}\right]=-174\] \end{exerciseAnswer}
 \end{exercise}



\begin{exercise}{GT3}{Eigenvalues}{0191} 
\begin{exerciseStatement} 

Explain how to find the eigenvalues of the matrix \(\left[\begin{array}{cc}
-1 & 2 \\
5 & 2
\end{array}\right]\).

 \end{exerciseStatement}
 \begin{exerciseAnswer} 

The characteristic polynomial of \(\left[\begin{array}{cc}
-1 & 2 \\
5 & 2
\end{array}\right]\) is \(\lambda^{2} - \lambda - 12\).

 

The eigenvalues of \(\left[\begin{array}{cc}
-1 & 2 \\
5 & 2
\end{array}\right]\) are \(-3\) and \(4\).

 \end{exerciseAnswer}
 \end{exercise}


\newpage




\begin{exercise}{GT3}{Eigenvalues}{0223} 
\begin{exerciseStatement} 

Explain how to find the eigenvalues of the matrix \(\left[\begin{array}{cc}
8 & 2 \\
-14 & -8
\end{array}\right]\).

 \end{exerciseStatement}
 \begin{exerciseAnswer} 

The characteristic polynomial of \(\left[\begin{array}{cc}
8 & 2 \\
-14 & -8
\end{array}\right]\) is \(\lambda^{2} - 36\).

 

The eigenvalues of \(\left[\begin{array}{cc}
8 & 2 \\
-14 & -8
\end{array}\right]\) are \(6\) and \(-6\).

 \end{exerciseAnswer}
 \end{exercise}



\begin{exercise}{GT3}{Eigenvalues}{0088} 
\begin{exerciseStatement} 

Explain how to find the eigenvalues of the matrix \(\left[\begin{array}{cc}
-1 & 1 \\
-2 & -4
\end{array}\right]\).

 \end{exerciseStatement}
 \begin{exerciseAnswer} 

The characteristic polynomial of \(\left[\begin{array}{cc}
-1 & 1 \\
-2 & -4
\end{array}\right]\) is \(\lambda^{2} + 5 \lambda + 6\).

 

The eigenvalues of \(\left[\begin{array}{cc}
-1 & 1 \\
-2 & -4
\end{array}\right]\) are \(-3\) and \(-2\).

 \end{exerciseAnswer}
 \end{exercise}


\newpage




\begin{exercise}{GT3}{Eigenvalues}{0066} 
\begin{exerciseStatement} 

Explain how to find the eigenvalues of the matrix \(\left[\begin{array}{cc}
5 & 2 \\
-9 & -6
\end{array}\right]\).

 \end{exerciseStatement}
 \begin{exerciseAnswer} 

The characteristic polynomial of \(\left[\begin{array}{cc}
5 & 2 \\
-9 & -6
\end{array}\right]\) is \(\lambda^{2} + \lambda - 12\).

 

The eigenvalues of \(\left[\begin{array}{cc}
5 & 2 \\
-9 & -6
\end{array}\right]\) are \(3\) and \(-4\).

 \end{exerciseAnswer}
 \end{exercise}



\begin{exercise}{GT3}{Eigenvalues}{0088} 
\begin{exerciseStatement} 

Explain how to find the eigenvalues of the matrix \(\left[\begin{array}{cc}
-1 & 1 \\
-2 & -4
\end{array}\right]\).

 \end{exerciseStatement}
 \begin{exerciseAnswer} 

The characteristic polynomial of \(\left[\begin{array}{cc}
-1 & 1 \\
-2 & -4
\end{array}\right]\) is \(\lambda^{2} + 5 \lambda + 6\).

 

The eigenvalues of \(\left[\begin{array}{cc}
-1 & 1 \\
-2 & -4
\end{array}\right]\) are \(-3\) and \(-2\).

 \end{exerciseAnswer}
 \end{exercise}


\newpage




\begin{exercise}{GT3}{Eigenvalues}{0209} 
\begin{exerciseStatement} 

Explain how to find the eigenvalues of the matrix \(\left[\begin{array}{cc}
-3 & 1 \\
6 & 2
\end{array}\right]\).

 \end{exerciseStatement}
 \begin{exerciseAnswer} 

The characteristic polynomial of \(\left[\begin{array}{cc}
-3 & 1 \\
6 & 2
\end{array}\right]\) is \(\lambda^{2} + \lambda - 12\).

 

The eigenvalues of \(\left[\begin{array}{cc}
-3 & 1 \\
6 & 2
\end{array}\right]\) are \(-4\) and \(3\).

 \end{exerciseAnswer}
 \end{exercise}



\begin{exercise}{GT3}{Eigenvalues}{0205} 
\begin{exerciseStatement} 

Explain how to find the eigenvalues of the matrix \(\left[\begin{array}{cc}
-1 & 1 \\
-8 & -7
\end{array}\right]\).

 \end{exerciseStatement}
 \begin{exerciseAnswer} 

The characteristic polynomial of \(\left[\begin{array}{cc}
-1 & 1 \\
-8 & -7
\end{array}\right]\) is \(\lambda^{2} + 8 \lambda + 15\).

 

The eigenvalues of \(\left[\begin{array}{cc}
-1 & 1 \\
-8 & -7
\end{array}\right]\) are \(-3\) and \(-5\).

 \end{exerciseAnswer}
 \end{exercise}


\newpage




\begin{exercise}{GT3}{Eigenvalues}{0291} 
\begin{exerciseStatement} 

Explain how to find the eigenvalues of the matrix \(\left[\begin{array}{cc}
10 & 2 \\
-21 & -3
\end{array}\right]\).

 \end{exerciseStatement}
 \begin{exerciseAnswer} 

The characteristic polynomial of \(\left[\begin{array}{cc}
10 & 2 \\
-21 & -3
\end{array}\right]\) is \(\lambda^{2} - 7 \lambda + 12\).

 

The eigenvalues of \(\left[\begin{array}{cc}
10 & 2 \\
-21 & -3
\end{array}\right]\) are \(4\) and \(3\).

 \end{exerciseAnswer}
 \end{exercise}



\begin{exercise}{GT3}{Eigenvalues}{0054} 
\begin{exerciseStatement} 

Explain how to find the eigenvalues of the matrix \(\left[\begin{array}{cc}
9 & 1 \\
-15 & 1
\end{array}\right]\).

 \end{exerciseStatement}
 \begin{exerciseAnswer} 

The characteristic polynomial of \(\left[\begin{array}{cc}
9 & 1 \\
-15 & 1
\end{array}\right]\) is \(\lambda^{2} - 10 \lambda + 24\).

 

The eigenvalues of \(\left[\begin{array}{cc}
9 & 1 \\
-15 & 1
\end{array}\right]\) are \(6\) and \(4\).

 \end{exerciseAnswer}
 \end{exercise}


\newpage




\begin{exercise}{GT3}{Eigenvalues}{0069} 
\begin{exerciseStatement} 

Explain how to find the eigenvalues of the matrix \(\left[\begin{array}{cc}
5 & 1 \\
-2 & 2
\end{array}\right]\).

 \end{exerciseStatement}
 \begin{exerciseAnswer} 

The characteristic polynomial of \(\left[\begin{array}{cc}
5 & 1 \\
-2 & 2
\end{array}\right]\) is \(\lambda^{2} - 7 \lambda + 12\).

 

The eigenvalues of \(\left[\begin{array}{cc}
5 & 1 \\
-2 & 2
\end{array}\right]\) are \(3\) and \(4\).

 \end{exerciseAnswer}
 \end{exercise}



\begin{exercise}{GT3}{Eigenvalues}{0232} 
\begin{exerciseStatement} 

Explain how to find the eigenvalues of the matrix \(\left[\begin{array}{cc}
0 & 1 \\
9 & 0
\end{array}\right]\).

 \end{exerciseStatement}
 \begin{exerciseAnswer} 

The characteristic polynomial of \(\left[\begin{array}{cc}
0 & 1 \\
9 & 0
\end{array}\right]\) is \(\lambda^{2} - 9\).

 

The eigenvalues of \(\left[\begin{array}{cc}
0 & 1 \\
9 & 0
\end{array}\right]\) are \(-3\) and \(3\).

 \end{exerciseAnswer}
 \end{exercise}


\newpage




\begin{exercise}{GT3}{Eigenvalues}{0143} 
\begin{exerciseStatement} 

Explain how to find the eigenvalues of the matrix \(\left[\begin{array}{cc}
-1 & 2 \\
6 & 3
\end{array}\right]\).

 \end{exerciseStatement}
 \begin{exerciseAnswer} 

The characteristic polynomial of \(\left[\begin{array}{cc}
-1 & 2 \\
6 & 3
\end{array}\right]\) is \(\lambda^{2} - 2 \lambda - 15\).

 

The eigenvalues of \(\left[\begin{array}{cc}
-1 & 2 \\
6 & 3
\end{array}\right]\) are \(-3\) and \(5\).

 \end{exerciseAnswer}
 \end{exercise}



\begin{exercise}{GT3}{Eigenvalues}{0209} 
\begin{exerciseStatement} 

Explain how to find the eigenvalues of the matrix \(\left[\begin{array}{cc}
-3 & 1 \\
6 & 2
\end{array}\right]\).

 \end{exerciseStatement}
 \begin{exerciseAnswer} 

The characteristic polynomial of \(\left[\begin{array}{cc}
-3 & 1 \\
6 & 2
\end{array}\right]\) is \(\lambda^{2} + \lambda - 12\).

 

The eigenvalues of \(\left[\begin{array}{cc}
-3 & 1 \\
6 & 2
\end{array}\right]\) are \(-4\) and \(3\).

 \end{exerciseAnswer}
 \end{exercise}


\newpage




\begin{exercise}{GT3}{Eigenvalues}{0099} 
\begin{exerciseStatement} 

Explain how to find the eigenvalues of the matrix \(\left[\begin{array}{cc}
-2 & 2 \\
-4 & -8
\end{array}\right]\).

 \end{exerciseStatement}
 \begin{exerciseAnswer} 

The characteristic polynomial of \(\left[\begin{array}{cc}
-2 & 2 \\
-4 & -8
\end{array}\right]\) is \(\lambda^{2} + 10 \lambda + 24\).

 

The eigenvalues of \(\left[\begin{array}{cc}
-2 & 2 \\
-4 & -8
\end{array}\right]\) are \(-6\) and \(-4\).

 \end{exerciseAnswer}
 \end{exercise}



\begin{exercise}{GT3}{Eigenvalues}{0156} 
\begin{exerciseStatement} 

Explain how to find the eigenvalues of the matrix \(\left[\begin{array}{cc}
5 & 2 \\
-9 & -6
\end{array}\right]\).

 \end{exerciseStatement}
 \begin{exerciseAnswer} 

The characteristic polynomial of \(\left[\begin{array}{cc}
5 & 2 \\
-9 & -6
\end{array}\right]\) is \(\lambda^{2} + \lambda - 12\).

 

The eigenvalues of \(\left[\begin{array}{cc}
5 & 2 \\
-9 & -6
\end{array}\right]\) are \(3\) and \(-4\).

 \end{exerciseAnswer}
 \end{exercise}


\newpage




\begin{exercise}{GT3}{Eigenvalues}{0016} 
\begin{exerciseStatement} 

Explain how to find the eigenvalues of the matrix \(\left[\begin{array}{cc}
3 & 1 \\
1 & 3
\end{array}\right]\).

 \end{exerciseStatement}
 \begin{exerciseAnswer} 

The characteristic polynomial of \(\left[\begin{array}{cc}
3 & 1 \\
1 & 3
\end{array}\right]\) is \(\lambda^{2} - 6 \lambda + 8\).

 

The eigenvalues of \(\left[\begin{array}{cc}
3 & 1 \\
1 & 3
\end{array}\right]\) are \(2\) and \(4\).

 \end{exerciseAnswer}
 \end{exercise}



\begin{exercise}{GT3}{Eigenvalues}{0116} 
\begin{exerciseStatement} 

Explain how to find the eigenvalues of the matrix \(\left[\begin{array}{cc}
12 & 2 \\
-51 & -11
\end{array}\right]\).

 \end{exerciseStatement}
 \begin{exerciseAnswer} 

The characteristic polynomial of \(\left[\begin{array}{cc}
12 & 2 \\
-51 & -11
\end{array}\right]\) is \(\lambda^{2} - \lambda - 30\).

 

The eigenvalues of \(\left[\begin{array}{cc}
12 & 2 \\
-51 & -11
\end{array}\right]\) are \(6\) and \(-5\).

 \end{exerciseAnswer}
 \end{exercise}


\newpage




\begin{exercise}{GT3}{Eigenvalues}{0210} 
\begin{exerciseStatement} 

Explain how to find the eigenvalues of the matrix \(\left[\begin{array}{cc}
-1 & 2 \\
6 & -2
\end{array}\right]\).

 \end{exerciseStatement}
 \begin{exerciseAnswer} 

The characteristic polynomial of \(\left[\begin{array}{cc}
-1 & 2 \\
6 & -2
\end{array}\right]\) is \(\lambda^{2} + 3 \lambda - 10\).

 

The eigenvalues of \(\left[\begin{array}{cc}
-1 & 2 \\
6 & -2
\end{array}\right]\) are \(-5\) and \(2\).

 \end{exerciseAnswer}
 \end{exercise}



\begin{exercise}{GT3}{Eigenvalues}{0142} 
\begin{exerciseStatement} 

Explain how to find the eigenvalues of the matrix \(\left[\begin{array}{cc}
3 & 2 \\
-15 & -8
\end{array}\right]\).

 \end{exerciseStatement}
 \begin{exerciseAnswer} 

The characteristic polynomial of \(\left[\begin{array}{cc}
3 & 2 \\
-15 & -8
\end{array}\right]\) is \(\lambda^{2} + 5 \lambda + 6\).

 

The eigenvalues of \(\left[\begin{array}{cc}
3 & 2 \\
-15 & -8
\end{array}\right]\) are \(-3\) and \(-2\).

 \end{exerciseAnswer}
 \end{exercise}


\newpage




\begin{exercise}{GT3}{Eigenvalues}{0050} 
\begin{exerciseStatement} 

Explain how to find the eigenvalues of the matrix \(\left[\begin{array}{cc}
5 & 1 \\
-11 & -7
\end{array}\right]\).

 \end{exerciseStatement}
 \begin{exerciseAnswer} 

The characteristic polynomial of \(\left[\begin{array}{cc}
5 & 1 \\
-11 & -7
\end{array}\right]\) is \(\lambda^{2} + 2 \lambda - 24\).

 

The eigenvalues of \(\left[\begin{array}{cc}
5 & 1 \\
-11 & -7
\end{array}\right]\) are \(4\) and \(-6\).

 \end{exerciseAnswer}
 \end{exercise}



\begin{exercise}{GT4}{Eigenvectors}{0000} 
\begin{exerciseStatement} 

Explain how to find a basis for the eigenspace associated to the eigenvalue \(-1\) in the matrix \[\left[\begin{array}{cccc}
2 & 4 & -1 & -5 \\
0 & 0 & -3 & -9 \\
1 & 1 & 0 & 2 \\
-2 & -2 & 3 & 5
\end{array}\right].\]

 \end{exerciseStatement}
 \begin{exerciseAnswer} 

\[\mathrm{RREF}\,\left[\begin{array}{cccc}
3 & 4 & -1 & -5 \\
0 & 1 & -3 & -9 \\
1 & 1 & 1 & 2 \\
-2 & -2 & 3 & 6
\end{array}\right]=\left[\begin{array}{cccc}
1 & 0 & 0 & 3 \\
0 & 1 & 0 & -3 \\
0 & 0 & 1 & 2 \\
0 & 0 & 0 & 0
\end{array}\right]\]

 

A basis of the eigenspace is \(\left\{ \left[\begin{array}{c}
-3 \\
3 \\
-2 \\
1
\end{array}\right] \right\}\).

 \end{exerciseAnswer}
 \end{exercise}


\newpage




\begin{exercise}{GT4}{Eigenvectors}{0204} 
\begin{exerciseStatement} 

Explain how to find a basis for the eigenspace associated to the eigenvalue \(2\) in the matrix \[\left[\begin{array}{cccc}
0 & -4 & 4 & 2 \\
-2 & -2 & 4 & 2 \\
0 & 0 & 2 & 0 \\
-3 & -6 & 6 & 5
\end{array}\right].\]

 \end{exerciseStatement}
 \begin{exerciseAnswer} 

\[\mathrm{RREF}\,\left[\begin{array}{cccc}
-2 & -4 & 4 & 2 \\
-2 & -4 & 4 & 2 \\
0 & 0 & 0 & 0 \\
-3 & -6 & 6 & 3
\end{array}\right]=\left[\begin{array}{cccc}
1 & 2 & -2 & -1 \\
0 & 0 & 0 & 0 \\
0 & 0 & 0 & 0 \\
0 & 0 & 0 & 0
\end{array}\right]\]

 

A basis of the eigenspace is \(\left\{ \left[\begin{array}{c}
-2 \\
1 \\
0 \\
0
\end{array}\right] , \left[\begin{array}{c}
2 \\
0 \\
1 \\
0
\end{array}\right] , \left[\begin{array}{c}
1 \\
0 \\
0 \\
1
\end{array}\right] \right\}\).

 \end{exerciseAnswer}
 \end{exercise}



\begin{exercise}{GT4}{Eigenvectors}{0117} 
\begin{exerciseStatement} 

Explain how to find a basis for the eigenspace associated to the eigenvalue \(-4\) in the matrix \[\left[\begin{array}{cccc}
-3 & -1 & -3 & 3 \\
2 & -5 & -2 & -2 \\
2 & -2 & -9 & 4 \\
0 & 0 & 0 & -4
\end{array}\right].\]

 \end{exerciseStatement}
 \begin{exerciseAnswer} 

\[\mathrm{RREF}\,\left[\begin{array}{cccc}
1 & -1 & -3 & 3 \\
2 & -1 & -2 & -2 \\
2 & -2 & -5 & 4 \\
0 & 0 & 0 & 0
\end{array}\right]=\left[\begin{array}{cccc}
1 & 0 & 0 & -3 \\
0 & 1 & 0 & 0 \\
0 & 0 & 1 & -2 \\
0 & 0 & 0 & 0
\end{array}\right]\]

 

A basis of the eigenspace is \(\left\{ \left[\begin{array}{c}
3 \\
0 \\
2 \\
1
\end{array}\right] \right\}\).

 \end{exerciseAnswer}
 \end{exercise}


\newpage




\begin{exercise}{GT4}{Eigenvectors}{0014} 
\begin{exerciseStatement} 

Explain how to find a basis for the eigenspace associated to the eigenvalue \(-2\) in the matrix \[\left[\begin{array}{cccc}
-1 & 4 & 10 & 1 \\
3 & -7 & -21 & 3 \\
0 & -3 & -11 & 0 \\
0 & 2 & 6 & -2
\end{array}\right].\]

 \end{exerciseStatement}
 \begin{exerciseAnswer} 

\[\mathrm{RREF}\,\left[\begin{array}{cccc}
1 & 4 & 10 & 1 \\
3 & -5 & -21 & 3 \\
0 & -3 & -9 & 0 \\
0 & 2 & 6 & 0
\end{array}\right]=\left[\begin{array}{cccc}
1 & 0 & -2 & 1 \\
0 & 1 & 3 & 0 \\
0 & 0 & 0 & 0 \\
0 & 0 & 0 & 0
\end{array}\right]\]

 

A basis of the eigenspace is \(\left\{ \left[\begin{array}{c}
2 \\
-3 \\
1 \\
0
\end{array}\right] , \left[\begin{array}{c}
-1 \\
0 \\
0 \\
1
\end{array}\right] \right\}\).

 \end{exerciseAnswer}
 \end{exercise}



\begin{exercise}{GT4}{Eigenvectors}{0228} 
\begin{exerciseStatement} 

Explain how to find a basis for the eigenspace associated to the eigenvalue \(-2\) in the matrix \[\left[\begin{array}{cccc}
2 & -4 & 4 & 12 \\
-1 & -1 & -1 & -3 \\
-4 & 4 & -6 & -12 \\
-4 & 4 & -4 & -14
\end{array}\right].\]

 \end{exerciseStatement}
 \begin{exerciseAnswer} 

\[\mathrm{RREF}\,\left[\begin{array}{cccc}
4 & -4 & 4 & 12 \\
-1 & 1 & -1 & -3 \\
-4 & 4 & -4 & -12 \\
-4 & 4 & -4 & -12
\end{array}\right]=\left[\begin{array}{cccc}
1 & -1 & 1 & 3 \\
0 & 0 & 0 & 0 \\
0 & 0 & 0 & 0 \\
0 & 0 & 0 & 0
\end{array}\right]\]

 

A basis of the eigenspace is \(\left\{ \left[\begin{array}{c}
1 \\
1 \\
0 \\
0
\end{array}\right] , \left[\begin{array}{c}
-1 \\
0 \\
1 \\
0
\end{array}\right] , \left[\begin{array}{c}
-3 \\
0 \\
0 \\
1
\end{array}\right] \right\}\).

 \end{exerciseAnswer}
 \end{exercise}


\newpage




\begin{exercise}{GT4}{Eigenvectors}{0001} 
\begin{exerciseStatement} 

Explain how to find a basis for the eigenspace associated to the eigenvalue \(2\) in the matrix \[\left[\begin{array}{cccc}
2 & 0 & -4 & -12 \\
1 & 1 & -4 & -13 \\
0 & 0 & 3 & 3 \\
1 & -1 & -1 & -2
\end{array}\right].\]

 \end{exerciseStatement}
 \begin{exerciseAnswer} 

\[\mathrm{RREF}\,\left[\begin{array}{cccc}
0 & 0 & -4 & -12 \\
1 & -1 & -4 & -13 \\
0 & 0 & 1 & 3 \\
1 & -1 & -1 & -4
\end{array}\right]=\left[\begin{array}{cccc}
1 & -1 & 0 & -1 \\
0 & 0 & 1 & 3 \\
0 & 0 & 0 & 0 \\
0 & 0 & 0 & 0
\end{array}\right]\]

 

A basis of the eigenspace is \(\left\{ \left[\begin{array}{c}
1 \\
1 \\
0 \\
0
\end{array}\right] , \left[\begin{array}{c}
1 \\
0 \\
-3 \\
1
\end{array}\right] \right\}\).

 \end{exerciseAnswer}
 \end{exercise}



\begin{exercise}{GT4}{Eigenvectors}{0041} 
\begin{exerciseStatement} 

Explain how to find a basis for the eigenspace associated to the eigenvalue \(2\) in the matrix \[\left[\begin{array}{cccc}
3 & 1 & 0 & 1 \\
-1 & 2 & 2 & -2 \\
3 & 0 & -3 & 5 \\
-5 & -2 & 2 & -2
\end{array}\right].\]

 \end{exerciseStatement}
 \begin{exerciseAnswer} 

\[\mathrm{RREF}\,\left[\begin{array}{cccc}
1 & 1 & 0 & 1 \\
-1 & 0 & 2 & -2 \\
3 & 0 & -5 & 5 \\
-5 & -2 & 2 & -4
\end{array}\right]=\left[\begin{array}{cccc}
1 & 0 & 0 & 0 \\
0 & 1 & 0 & 1 \\
0 & 0 & 1 & -1 \\
0 & 0 & 0 & 0
\end{array}\right]\]

 

A basis of the eigenspace is \(\left\{ \left[\begin{array}{c}
0 \\
-1 \\
1 \\
1
\end{array}\right] \right\}\).

 \end{exerciseAnswer}
 \end{exercise}


\newpage




\begin{exercise}{GT4}{Eigenvectors}{0050} 
\begin{exerciseStatement} 

Explain how to find a basis for the eigenspace associated to the eigenvalue \(-2\) in the matrix \[\left[\begin{array}{cccc}
-1 & -4 & -11 & 1 \\
-1 & -1 & 5 & -5 \\
-1 & -4 & -7 & -4 \\
0 & 4 & 8 & 2
\end{array}\right].\]

 \end{exerciseStatement}
 \begin{exerciseAnswer} 

\[\mathrm{RREF}\,\left[\begin{array}{cccc}
1 & -4 & -11 & 1 \\
-1 & 1 & 5 & -5 \\
-1 & -4 & -5 & -4 \\
0 & 4 & 8 & 4
\end{array}\right]=\left[\begin{array}{cccc}
1 & 0 & -3 & 0 \\
0 & 1 & 2 & 0 \\
0 & 0 & 0 & 1 \\
0 & 0 & 0 & 0
\end{array}\right]\]

 

A basis of the eigenspace is \(\left\{ \left[\begin{array}{c}
3 \\
-2 \\
1 \\
0
\end{array}\right] \right\}\).

 \end{exerciseAnswer}
 \end{exercise}



\begin{exercise}{GT4}{Eigenvectors}{0061} 
\begin{exerciseStatement} 

Explain how to find a basis for the eigenspace associated to the eigenvalue \(-3\) in the matrix \[\left[\begin{array}{cccc}
-3 & 0 & -1 & -2 \\
1 & -6 & -3 & -18 \\
1 & -4 & -7 & -23 \\
0 & -1 & -1 & -8
\end{array}\right].\]

 \end{exerciseStatement}
 \begin{exerciseAnswer} 

\[\mathrm{RREF}\,\left[\begin{array}{cccc}
0 & 0 & -1 & -2 \\
1 & -3 & -3 & -18 \\
1 & -4 & -4 & -23 \\
0 & -1 & -1 & -5
\end{array}\right]=\left[\begin{array}{cccc}
1 & 0 & 0 & -3 \\
0 & 1 & 0 & 3 \\
0 & 0 & 1 & 2 \\
0 & 0 & 0 & 0
\end{array}\right]\]

 

A basis of the eigenspace is \(\left\{ \left[\begin{array}{c}
3 \\
-3 \\
-2 \\
1
\end{array}\right] \right\}\).

 \end{exerciseAnswer}
 \end{exercise}


\newpage




\begin{exercise}{GT4}{Eigenvectors}{0185} 
\begin{exerciseStatement} 

Explain how to find a basis for the eigenspace associated to the eigenvalue \(-2\) in the matrix \[\left[\begin{array}{cccc}
-2 & -1 & 2 & 1 \\
1 & -3 & 4 & -1 \\
3 & -2 & 8 & -4 \\
-2 & 5 & -14 & -3
\end{array}\right].\]

 \end{exerciseStatement}
 \begin{exerciseAnswer} 

\[\mathrm{RREF}\,\left[\begin{array}{cccc}
0 & -1 & 2 & 1 \\
1 & -1 & 4 & -1 \\
3 & -2 & 10 & -4 \\
-2 & 5 & -14 & -1
\end{array}\right]=\left[\begin{array}{cccc}
1 & 0 & 2 & -2 \\
0 & 1 & -2 & -1 \\
0 & 0 & 0 & 0 \\
0 & 0 & 0 & 0
\end{array}\right]\]

 

A basis of the eigenspace is \(\left\{ \left[\begin{array}{c}
-2 \\
2 \\
1 \\
0
\end{array}\right] , \left[\begin{array}{c}
2 \\
1 \\
0 \\
1
\end{array}\right] \right\}\).

 \end{exerciseAnswer}
 \end{exercise}



\begin{exercise}{GT4}{Eigenvectors}{0250} 
\begin{exerciseStatement} 

Explain how to find a basis for the eigenspace associated to the eigenvalue \(3\) in the matrix \[\left[\begin{array}{cccc}
-1 & 8 & -12 & -4 \\
2 & -1 & 6 & 2 \\
-5 & 10 & -12 & -5 \\
-3 & 6 & -9 & 0
\end{array}\right].\]

 \end{exerciseStatement}
 \begin{exerciseAnswer} 

\[\mathrm{RREF}\,\left[\begin{array}{cccc}
-4 & 8 & -12 & -4 \\
2 & -4 & 6 & 2 \\
-5 & 10 & -15 & -5 \\
-3 & 6 & -9 & -3
\end{array}\right]=\left[\begin{array}{cccc}
1 & -2 & 3 & 1 \\
0 & 0 & 0 & 0 \\
0 & 0 & 0 & 0 \\
0 & 0 & 0 & 0
\end{array}\right]\]

 

A basis of the eigenspace is \(\left\{ \left[\begin{array}{c}
2 \\
1 \\
0 \\
0
\end{array}\right] , \left[\begin{array}{c}
-3 \\
0 \\
1 \\
0
\end{array}\right] , \left[\begin{array}{c}
-1 \\
0 \\
0 \\
1
\end{array}\right] \right\}\).

 \end{exerciseAnswer}
 \end{exercise}


\newpage




\begin{exercise}{GT4}{Eigenvectors}{0235} 
\begin{exerciseStatement} 

Explain how to find a basis for the eigenspace associated to the eigenvalue \(-3\) in the matrix \[\left[\begin{array}{cccc}
-4 & -2 & 2 & -3 \\
-4 & -11 & 8 & -12 \\
-1 & -2 & -1 & -3 \\
-3 & -6 & 6 & -12
\end{array}\right].\]

 \end{exerciseStatement}
 \begin{exerciseAnswer} 

\[\mathrm{RREF}\,\left[\begin{array}{cccc}
-1 & -2 & 2 & -3 \\
-4 & -8 & 8 & -12 \\
-1 & -2 & 2 & -3 \\
-3 & -6 & 6 & -9
\end{array}\right]=\left[\begin{array}{cccc}
1 & 2 & -2 & 3 \\
0 & 0 & 0 & 0 \\
0 & 0 & 0 & 0 \\
0 & 0 & 0 & 0
\end{array}\right]\]

 

A basis of the eigenspace is \(\left\{ \left[\begin{array}{c}
-2 \\
1 \\
0 \\
0
\end{array}\right] , \left[\begin{array}{c}
2 \\
0 \\
1 \\
0
\end{array}\right] , \left[\begin{array}{c}
-3 \\
0 \\
0 \\
1
\end{array}\right] \right\}\).

 \end{exerciseAnswer}
 \end{exercise}



\begin{exercise}{GT4}{Eigenvectors}{0201} 
\begin{exerciseStatement} 

Explain how to find a basis for the eigenspace associated to the eigenvalue \(-2\) in the matrix \[\left[\begin{array}{cccc}
-1 & 1 & -2 & -1 \\
5 & 3 & -10 & -5 \\
-4 & -4 & 6 & 4 \\
5 & 5 & -10 & -7
\end{array}\right].\]

 \end{exerciseStatement}
 \begin{exerciseAnswer} 

\[\mathrm{RREF}\,\left[\begin{array}{cccc}
1 & 1 & -2 & -1 \\
5 & 5 & -10 & -5 \\
-4 & -4 & 8 & 4 \\
5 & 5 & -10 & -5
\end{array}\right]=\left[\begin{array}{cccc}
1 & 1 & -2 & -1 \\
0 & 0 & 0 & 0 \\
0 & 0 & 0 & 0 \\
0 & 0 & 0 & 0
\end{array}\right]\]

 

A basis of the eigenspace is \(\left\{ \left[\begin{array}{c}
-1 \\
1 \\
0 \\
0
\end{array}\right] , \left[\begin{array}{c}
2 \\
0 \\
1 \\
0
\end{array}\right] , \left[\begin{array}{c}
1 \\
0 \\
0 \\
1
\end{array}\right] \right\}\).

 \end{exerciseAnswer}
 \end{exercise}


\newpage




\begin{exercise}{GT4}{Eigenvectors}{0165} 
\begin{exerciseStatement} 

Explain how to find a basis for the eigenspace associated to the eigenvalue \(2\) in the matrix \[\left[\begin{array}{cccc}
3 & -2 & 2 & -1 \\
5 & -8 & 10 & -5 \\
4 & -8 & 10 & -4 \\
-2 & 4 & -4 & 4
\end{array}\right].\]

 \end{exerciseStatement}
 \begin{exerciseAnswer} 

\[\mathrm{RREF}\,\left[\begin{array}{cccc}
1 & -2 & 2 & -1 \\
5 & -10 & 10 & -5 \\
4 & -8 & 8 & -4 \\
-2 & 4 & -4 & 2
\end{array}\right]=\left[\begin{array}{cccc}
1 & -2 & 2 & -1 \\
0 & 0 & 0 & 0 \\
0 & 0 & 0 & 0 \\
0 & 0 & 0 & 0
\end{array}\right]\]

 

A basis of the eigenspace is \(\left\{ \left[\begin{array}{c}
2 \\
1 \\
0 \\
0
\end{array}\right] , \left[\begin{array}{c}
-2 \\
0 \\
1 \\
0
\end{array}\right] , \left[\begin{array}{c}
1 \\
0 \\
0 \\
1
\end{array}\right] \right\}\).

 \end{exerciseAnswer}
 \end{exercise}



\begin{exercise}{GT4}{Eigenvectors}{0194} 
\begin{exerciseStatement} 

Explain how to find a basis for the eigenspace associated to the eigenvalue \(2\) in the matrix \[\left[\begin{array}{cccc}
3 & -1 & 2 & 1 \\
0 & 2 & 0 & 0 \\
4 & -4 & 10 & 4 \\
2 & -2 & 4 & 4
\end{array}\right].\]

 \end{exerciseStatement}
 \begin{exerciseAnswer} 

\[\mathrm{RREF}\,\left[\begin{array}{cccc}
1 & -1 & 2 & 1 \\
0 & 0 & 0 & 0 \\
4 & -4 & 8 & 4 \\
2 & -2 & 4 & 2
\end{array}\right]=\left[\begin{array}{cccc}
1 & -1 & 2 & 1 \\
0 & 0 & 0 & 0 \\
0 & 0 & 0 & 0 \\
0 & 0 & 0 & 0
\end{array}\right]\]

 

A basis of the eigenspace is \(\left\{ \left[\begin{array}{c}
1 \\
1 \\
0 \\
0
\end{array}\right] , \left[\begin{array}{c}
-2 \\
0 \\
1 \\
0
\end{array}\right] , \left[\begin{array}{c}
-1 \\
0 \\
0 \\
1
\end{array}\right] \right\}\).

 \end{exerciseAnswer}
 \end{exercise}


\newpage




\begin{exercise}{GT4}{Eigenvectors}{0243} 
\begin{exerciseStatement} 

Explain how to find a basis for the eigenspace associated to the eigenvalue \(-4\) in the matrix \[\left[\begin{array}{cccc}
-1 & 2 & -5 & 3 \\
1 & -3 & -4 & 2 \\
1 & 1 & -7 & 2 \\
2 & 0 & -3 & -6
\end{array}\right].\]

 \end{exerciseStatement}
 \begin{exerciseAnswer} 

\[\mathrm{RREF}\,\left[\begin{array}{cccc}
3 & 2 & -5 & 3 \\
1 & 1 & -4 & 2 \\
1 & 1 & -3 & 2 \\
2 & 0 & -3 & -2
\end{array}\right]=\left[\begin{array}{cccc}
1 & 0 & 0 & -1 \\
0 & 1 & 0 & 3 \\
0 & 0 & 1 & 0 \\
0 & 0 & 0 & 0
\end{array}\right]\]

 

A basis of the eigenspace is \(\left\{ \left[\begin{array}{c}
1 \\
-3 \\
0 \\
1
\end{array}\right] \right\}\).

 \end{exerciseAnswer}
 \end{exercise}



\begin{exercise}{GT4}{Eigenvectors}{0121} 
\begin{exerciseStatement} 

Explain how to find a basis for the eigenspace associated to the eigenvalue \(3\) in the matrix \[\left[\begin{array}{cccc}
4 & 0 & 3 & -2 \\
1 & 3 & 3 & -3 \\
0 & 1 & 5 & -4 \\
5 & -3 & 9 & 8
\end{array}\right].\]

 \end{exerciseStatement}
 \begin{exerciseAnswer} 

\[\mathrm{RREF}\,\left[\begin{array}{cccc}
1 & 0 & 3 & -2 \\
1 & 0 & 3 & -3 \\
0 & 1 & 2 & -4 \\
5 & -3 & 9 & 5
\end{array}\right]=\left[\begin{array}{cccc}
1 & 0 & 3 & 0 \\
0 & 1 & 2 & 0 \\
0 & 0 & 0 & 1 \\
0 & 0 & 0 & 0
\end{array}\right]\]

 

A basis of the eigenspace is \(\left\{ \left[\begin{array}{c}
-3 \\
-2 \\
1 \\
0
\end{array}\right] \right\}\).

 \end{exerciseAnswer}
 \end{exercise}


\newpage




\begin{exercise}{GT4}{Eigenvectors}{0219} 
\begin{exerciseStatement} 

Explain how to find a basis for the eigenspace associated to the eigenvalue \(3\) in the matrix \[\left[\begin{array}{cccc}
6 & -9 & 9 & -3 \\
-4 & 15 & -12 & 4 \\
2 & -6 & 9 & -2 \\
0 & 0 & 0 & 3
\end{array}\right].\]

 \end{exerciseStatement}
 \begin{exerciseAnswer} 

\[\mathrm{RREF}\,\left[\begin{array}{cccc}
3 & -9 & 9 & -3 \\
-4 & 12 & -12 & 4 \\
2 & -6 & 6 & -2 \\
0 & 0 & 0 & 0
\end{array}\right]=\left[\begin{array}{cccc}
1 & -3 & 3 & -1 \\
0 & 0 & 0 & 0 \\
0 & 0 & 0 & 0 \\
0 & 0 & 0 & 0
\end{array}\right]\]

 

A basis of the eigenspace is \(\left\{ \left[\begin{array}{c}
3 \\
1 \\
0 \\
0
\end{array}\right] , \left[\begin{array}{c}
-3 \\
0 \\
1 \\
0
\end{array}\right] , \left[\begin{array}{c}
1 \\
0 \\
0 \\
1
\end{array}\right] \right\}\).

 \end{exerciseAnswer}
 \end{exercise}



\begin{exercise}{GT4}{Eigenvectors}{0111} 
\begin{exerciseStatement} 

Explain how to find a basis for the eigenspace associated to the eigenvalue \(1\) in the matrix \[\left[\begin{array}{cccc}
2 & 3 & 1 & 2 \\
-2 & -5 & -2 & -4 \\
3 & 9 & 4 & 6 \\
0 & 0 & 0 & 1
\end{array}\right].\]

 \end{exerciseStatement}
 \begin{exerciseAnswer} 

\[\mathrm{RREF}\,\left[\begin{array}{cccc}
1 & 3 & 1 & 2 \\
-2 & -6 & -2 & -4 \\
3 & 9 & 3 & 6 \\
0 & 0 & 0 & 0
\end{array}\right]=\left[\begin{array}{cccc}
1 & 3 & 1 & 2 \\
0 & 0 & 0 & 0 \\
0 & 0 & 0 & 0 \\
0 & 0 & 0 & 0
\end{array}\right]\]

 

A basis of the eigenspace is \(\left\{ \left[\begin{array}{c}
-3 \\
1 \\
0 \\
0
\end{array}\right] , \left[\begin{array}{c}
-1 \\
0 \\
1 \\
0
\end{array}\right] , \left[\begin{array}{c}
-2 \\
0 \\
0 \\
1
\end{array}\right] \right\}\).

 \end{exerciseAnswer}
 \end{exercise}


\newpage




\begin{exercise}{GT4}{Eigenvectors}{0233} 
\begin{exerciseStatement} 

Explain how to find a basis for the eigenspace associated to the eigenvalue \(3\) in the matrix \[\left[\begin{array}{cccc}
5 & -2 & -5 & -4 \\
-1 & 4 & 5 & 7 \\
-2 & 2 & 7 & 2 \\
-1 & 1 & 1 & 2
\end{array}\right].\]

 \end{exerciseStatement}
 \begin{exerciseAnswer} 

\[\mathrm{RREF}\,\left[\begin{array}{cccc}
2 & -2 & -5 & -4 \\
-1 & 1 & 5 & 7 \\
-2 & 2 & 4 & 2 \\
-1 & 1 & 1 & -1
\end{array}\right]=\left[\begin{array}{cccc}
1 & -1 & 0 & 3 \\
0 & 0 & 1 & 2 \\
0 & 0 & 0 & 0 \\
0 & 0 & 0 & 0
\end{array}\right]\]

 

A basis of the eigenspace is \(\left\{ \left[\begin{array}{c}
1 \\
1 \\
0 \\
0
\end{array}\right] , \left[\begin{array}{c}
-3 \\
0 \\
-2 \\
1
\end{array}\right] \right\}\).

 \end{exerciseAnswer}
 \end{exercise}


