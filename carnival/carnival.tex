\documentclass{article}
\usepackage[left=1in,right=1in,bottom=1in,top=0.2in]{geometry}
\usepackage{newenviron, graphicx, amsmath, amssymb, enumerate}
\usepackage{graphicx}


\newcommand{\ferriswheel}{\includegraphics[scale=0.05]{ferris_wheel.png} }

\newenviron{exercise}[3]{
\ferriswheel \hfill  {\Huge \textbf{#1: #2}} \hfill \ferriswheel
\ \\
\vspace{0.2in}

\exercisebody
\vfill \begin{flushright}Version {#3} \end{flushright} \newpage
}{}
\newenvironment{exerciseStatement}{}{}
\newenviron{exerciseAnswer}{}{}

\pagestyle{empty}

\begin{document}



\begin{exercise}{VS1}{Vector spaces}{1164} 
\begin{exerciseStatement} 

 Let \(V\) be the set of all pairs \((x,y)\) of real numbers together with the following operations: 

 \[(x_1,y_1)\oplus (x_2,y_2)=\left(x_{1} + x_{2} + 5,\,\sqrt{y_{1}^{2} + y_{2}^{2}}\right)\]\[c \odot (x,y) =\left(c x,\,c y\right).\] 

 (a) Show that vector addition is associative, that is: 

 \[
      \left((x_1,y_1)\oplus(x_2,y_2)\right)\oplus(x_3,y_3)=(x_1,y_1)\oplus\left((x_2,y_2)\oplus(x_3,y_3)\right).
    \] 

 (b) Explain why \(V\) nonetheless is not a vector space. 

 \end{exerciseStatement}
 \begin{exerciseAnswer} 

 \(V\) is not a vector space, which may be shown by demonstrating that any one of the following properties do not hold: 

 

\begin{itemize}
\item there is no additive identity element
\item scalar multiplication does not distribute over vector addition
\item scalar multiplication does not distribute over scalar addition
\end{itemize}

     \end{exerciseAnswer}
 \end{exercise}
 
 
\begin{exercise}{VS2}{Linear combinations}{4846} 

\begin{exerciseStatement} 

\begin{enumerate}[(a)]
\item  

 Write a statement involving the solutions of a vector equation that's equivalent to each claim below. 

 

\begin{itemize}
\item   \(\left[\begin{array}{c}
-6 \\
1 \\
-4 \\
-4
\end{array}\right]\)is a linear combination of the vectors \(\left[\begin{array}{c}
-2 \\
0 \\
-1 \\
-1
\end{array}\right] , \left[\begin{array}{c}
2 \\
0 \\
1 \\
1
\end{array}\right] , \text{ and } \left[\begin{array}{c}
-2 \\
1 \\
-2 \\
-2
\end{array}\right]\). 

 
\item   \(\left[\begin{array}{c}
-7 \\
0 \\
-5 \\
-3
\end{array}\right]\)is a linear combination of the vectors \(\left[\begin{array}{c}
-2 \\
0 \\
-1 \\
-1
\end{array}\right] , \left[\begin{array}{c}
2 \\
0 \\
1 \\
1
\end{array}\right] , \text{ and } \left[\begin{array}{c}
-2 \\
1 \\
-2 \\
-2
\end{array}\right]\). 

 
\end{itemize}

     
\item   Use these statements to determine if each vector is or is not a linear combination. If it is, give an example of such a linear combination. 

 
\end{enumerate}

     \end{exerciseStatement}
 \begin{exerciseAnswer} 

\begin{itemize}
\item   \(
\mathrm{RREF}\, \left[\begin{array}{ccc|c}
-2 & 2 & -2 & -6 \\
0 & 0 & 1 & 1 \\
-1 & 1 & -2 & -4 \\
-1 & 1 & -2 & -4
\end{array}\right] = \left[\begin{array}{ccc|c}
1 & -1 & 0 & 2 \\
0 & 0 & 1 & 1 \\
0 & 0 & 0 & 0 \\
0 & 0 & 0 & 0
\end{array}\right]
                        \) 

 

 \(\left[\begin{array}{c}
-6 \\
1 \\
-4 \\
-4
\end{array}\right]\) is a linear combination, for example: \(
2 \left[\begin{array}{c}
-2 \\
0 \\
-1 \\
-1
\end{array}\right] + 1 \left[\begin{array}{c}
-2 \\
1 \\
-2 \\
-2
\end{array}\right] = \left[\begin{array}{c}
-6 \\
1 \\
-4 \\
-4
\end{array}\right]
                            \) 

 
\item   \(
\mathrm{RREF}\, \left[\begin{array}{ccc|c}
-2 & 2 & -2 & -7 \\
0 & 0 & 1 & 0 \\
-1 & 1 & -2 & -5 \\
-1 & 1 & -2 & -3
\end{array}\right] = \left[\begin{array}{ccc|c}
1 & -1 & 0 & 0 \\
0 & 0 & 1 & 0 \\
0 & 0 & 0 & 1 \\
0 & 0 & 0 & 0
\end{array}\right]
                        \) 

 

 \(\left[\begin{array}{c}
-7 \\
0 \\
-5 \\
-3
\end{array}\right]\) is not a linear combination. 

 
\end{itemize}

     \end{exerciseAnswer}
 \end{exercise}
 
 
\end{document}