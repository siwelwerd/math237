\begin{exercise}
  \begin{exerciseTitle}V10 - Basis of Solution Space (ver. 570)\end{exerciseTitle}
  \begin{exerciseStatement}
    Find a basis for the solution space of the homogeneous system 
\begin{align*}
 4 x_ 1 -4 x_ 2 -5 x_ 3 + 2 x_ 4 &= 0  \\ 
  + 2 x_ 2 + 3 x_ 3 &= 0  \\ 
  -10 x_ 1 -5 x_ 2 + 4 x_ 3 -5 x_ 4 &= 0  \\ 
  6 x_ 1 + 1 x_ 2 -6 x_ 3 + 3 x_ 4 &= 0  \\ 
 \end{align*}


 
  \end{exerciseStatement}

  \begin{exerciseAnswer}
   A basis is   
\[\left\{\left[\begin{array}{c}
-\frac{1}{2} \\
0 \\
0 \\
1
\end{array}\right]\right\}.\]

  


  \end{exerciseAnswer}
\end{exercise}