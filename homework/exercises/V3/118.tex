\begin{exercise}
  \begin{exerciseTitle}V3 - Spanning sets (ver. 118)\end{exerciseTitle}
  \begin{exerciseStatement}
    Explain why the vectors \(\left[\begin{array}{r}
0 \\
-4 \\
4 \\
-5
\end{array}\right] , \left[\begin{array}{r}
0 \\
0 \\
-2 \\
-5
\end{array}\right] , \left[\begin{array}{r}
3 \\
-4 \\
-5 \\
0
\end{array}\right] , \left[\begin{array}{r}
-6 \\
4 \\
16 \\
0
\end{array}\right] , \left[\begin{array}{r}
1 \\
3 \\
-4 \\
4
\end{array}\right] , \text{ and } \left[\begin{array}{r}
4 \\
3 \\
3 \\
-4
\end{array}\right]\) span or don't span \(\mathbb{R}^4\). 
	


  \end{exerciseStatement}
  \begin{exerciseAnswer}
   The vectors \(\left[\begin{array}{r}
0 \\
-4 \\
4 \\
-5
\end{array}\right] , \left[\begin{array}{r}
0 \\
0 \\
-2 \\
-5
\end{array}\right] , \left[\begin{array}{r}
3 \\
-4 \\
-5 \\
0
\end{array}\right] , \left[\begin{array}{r}
-6 \\
4 \\
16 \\
0
\end{array}\right] , \left[\begin{array}{r}
1 \\
3 \\
-4 \\
4
\end{array}\right] , \text{ and } \left[\begin{array}{r}
4 \\
3 \\
3 \\
-4
\end{array}\right]\) 
  	 do  
	span \(\mathbb{R}^4\).
  


  \end{exerciseAnswer}
\end{exercise}