%% 
%% This is file, `homework-library.tex',
%% generated with the extract package.
%% 
%% Generated on :  2019/12/18,15:17
%% From source  :  homework-library_solutions.tex
%% Using options:  active,generate=homework-library,extract-env={exerciseStatement},extract-cmdline={startStandard}
%% 
\documentclass{article}
\usepackage{amsmath,ifthen,amsthm,amssymb,enumerate}
\usepackage[margin=1in]{geometry}
\usepackage[colorlinks=true,linkcolor=blue]{hyperref}
\usepackage{environ}

\NewEnviron{exerciseTitle}{}{}

\newcommand{\standard}{00}

\newenvironment{exercise}
{
}{
}

\newcounter{exercise}
\newenvironment{exerciseStatement}
{
	\stepcounter{exercise}
	\noindent\textbf{\standard.\theexercise}
	\ignorespaces
}{
	\vspace{0.1in}
}

\newenvironment{exerciseAnswer}
{
	\noindent
	\textit{Solution.}
	\ignorespaces
}{
	\qed

	\vspace{3em}
}

\newcommand{\IR}{\mathbb{R}}
\newcommand{\IC}{\mathbb{C}}
\renewcommand{\P}{\mathcal{P}}
\newcommand{\setList}[1]{\left\{#1\right\}}
\let\Im\relax\DeclareMathOperator{\Im}{Im}
\DeclareMathOperator{\RREF}{RREF}
\DeclareMathOperator{\vspan}{span}
\newcommand{\setBuilder}[2]{\left\{#1\,\middle|\,#2\right\}}

\makeatletter
\renewcommand*\env@matrix[1][*\c@MaxMatrixCols c]{%
  \hskip -\arraycolsep
  \let\@ifnextchar\new@ifnextchar
  \array{#1}}
\makeatother


\newcommand{\startStandard}[1]
{
	\setcounter{exercise}{0}
	\renewcommand{\standard}{#1}
	\newpage
	\subsection*{Standard \standard}
	\addcontentsline{toc}{subsection}{Standard \standard}
}


\begin{document}

\startStandard{V1}

\begin{exerciseStatement}
    Let \(V\) be the set of all pairs of numbers \((x_1,y_1)\in\mathbb{R}^2\)  together with the operations
\begin{align*}
(x_1,y_1)\oplus (x_2,y_2)&= (2x_1+x_2, y_1+2y_2)\\
c \odot (x_1,y_1) &= (cx_1,cy_1)\\
\end{align*}

Show that scalar multiplication distributes over vector addition, i.e.

\[c\odot \left((x_1,y_1)\oplus(x_2,y_2)\right)=c\odot(x_1,y_1)\oplus c\odot(x_2,y_2).\]
.
Show that \(V\) nonetheless is not a vector space.



  
\end{exerciseStatement}

\begin{exerciseStatement}
    Let \(V\) be the set of all pairs of numbers \((x_1,y_1)\in\mathbb{R}^2\)  together with the operations
\begin{align*}
(x_1,y_1)\oplus (x_2,y_2)&= (4x_1+x_2, y_1+2y_2)\\
c \odot (x_1,y_1) &= (cx_1,cy_1)\\
\end{align*}

Show that scalar multiplication distributes over vector addition, i.e.

\[c\odot \left((x_1,y_1)\oplus(x_2,y_2)\right)=c\odot(x_1,y_1)\oplus c\odot(x_2,y_2).\]
.
Show that \(V\) nonetheless is not a vector space.



  
\end{exerciseStatement}

\begin{exerciseStatement}
    Let \(V\) be the set of all pairs of numbers \((x_1,y_1)\in\mathbb{R}^2\)  together with the operations
\begin{align*}
(x_1,y_1)\oplus (x_2,y_2)&= (4x_1+x_2, y_1+2y_2)\\
c \odot (x_1,y_1) &= (cx_1,cy_1)\\
\end{align*}

Show that scalar multiplication distributes over vector addition, i.e.

\[c\odot \left((x_1,y_1)\oplus(x_2,y_2)\right)=c\odot(x_1,y_1)\oplus c\odot(x_2,y_2).\]
.
Show that \(V\) nonetheless is not a vector space.



  
\end{exerciseStatement}

\begin{exerciseStatement}
    Let \(V\) be the set of all pairs of numbers \((x_1,y_1)\in\mathbb{R}^2\)  together with the operations
\begin{align*}
(x_1,y_1)\oplus (x_2,y_2)&= (3x_1+x_2, y_1+3y_2)\\
c \odot (x_1,y_1) &= (cx_1,cy_1)\\
\end{align*}

Show that scalar multiplication distributes over vector addition, i.e.

\[c\odot \left((x_1,y_1)\oplus(x_2,y_2)\right)=c\odot(x_1,y_1)\oplus c\odot(x_2,y_2).\]
.
Show that \(V\) nonetheless is not a vector space.



  
\end{exerciseStatement}

\begin{exerciseStatement}
    Let \(V\) be the set of all pairs of numbers \((x_1,y_1)\in\mathbb{R}^2\)  together with the operations
\begin{align*}
(x_1,y_1)\oplus (x_2,y_2)&= (3x_1+x_2, y_1+1y_2)\\
c \odot (x_1,y_1) &= (cx_1,cy_1)\\
\end{align*}

Show that scalar multiplication distributes over vector addition, i.e.

\[c\odot \left((x_1,y_1)\oplus(x_2,y_2)\right)=c\odot(x_1,y_1)\oplus c\odot(x_2,y_2).\]
.
Show that \(V\) nonetheless is not a vector space.



  
\end{exerciseStatement}

\begin{exerciseStatement}
    Let \(V\) be the set of all pairs of numbers \((x_1,y_1)\in\mathbb{R}^2\)  together with the operations
\begin{align*}
(x_1,y_1)\oplus (x_2,y_2)&= (2x_1+x_2, y_1+2y_2)\\
c \odot (x_1,y_1) &= (cx_1,cy_1)\\
\end{align*}

Show that scalar multiplication distributes over vector addition, i.e.

\[c\odot \left((x_1,y_1)\oplus(x_2,y_2)\right)=c\odot(x_1,y_1)\oplus c\odot(x_2,y_2).\]
.
Show that \(V\) nonetheless is not a vector space.



  
\end{exerciseStatement}

\begin{exerciseStatement}
    Let \(V\) be the set of all pairs of numbers \((x_1,y_1)\in\mathbb{R}^2\)  together with the operations
\begin{align*}
(x_1,y_1)\oplus (x_2,y_2)&= (4x_1+x_2, y_1+3y_2)\\
c \odot (x_1,y_1) &= (cx_1,cy_1)\\
\end{align*}

Show that scalar multiplication distributes over vector addition, i.e.

\[c\odot \left((x_1,y_1)\oplus(x_2,y_2)\right)=c\odot(x_1,y_1)\oplus c\odot(x_2,y_2).\]
.
Show that \(V\) nonetheless is not a vector space.



  
\end{exerciseStatement}

\begin{exerciseStatement}
    Let \(V\) be the set of all pairs of numbers \((x_1,y_1)\in\mathbb{R}^2\)  together with the operations
\begin{align*}
(x_1,y_1)\oplus (x_2,y_2)&= (4x_1+x_2, y_1+3y_2)\\
c \odot (x_1,y_1) &= (cx_1,cy_1)\\
\end{align*}

Show that scalar multiplication distributes over vector addition, i.e.

\[c\odot \left((x_1,y_1)\oplus(x_2,y_2)\right)=c\odot(x_1,y_1)\oplus c\odot(x_2,y_2).\]
.
Show that \(V\) nonetheless is not a vector space.



  
\end{exerciseStatement}

\begin{exerciseStatement}
    Let \(V\) be the set of all pairs of numbers \((x_1,y_1)\in\mathbb{R}^2\)  together with the operations
\begin{align*}
(x_1,y_1)\oplus (x_2,y_2)&= (4x_1+x_2, y_1+3y_2)\\
c \odot (x_1,y_1) &= (cx_1,cy_1)\\
\end{align*}

Show that scalar multiplication distributes over vector addition, i.e.

\[c\odot \left((x_1,y_1)\oplus(x_2,y_2)\right)=c\odot(x_1,y_1)\oplus c\odot(x_2,y_2).\]
.
Show that \(V\) nonetheless is not a vector space.



  
\end{exerciseStatement}

\begin{exerciseStatement}
    Let \(V\) be the set of all pairs of numbers \((x_1,y_1)\in\mathbb{R}^2\)  together with the operations
\begin{align*}
(x_1,y_1)\oplus (x_2,y_2)&= (3x_1+x_2, y_1+1y_2)\\
c \odot (x_1,y_1) &= (cx_1,cy_1)\\
\end{align*}

Show that scalar multiplication distributes over vector addition, i.e.

\[c\odot \left((x_1,y_1)\oplus(x_2,y_2)\right)=c\odot(x_1,y_1)\oplus c\odot(x_2,y_2).\]
.
Show that \(V\) nonetheless is not a vector space.



  
\end{exerciseStatement}

\begin{exerciseStatement}
    Let \(V\) be the set of all pairs of numbers \((x_1,y_1)\in\mathbb{R}^2\)  together with the operations
\begin{align*}
(x_1,y_1)\oplus (x_2,y_2)&= (4x_1+x_2, y_1+2y_2)\\
c \odot (x_1,y_1) &= (cx_1,cy_1)\\
\end{align*}

Show that scalar multiplication distributes over vector addition, i.e.

\[c\odot \left((x_1,y_1)\oplus(x_2,y_2)\right)=c\odot(x_1,y_1)\oplus c\odot(x_2,y_2).\]
.
Show that \(V\) nonetheless is not a vector space.



  
\end{exerciseStatement}

\begin{exerciseStatement}
    Let \(V\) be the set of all pairs of numbers \((x_1,y_1)\in\mathbb{R}^2\)  together with the operations
\begin{align*}
(x_1,y_1)\oplus (x_2,y_2)&= (3x_1+x_2, y_1+2y_2)\\
c \odot (x_1,y_1) &= (cx_1,cy_1)\\
\end{align*}

Show that scalar multiplication distributes over vector addition, i.e.

\[c\odot \left((x_1,y_1)\oplus(x_2,y_2)\right)=c\odot(x_1,y_1)\oplus c\odot(x_2,y_2).\]
.
Show that \(V\) nonetheless is not a vector space.



  
\end{exerciseStatement}

\begin{exerciseStatement}
    Let \(V\) be the set of all pairs of numbers \((x_1,y_1)\in\mathbb{R}^2\)  together with the operations
\begin{align*}
(x_1,y_1)\oplus (x_2,y_2)&= (3x_1+x_2, y_1+2y_2)\\
c \odot (x_1,y_1) &= (cx_1,cy_1)\\
\end{align*}

Show that scalar multiplication distributes over vector addition, i.e.

\[c\odot \left((x_1,y_1)\oplus(x_2,y_2)\right)=c\odot(x_1,y_1)\oplus c\odot(x_2,y_2).\]
.
Show that \(V\) nonetheless is not a vector space.



  
\end{exerciseStatement}

\begin{exerciseStatement}
    Let \(V\) be the set of all pairs of numbers \((x_1,y_1)\in\mathbb{R}^2\)  together with the operations
\begin{align*}
(x_1,y_1)\oplus (x_2,y_2)&= (2x_1+x_2, y_1+3y_2)\\
c \odot (x_1,y_1) &= (cx_1,cy_1)\\
\end{align*}

Show that scalar multiplication distributes over vector addition, i.e.

\[c\odot \left((x_1,y_1)\oplus(x_2,y_2)\right)=c\odot(x_1,y_1)\oplus c\odot(x_2,y_2).\]
.
Show that \(V\) nonetheless is not a vector space.



  
\end{exerciseStatement}

\begin{exerciseStatement}
    Let \(V\) be the set of all pairs of numbers \((x_1,y_1)\in\mathbb{R}^2\)  together with the operations
\begin{align*}
(x_1,y_1)\oplus (x_2,y_2)&= (2x_1+x_2, y_1+1y_2)\\
c \odot (x_1,y_1) &= (cx_1,cy_1)\\
\end{align*}

Show that scalar multiplication distributes over vector addition, i.e.

\[c\odot \left((x_1,y_1)\oplus(x_2,y_2)\right)=c\odot(x_1,y_1)\oplus c\odot(x_2,y_2).\]
.
Show that \(V\) nonetheless is not a vector space.



  
\end{exerciseStatement}

\begin{exerciseStatement}
    Let \(V\) be the set of all pairs of numbers \((x_1,y_1)\in\mathbb{R}^2\)  together with the operations
\begin{align*}
(x_1,y_1)\oplus (x_2,y_2)&= (4x_1+x_2, y_1+3y_2)\\
c \odot (x_1,y_1) &= (cx_1,cy_1)\\
\end{align*}

Show that scalar multiplication distributes over vector addition, i.e.

\[c\odot \left((x_1,y_1)\oplus(x_2,y_2)\right)=c\odot(x_1,y_1)\oplus c\odot(x_2,y_2).\]
.
Show that \(V\) nonetheless is not a vector space.



  
\end{exerciseStatement}

\begin{exerciseStatement}
    Let \(V\) be the set of all pairs of numbers \((x_1,y_1)\in\mathbb{R}^2\)  together with the operations
\begin{align*}
(x_1,y_1)\oplus (x_2,y_2)&= (3x_1+x_2, y_1+2y_2)\\
c \odot (x_1,y_1) &= (cx_1,cy_1)\\
\end{align*}

Show that scalar multiplication distributes over vector addition, i.e.

\[c\odot \left((x_1,y_1)\oplus(x_2,y_2)\right)=c\odot(x_1,y_1)\oplus c\odot(x_2,y_2).\]
.
Show that \(V\) nonetheless is not a vector space.



  
\end{exerciseStatement}

\begin{exerciseStatement}
    Let \(V\) be the set of all pairs of numbers \((x_1,y_1)\in\mathbb{R}^2\)  together with the operations
\begin{align*}
(x_1,y_1)\oplus (x_2,y_2)&= (3x_1+x_2, y_1+2y_2)\\
c \odot (x_1,y_1) &= (cx_1,cy_1)\\
\end{align*}

Show that scalar multiplication distributes over vector addition, i.e.

\[c\odot \left((x_1,y_1)\oplus(x_2,y_2)\right)=c\odot(x_1,y_1)\oplus c\odot(x_2,y_2).\]
.
Show that \(V\) nonetheless is not a vector space.



  
\end{exerciseStatement}

\begin{exerciseStatement}
    Let \(V\) be the set of all pairs of numbers \((x_1,y_1)\in\mathbb{R}^2\)  together with the operations
\begin{align*}
(x_1,y_1)\oplus (x_2,y_2)&= (4x_1+x_2, y_1+2y_2)\\
c \odot (x_1,y_1) &= (cx_1,cy_1)\\
\end{align*}

Show that scalar multiplication distributes over vector addition, i.e.

\[c\odot \left((x_1,y_1)\oplus(x_2,y_2)\right)=c\odot(x_1,y_1)\oplus c\odot(x_2,y_2).\]
.
Show that \(V\) nonetheless is not a vector space.



  
\end{exerciseStatement}

\begin{exerciseStatement}
    Let \(V\) be the set of all pairs of numbers \((x_1,y_1)\in\mathbb{R}^2\)  together with the operations
\begin{align*}
(x_1,y_1)\oplus (x_2,y_2)&= (3x_1+x_2, y_1+3y_2)\\
c \odot (x_1,y_1) &= (cx_1,cy_1)\\
\end{align*}

Show that scalar multiplication distributes over vector addition, i.e.

\[c\odot \left((x_1,y_1)\oplus(x_2,y_2)\right)=c\odot(x_1,y_1)\oplus c\odot(x_2,y_2).\]
.
Show that \(V\) nonetheless is not a vector space.



  
\end{exerciseStatement}

\begin{exerciseStatement}
    Let \(V\) be the set of all pairs of numbers \((x_1,y_1)\in\mathbb{R}^2\)  together with the operations
\begin{align*}
(x_1,y_1)\oplus (x_2,y_2)&= (3x_1+x_2, y_1+2y_2)\\
c \odot (x_1,y_1) &= (cx_1,cy_1)\\
\end{align*}

Show that scalar multiplication distributes over vector addition, i.e.

\[c\odot \left((x_1,y_1)\oplus(x_2,y_2)\right)=c\odot(x_1,y_1)\oplus c\odot(x_2,y_2).\]
.
Show that \(V\) nonetheless is not a vector space.



  
\end{exerciseStatement}

\begin{exerciseStatement}
    Let \(V\) be the set of all pairs of numbers \((x_1,y_1)\in\mathbb{R}^2\)  together with the operations
\begin{align*}
(x_1,y_1)\oplus (x_2,y_2)&= (2x_1+x_2, y_1+1y_2)\\
c \odot (x_1,y_1) &= (cx_1,cy_1)\\
\end{align*}

Show that scalar multiplication distributes over vector addition, i.e.

\[c\odot \left((x_1,y_1)\oplus(x_2,y_2)\right)=c\odot(x_1,y_1)\oplus c\odot(x_2,y_2).\]
.
Show that \(V\) nonetheless is not a vector space.



  
\end{exerciseStatement}

\begin{exerciseStatement}
    Let \(V\) be the set of all pairs of numbers \((x_1,y_1)\in\mathbb{R}^2\)  together with the operations
\begin{align*}
(x_1,y_1)\oplus (x_2,y_2)&= (2x_1+x_2, y_1+2y_2)\\
c \odot (x_1,y_1) &= (cx_1,cy_1)\\
\end{align*}

Show that scalar multiplication distributes over vector addition, i.e.

\[c\odot \left((x_1,y_1)\oplus(x_2,y_2)\right)=c\odot(x_1,y_1)\oplus c\odot(x_2,y_2).\]
.
Show that \(V\) nonetheless is not a vector space.



  
\end{exerciseStatement}

\begin{exerciseStatement}
    Let \(V\) be the set of all pairs of numbers \((x_1,y_1)\in\mathbb{R}^2\)  together with the operations
\begin{align*}
(x_1,y_1)\oplus (x_2,y_2)&= (2x_1+x_2, y_1+1y_2)\\
c \odot (x_1,y_1) &= (cx_1,cy_1)\\
\end{align*}

Show that scalar multiplication distributes over vector addition, i.e.

\[c\odot \left((x_1,y_1)\oplus(x_2,y_2)\right)=c\odot(x_1,y_1)\oplus c\odot(x_2,y_2).\]
.
Show that \(V\) nonetheless is not a vector space.



  
\end{exerciseStatement}

\begin{exerciseStatement}
    Let \(V\) be the set of all pairs of numbers \((x_1,y_1)\in\mathbb{R}^2\)  together with the operations
\begin{align*}
(x_1,y_1)\oplus (x_2,y_2)&= (3x_1+x_2, y_1+3y_2)\\
c \odot (x_1,y_1) &= (cx_1,cy_1)\\
\end{align*}

Show that scalar multiplication distributes over vector addition, i.e.

\[c\odot \left((x_1,y_1)\oplus(x_2,y_2)\right)=c\odot(x_1,y_1)\oplus c\odot(x_2,y_2).\]
.
Show that \(V\) nonetheless is not a vector space.



  
\end{exerciseStatement}

\startStandard{V2}

\begin{exerciseStatement}
    Explain why the vector \(\left[\begin{array}{c}
4 \\
-5 \\
3 \\
-5
\end{array}\right]\)  is or is not a linear
combination of the vectors \(\left[\begin{array}{c}
0 \\
-3 \\
-2 \\
0
\end{array}\right] , \left[\begin{array}{c}
-4 \\
1 \\
-5 \\
2
\end{array}\right] , \text{ and } \left[\begin{array}{c}
0 \\
2 \\
2 \\
-3
\end{array}\right]\).



  
\end{exerciseStatement}

\begin{exerciseStatement}
    Explain why the vector \(\left[\begin{array}{c}
3 \\
-2 \\
3 \\
3
\end{array}\right]\)  is or is not a linear
combination of the vectors \(\left[\begin{array}{c}
0 \\
-1 \\
-5 \\
2
\end{array}\right] , \left[\begin{array}{c}
2 \\
-3 \\
-4 \\
-3
\end{array}\right] , \left[\begin{array}{c}
-3 \\
-1 \\
-4 \\
-2
\end{array}\right] , \left[\begin{array}{c}
0 \\
11 \\
20 \\
13
\end{array}\right] , \text{ and } \left[\begin{array}{c}
3 \\
4 \\
19 \\
-4
\end{array}\right]\).



  
\end{exerciseStatement}

\begin{exerciseStatement}
    Explain why the vector \(\left[\begin{array}{c}
-7 \\
9 \\
4 \\
1
\end{array}\right]\)  is or is not a linear
combination of the vectors \(\left[\begin{array}{c}
-1 \\
-3 \\
-2 \\
-2
\end{array}\right] , \left[\begin{array}{c}
1 \\
2 \\
3 \\
-2
\end{array}\right] , \left[\begin{array}{c}
-1 \\
-2 \\
-3 \\
2
\end{array}\right] , \left[\begin{array}{c}
-6 \\
-13 \\
-17 \\
8
\end{array}\right] , \text{ and } \left[\begin{array}{c}
4 \\
10 \\
10 \\
0
\end{array}\right]\).



  
\end{exerciseStatement}

\begin{exerciseStatement}
    Explain why the vector \(\left[\begin{array}{c}
5 \\
-1 \\
2 \\
-4
\end{array}\right]\)  is or is not a linear
combination of the vectors \(\left[\begin{array}{c}
3 \\
1 \\
-1 \\
1
\end{array}\right] , \left[\begin{array}{c}
2 \\
1 \\
-4 \\
2
\end{array}\right] , \left[\begin{array}{c}
-6 \\
-2 \\
2 \\
-2
\end{array}\right] , \text{ and } \left[\begin{array}{c}
0 \\
0 \\
0 \\
0
\end{array}\right]\).



  
\end{exerciseStatement}

\begin{exerciseStatement}
    Explain why the vector \(\left[\begin{array}{c}
-2 \\
-6 \\
7 \\
5
\end{array}\right]\)  is or is not a linear
combination of the vectors \(\left[\begin{array}{c}
-2 \\
-3 \\
0 \\
2
\end{array}\right] , \left[\begin{array}{c}
-5 \\
0 \\
-3 \\
-2
\end{array}\right] , \left[\begin{array}{c}
-1 \\
0 \\
3 \\
0
\end{array}\right] , \text{ and } \left[\begin{array}{c}
9 \\
6 \\
3 \\
-2
\end{array}\right]\).



  
\end{exerciseStatement}

\begin{exerciseStatement}
    Explain why the vector \(\left[\begin{array}{c}
-6 \\
6 \\
5 \\
-5
\end{array}\right]\)  is or is not a linear
combination of the vectors \(\left[\begin{array}{c}
0 \\
-2 \\
-2 \\
0
\end{array}\right] , \left[\begin{array}{c}
-3 \\
-1 \\
-2 \\
-2
\end{array}\right] , \text{ and } \left[\begin{array}{c}
-6 \\
2 \\
0 \\
-4
\end{array}\right]\).



  
\end{exerciseStatement}

\begin{exerciseStatement}
    Explain why the vector \(\left[\begin{array}{c}
2 \\
-8 \\
-12 \\
8
\end{array}\right]\)  is or is not a linear
combination of the vectors \(\left[\begin{array}{c}
3 \\
-1 \\
-3 \\
1
\end{array}\right] , \left[\begin{array}{c}
-1 \\
1 \\
3 \\
-3
\end{array}\right] , \left[\begin{array}{c}
1 \\
-3 \\
-3 \\
0
\end{array}\right] , \left[\begin{array}{c}
0 \\
2 \\
1 \\
2
\end{array}\right] , \text{ and } \left[\begin{array}{c}
4 \\
1 \\
-5 \\
0
\end{array}\right]\).



  
\end{exerciseStatement}

\begin{exerciseStatement}
    Explain why the vector \(\left[\begin{array}{c}
-9 \\
-5 \\
-15 \\
-7
\end{array}\right]\)  is or is not a linear
combination of the vectors \(\left[\begin{array}{c}
2 \\
0 \\
4 \\
1
\end{array}\right] , \left[\begin{array}{c}
0 \\
1 \\
0 \\
2
\end{array}\right] , \left[\begin{array}{c}
-3 \\
-2 \\
-3 \\
2
\end{array}\right] , \left[\begin{array}{c}
-1 \\
2 \\
-1 \\
-2
\end{array}\right] , \text{ and } \left[\begin{array}{c}
1 \\
-3 \\
3 \\
2
\end{array}\right]\).



  
\end{exerciseStatement}

\begin{exerciseStatement}
    Explain why the vector \(\left[\begin{array}{c}
-24 \\
-2 \\
13 \\
11
\end{array}\right]\)  is or is not a linear
combination of the vectors \(\left[\begin{array}{c}
4 \\
0 \\
-4 \\
2
\end{array}\right] , \left[\begin{array}{c}
4 \\
0 \\
1 \\
-3
\end{array}\right] , \left[\begin{array}{c}
2 \\
1 \\
-4 \\
-3
\end{array}\right] , \left[\begin{array}{c}
-3 \\
-2 \\
2 \\
-3
\end{array}\right] , \text{ and } \left[\begin{array}{c}
0 \\
0 \\
-4 \\
1
\end{array}\right]\).



  
\end{exerciseStatement}

\begin{exerciseStatement}
    Explain why the vector \(\left[\begin{array}{c}
-9 \\
0 \\
-4 \\
-4
\end{array}\right]\)  is or is not a linear
combination of the vectors \(\left[\begin{array}{c}
-3 \\
0 \\
-2 \\
-2
\end{array}\right] , \left[\begin{array}{c}
-1 \\
0 \\
4 \\
1
\end{array}\right] , \left[\begin{array}{c}
1 \\
0 \\
0 \\
0
\end{array}\right] , \text{ and } \left[\begin{array}{c}
-2 \\
0 \\
3 \\
0
\end{array}\right]\).



  
\end{exerciseStatement}

\begin{exerciseStatement}
    Explain why the vector \(\left[\begin{array}{c}
2 \\
10 \\
2 \\
-2
\end{array}\right]\)  is or is not a linear
combination of the vectors \(\left[\begin{array}{c}
-1 \\
-1 \\
-5 \\
2
\end{array}\right] , \left[\begin{array}{c}
-4 \\
-2 \\
-1 \\
1
\end{array}\right] , \left[\begin{array}{c}
-1 \\
-3 \\
4 \\
-1
\end{array}\right] , \left[\begin{array}{c}
3 \\
1 \\
2 \\
2
\end{array}\right] , \text{ and } \left[\begin{array}{c}
3 \\
1 \\
2 \\
1
\end{array}\right]\).



  
\end{exerciseStatement}

\begin{exerciseStatement}
    Explain why the vector \(\left[\begin{array}{c}
8 \\
-1 \\
17 \\
6
\end{array}\right]\)  is or is not a linear
combination of the vectors \(\left[\begin{array}{c}
-1 \\
-3 \\
-4 \\
-2
\end{array}\right] , \left[\begin{array}{c}
-4 \\
-2 \\
-1 \\
-1
\end{array}\right] , \left[\begin{array}{c}
-5 \\
0 \\
-5 \\
-2
\end{array}\right] , \text{ and } \left[\begin{array}{c}
-1 \\
-1 \\
-2 \\
0
\end{array}\right]\).



  
\end{exerciseStatement}

\begin{exerciseStatement}
    Explain why the vector \(\left[\begin{array}{c}
-9 \\
2 \\
-6 \\
0
\end{array}\right]\)  is or is not a linear
combination of the vectors \(\left[\begin{array}{c}
0 \\
-1 \\
3 \\
-1
\end{array}\right] , \left[\begin{array}{c}
3 \\
-2 \\
3 \\
1
\end{array}\right] , \left[\begin{array}{c}
3 \\
1 \\
3 \\
-3
\end{array}\right] , \text{ and } \left[\begin{array}{c}
2 \\
-2 \\
3 \\
1
\end{array}\right]\).



  
\end{exerciseStatement}

\begin{exerciseStatement}
    Explain why the vector \(\left[\begin{array}{c}
9 \\
15 \\
2 \\
3
\end{array}\right]\)  is or is not a linear
combination of the vectors \(\left[\begin{array}{c}
3 \\
-3 \\
2 \\
-3
\end{array}\right] , \left[\begin{array}{c}
-2 \\
-3 \\
-1 \\
0
\end{array}\right] , \text{ and } \left[\begin{array}{c}
0 \\
-3 \\
1 \\
-2
\end{array}\right]\).



  
\end{exerciseStatement}

\begin{exerciseStatement}
    Explain why the vector \(\left[\begin{array}{c}
-7 \\
7 \\
5 \\
9
\end{array}\right]\)  is or is not a linear
combination of the vectors \(\left[\begin{array}{c}
-4 \\
2 \\
-3 \\
-1
\end{array}\right] , \left[\begin{array}{c}
0 \\
-2 \\
2 \\
1
\end{array}\right] , \text{ and } \left[\begin{array}{c}
-8 \\
2 \\
-4 \\
-1
\end{array}\right]\).



  
\end{exerciseStatement}

\begin{exerciseStatement}
    Explain why the vector \(\left[\begin{array}{c}
4 \\
-5 \\
4 \\
8
\end{array}\right]\)  is or is not a linear
combination of the vectors \(\left[\begin{array}{c}
1 \\
-2 \\
1 \\
0
\end{array}\right] , \left[\begin{array}{c}
-4 \\
1 \\
-3 \\
0
\end{array}\right] , \left[\begin{array}{c}
-2 \\
-3 \\
2 \\
-2
\end{array}\right] , \left[\begin{array}{c}
-3 \\
-1 \\
4 \\
-4
\end{array}\right] , \text{ and } \left[\begin{array}{c}
4 \\
6 \\
-1 \\
2
\end{array}\right]\).



  
\end{exerciseStatement}

\begin{exerciseStatement}
    Explain why the vector \(\left[\begin{array}{c}
9 \\
6 \\
-14 \\
3
\end{array}\right]\)  is or is not a linear
combination of the vectors \(\left[\begin{array}{c}
0 \\
-3 \\
2 \\
-3
\end{array}\right] , \left[\begin{array}{c}
3 \\
-3 \\
-4 \\
0
\end{array}\right] , \left[\begin{array}{c}
-3 \\
-3 \\
4 \\
0
\end{array}\right] , \text{ and } \left[\begin{array}{c}
2 \\
-2 \\
1 \\
-1
\end{array}\right]\).



  
\end{exerciseStatement}

\begin{exerciseStatement}
    Explain why the vector \(\left[\begin{array}{c}
8 \\
-7 \\
-9 \\
2
\end{array}\right]\)  is or is not a linear
combination of the vectors \(\left[\begin{array}{c}
-1 \\
-1 \\
3 \\
-1
\end{array}\right] , \left[\begin{array}{c}
2 \\
-2 \\
3 \\
-3
\end{array}\right] , \left[\begin{array}{c}
-3 \\
2 \\
4 \\
-1
\end{array}\right] , \text{ and } \left[\begin{array}{c}
-3 \\
-1 \\
2 \\
-3
\end{array}\right]\).



  
\end{exerciseStatement}

\begin{exerciseStatement}
    Explain why the vector \(\left[\begin{array}{c}
8 \\
4 \\
-7 \\
-2
\end{array}\right]\)  is or is not a linear
combination of the vectors \(\left[\begin{array}{c}
-2 \\
0 \\
-3 \\
2
\end{array}\right] , \left[\begin{array}{c}
-1 \\
1 \\
-4 \\
-3
\end{array}\right] , \left[\begin{array}{c}
3 \\
-1 \\
1 \\
-1
\end{array}\right] , \left[\begin{array}{c}
7 \\
-3 \\
18 \\
5
\end{array}\right] , \text{ and } \left[\begin{array}{c}
3 \\
-3 \\
12 \\
9
\end{array}\right]\).



  
\end{exerciseStatement}

\begin{exerciseStatement}
    Explain why the vector \(\left[\begin{array}{c}
-3 \\
-12 \\
-5 \\
5
\end{array}\right]\)  is or is not a linear
combination of the vectors \(\left[\begin{array}{c}
4 \\
2 \\
0 \\
-2
\end{array}\right] , \left[\begin{array}{c}
-3 \\
0 \\
-2 \\
-1
\end{array}\right] , \left[\begin{array}{c}
-1 \\
2 \\
3 \\
1
\end{array}\right] , \text{ and } \left[\begin{array}{c}
-4 \\
1 \\
2 \\
1
\end{array}\right]\).



  
\end{exerciseStatement}

\begin{exerciseStatement}
    Explain why the vector \(\left[\begin{array}{c}
-8 \\
6 \\
-9 \\
-4
\end{array}\right]\)  is or is not a linear
combination of the vectors \(\left[\begin{array}{c}
1 \\
0 \\
2 \\
1
\end{array}\right] , \left[\begin{array}{c}
-4 \\
-2 \\
3 \\
0
\end{array}\right] , \left[\begin{array}{c}
3 \\
1 \\
3 \\
1
\end{array}\right] , \text{ and } \left[\begin{array}{c}
-2 \\
0 \\
-9 \\
-2
\end{array}\right]\).



  
\end{exerciseStatement}

\begin{exerciseStatement}
    Explain why the vector \(\left[\begin{array}{c}
14 \\
5 \\
-7 \\
-2
\end{array}\right]\)  is or is not a linear
combination of the vectors \(\left[\begin{array}{c}
-5 \\
-2 \\
1 \\
2
\end{array}\right] , \left[\begin{array}{c}
-2 \\
0 \\
-1 \\
-2
\end{array}\right] , \text{ and } \left[\begin{array}{c}
-1 \\
-1 \\
3 \\
2
\end{array}\right]\).



  
\end{exerciseStatement}

\begin{exerciseStatement}
    Explain why the vector \(\left[\begin{array}{c}
7 \\
3 \\
-3 \\
4
\end{array}\right]\)  is or is not a linear
combination of the vectors \(\left[\begin{array}{c}
-1 \\
-3 \\
-5 \\
2
\end{array}\right] , \left[\begin{array}{c}
4 \\
0 \\
2 \\
0
\end{array}\right] , \text{ and } \left[\begin{array}{c}
3 \\
-2 \\
-2 \\
1
\end{array}\right]\).



  
\end{exerciseStatement}

\begin{exerciseStatement}
    Explain why the vector \(\left[\begin{array}{c}
4 \\
-3 \\
8 \\
2
\end{array}\right]\)  is or is not a linear
combination of the vectors \(\left[\begin{array}{c}
-4 \\
-3 \\
-2 \\
2
\end{array}\right] , \left[\begin{array}{c}
-4 \\
0 \\
-5 \\
0
\end{array}\right] , \text{ and } \left[\begin{array}{c}
3 \\
-2 \\
-3 \\
-1
\end{array}\right]\).



  
\end{exerciseStatement}

\begin{exerciseStatement}
    Explain why the vector \(\left[\begin{array}{c}
-2 \\
2 \\
3 \\
7
\end{array}\right]\)  is or is not a linear
combination of the vectors \(\left[\begin{array}{c}
3 \\
-2 \\
0 \\
2
\end{array}\right] , \left[\begin{array}{c}
-2 \\
-2 \\
-1 \\
-3
\end{array}\right] , \left[\begin{array}{c}
4 \\
2 \\
0 \\
1
\end{array}\right] , \text{ and } \left[\begin{array}{c}
2 \\
1 \\
-5 \\
-2
\end{array}\right]\).



  
\end{exerciseStatement}

\startStandard{V3}

\begin{exerciseStatement}
    Explain why the vectors \(\left[\begin{array}{r}
0 \\
-5 \\
-2 \\
0
\end{array}\right] , \left[\begin{array}{r}
-4 \\
2 \\
-5 \\
3
\end{array}\right] , \left[\begin{array}{r}
0 \\
3 \\
2 \\
-4
\end{array}\right] , \text{ and } \left[\begin{array}{r}
4 \\
-6 \\
5 \\
-11
\end{array}\right]\) span or don't span \(\mathbb{R}^4\).



  
\end{exerciseStatement}

\begin{exerciseStatement}
    Explain why the vectors \(\left[\begin{array}{r}
0 \\
-1 \\
-5 \\
4
\end{array}\right] , \left[\begin{array}{r}
2 \\
-4 \\
-4 \\
-5
\end{array}\right] , \left[\begin{array}{r}
-3 \\
-1 \\
-4 \\
-2
\end{array}\right] , \left[\begin{array}{r}
-2 \\
4 \\
0 \\
-3
\end{array}\right] , \left[\begin{array}{r}
1 \\
-3 \\
-5 \\
-3
\end{array}\right] , \text{ and } \left[\begin{array}{r}
-2 \\
1 \\
-1 \\
1
\end{array}\right]\) span or don't span \(\mathbb{R}^4\).



  
\end{exerciseStatement}

\begin{exerciseStatement}
    Explain why the vectors \(\left[\begin{array}{r}
-1 \\
-4 \\
-2 \\
-3
\end{array}\right] , \left[\begin{array}{r}
1 \\
4 \\
3 \\
-2
\end{array}\right] , \left[\begin{array}{r}
2 \\
-1 \\
0 \\
3
\end{array}\right] , \left[\begin{array}{r}
-5 \\
-4 \\
3 \\
-4
\end{array}\right] , \left[\begin{array}{r}
1 \\
3 \\
-3 \\
3
\end{array}\right] , \text{ and } \left[\begin{array}{r}
-1 \\
-4 \\
4 \\
2
\end{array}\right]\) span or don't span \(\mathbb{R}^4\).



  
\end{exerciseStatement}

\begin{exerciseStatement}
    Explain why the vectors \(\left[\begin{array}{r}
3 \\
3 \\
-1 \\
2
\end{array}\right] , \left[\begin{array}{r}
2 \\
1 \\
-4 \\
4
\end{array}\right] , \left[\begin{array}{r}
-6 \\
-6 \\
2 \\
-4
\end{array}\right] , \left[\begin{array}{r}
-4 \\
-5 \\
-2 \\
0
\end{array}\right] , \text{ and } \left[\begin{array}{r}
-2 \\
2 \\
-1 \\
1
\end{array}\right]\) span or don't span \(\mathbb{R}^4\).



  
\end{exerciseStatement}

\begin{exerciseStatement}
    Explain why the vectors \(\left[\begin{array}{r}
-2 \\
-4 \\
0 \\
3
\end{array}\right] , \left[\begin{array}{r}
-5 \\
0 \\
-3 \\
-4
\end{array}\right] , \left[\begin{array}{r}
-1 \\
1 \\
3 \\
0
\end{array}\right] , \left[\begin{array}{r}
1 \\
-3 \\
-4 \\
-2
\end{array}\right] , \text{ and } \left[\begin{array}{r}
-1 \\
-1 \\
-3 \\
3
\end{array}\right]\) span or don't span \(\mathbb{R}^4\).



  
\end{exerciseStatement}

\begin{exerciseStatement}
    Explain why the vectors \(\left[\begin{array}{r}
0 \\
-3 \\
-2 \\
0
\end{array}\right] , \left[\begin{array}{r}
-3 \\
-1 \\
-2 \\
-2
\end{array}\right] , \left[\begin{array}{r}
-9 \\
3 \\
-2 \\
-6
\end{array}\right] , \text{ and } \left[\begin{array}{r}
0 \\
6 \\
4 \\
0
\end{array}\right]\) span or don't span \(\mathbb{R}^4\).



  
\end{exerciseStatement}

\begin{exerciseStatement}
    Explain why the vectors \(\left[\begin{array}{r}
3 \\
-2 \\
-3 \\
1
\end{array}\right] , \left[\begin{array}{r}
-1 \\
2 \\
3 \\
-5
\end{array}\right] , \left[\begin{array}{r}
1 \\
-5 \\
-3 \\
1
\end{array}\right] , \left[\begin{array}{r}
0 \\
4 \\
1 \\
4
\end{array}\right] , \left[\begin{array}{r}
4 \\
3 \\
-5 \\
0
\end{array}\right] , \text{ and } \left[\begin{array}{r}
2 \\
-2 \\
-4 \\
3
\end{array}\right]\) span or don't span \(\mathbb{R}^4\).



  
\end{exerciseStatement}

\begin{exerciseStatement}
    Explain why the vectors \(\left[\begin{array}{r}
2 \\
0 \\
4 \\
2
\end{array}\right] , \left[\begin{array}{r}
0 \\
2 \\
0 \\
3
\end{array}\right] , \left[\begin{array}{r}
-6 \\
0 \\
-12 \\
-6
\end{array}\right] , \left[\begin{array}{r}
12 \\
0 \\
24 \\
12
\end{array}\right] , \left[\begin{array}{r}
1 \\
-5 \\
3 \\
4
\end{array}\right] , \text{ and } \left[\begin{array}{r}
1 \\
-5 \\
-4 \\
2
\end{array}\right]\) span or don't span \(\mathbb{R}^4\).



  
\end{exerciseStatement}

\begin{exerciseStatement}
    Explain why the vectors \(\left[\begin{array}{r}
4 \\
0 \\
-4 \\
4
\end{array}\right] , \left[\begin{array}{r}
4 \\
0 \\
1 \\
-4
\end{array}\right] , \left[\begin{array}{r}
2 \\
2 \\
-4 \\
-5
\end{array}\right] , \left[\begin{array}{r}
2 \\
-2 \\
10 \\
-7
\end{array}\right] , \left[\begin{array}{r}
0 \\
0 \\
-4 \\
2
\end{array}\right] , \text{ and } \left[\begin{array}{r}
3 \\
-3 \\
-4 \\
-2
\end{array}\right]\) span or don't span \(\mathbb{R}^4\).



  
\end{exerciseStatement}

\begin{exerciseStatement}
    Explain why the vectors \(\left[\begin{array}{r}
-3 \\
0 \\
-2 \\
-3
\end{array}\right] , \left[\begin{array}{r}
-1 \\
0 \\
4 \\
2
\end{array}\right] , \left[\begin{array}{r}
1 \\
0 \\
0 \\
0
\end{array}\right] , \left[\begin{array}{r}
-2 \\
0 \\
3 \\
1
\end{array}\right] , \text{ and } \left[\begin{array}{r}
3 \\
4 \\
1 \\
-5
\end{array}\right]\) span or don't span \(\mathbb{R}^4\).



  
\end{exerciseStatement}

\begin{exerciseStatement}
    Explain why the vectors \(\left[\begin{array}{r}
-1 \\
-1 \\
-5 \\
4
\end{array}\right] , \left[\begin{array}{r}
-4 \\
-3 \\
-1 \\
2
\end{array}\right] , \left[\begin{array}{r}
4 \\
3 \\
1 \\
-2
\end{array}\right] , \left[\begin{array}{r}
4 \\
3 \\
1 \\
-2
\end{array}\right] , \left[\begin{array}{r}
3 \\
3 \\
2 \\
3
\end{array}\right] , \text{ and } \left[\begin{array}{r}
2 \\
-4 \\
2 \\
-5
\end{array}\right]\) span or don't span \(\mathbb{R}^4\).



  
\end{exerciseStatement}

\begin{exerciseStatement}
    Explain why the vectors \(\left[\begin{array}{r}
-1 \\
-4 \\
-4 \\
-3
\end{array}\right] , \left[\begin{array}{r}
-4 \\
-2 \\
-1 \\
-2
\end{array}\right] , \left[\begin{array}{r}
7 \\
0 \\
-2 \\
1
\end{array}\right] , \left[\begin{array}{r}
-3 \\
2 \\
3 \\
1
\end{array}\right] , \text{ and } \left[\begin{array}{r}
3 \\
-2 \\
4 \\
-5
\end{array}\right]\) span or don't span \(\mathbb{R}^4\).



  
\end{exerciseStatement}

\begin{exerciseStatement}
    Explain why the vectors \(\left[\begin{array}{r}
0 \\
-1 \\
3 \\
-1
\end{array}\right] , \left[\begin{array}{r}
3 \\
-2 \\
3 \\
2
\end{array}\right] , \left[\begin{array}{r}
-9 \\
9 \\
-18 \\
-3
\end{array}\right] , \left[\begin{array}{r}
-6 \\
3 \\
-3 \\
-5
\end{array}\right] , \text{ and } \left[\begin{array}{r}
3 \\
3 \\
-3 \\
-1
\end{array}\right]\) span or don't span \(\mathbb{R}^4\).



  
\end{exerciseStatement}

\begin{exerciseStatement}
    Explain why the vectors \(\left[\begin{array}{r}
3 \\
-4 \\
2 \\
-4
\end{array}\right] , \left[\begin{array}{r}
-2 \\
-5 \\
-1 \\
1
\end{array}\right] , \left[\begin{array}{r}
0 \\
-4 \\
1 \\
-4
\end{array}\right] , \text{ and } \left[\begin{array}{r}
4 \\
3 \\
-4 \\
-4
\end{array}\right]\) span or don't span \(\mathbb{R}^4\).



  
\end{exerciseStatement}

\begin{exerciseStatement}
    Explain why the vectors \(\left[\begin{array}{r}
-4 \\
3 \\
-3 \\
-1
\end{array}\right] , \left[\begin{array}{r}
0 \\
-2 \\
2 \\
1
\end{array}\right] , \left[\begin{array}{r}
-4 \\
0 \\
3 \\
-2
\end{array}\right] , \text{ and } \left[\begin{array}{r}
-2 \\
-4 \\
3 \\
2
\end{array}\right]\) span or don't span \(\mathbb{R}^4\).



  
\end{exerciseStatement}

\begin{exerciseStatement}
    Explain why the vectors \(\left[\begin{array}{r}
1 \\
-4 \\
1 \\
1
\end{array}\right] , \left[\begin{array}{r}
-4 \\
2 \\
-3 \\
1
\end{array}\right] , \left[\begin{array}{r}
-2 \\
-5 \\
2 \\
-3
\end{array}\right] , \left[\begin{array}{r}
-2 \\
-1 \\
-1 \\
0
\end{array}\right] , \left[\begin{array}{r}
-1 \\
3 \\
4 \\
-2
\end{array}\right] , \text{ and } \left[\begin{array}{r}
-4 \\
2 \\
-3 \\
2
\end{array}\right]\) span or don't span \(\mathbb{R}^4\).



  
\end{exerciseStatement}

\begin{exerciseStatement}
    Explain why the vectors \(\left[\begin{array}{r}
0 \\
-5 \\
2 \\
-5
\end{array}\right] , \left[\begin{array}{r}
3 \\
-5 \\
-4 \\
0
\end{array}\right] , \left[\begin{array}{r}
0 \\
0 \\
0 \\
0
\end{array}\right] , \left[\begin{array}{r}
3 \\
5 \\
-8 \\
10
\end{array}\right] , \text{ and } \left[\begin{array}{r}
1 \\
-2 \\
2 \\
-3
\end{array}\right]\) span or don't span \(\mathbb{R}^4\).



  
\end{exerciseStatement}

\begin{exerciseStatement}
    Explain why the vectors \(\left[\begin{array}{r}
-1 \\
-1 \\
3 \\
-1
\end{array}\right] , \left[\begin{array}{r}
2 \\
-3 \\
3 \\
-5
\end{array}\right] , \left[\begin{array}{r}
-3 \\
3 \\
4 \\
-2
\end{array}\right] , \left[\begin{array}{r}
12 \\
-10 \\
-8 \\
-4
\end{array}\right] , \text{ and } \left[\begin{array}{r}
3 \\
2 \\
1 \\
-5
\end{array}\right]\) span or don't span \(\mathbb{R}^4\).



  
\end{exerciseStatement}

\begin{exerciseStatement}
    Explain why the vectors \(\left[\begin{array}{r}
-2 \\
0 \\
-3 \\
4
\end{array}\right] , \left[\begin{array}{r}
-1 \\
1 \\
-4 \\
-4
\end{array}\right] , \left[\begin{array}{r}
3 \\
-1 \\
1 \\
-1
\end{array}\right] , \left[\begin{array}{r}
3 \\
-1 \\
19 \\
2
\end{array}\right] , \left[\begin{array}{r}
2 \\
2 \\
-3 \\
0
\end{array}\right] , \text{ and } \left[\begin{array}{r}
-5 \\
4 \\
2 \\
-4
\end{array}\right]\) span or don't span \(\mathbb{R}^4\).



  
\end{exerciseStatement}

\begin{exerciseStatement}
    Explain why the vectors \(\left[\begin{array}{r}
4 \\
4 \\
0 \\
-3
\end{array}\right] , \left[\begin{array}{r}
-3 \\
1 \\
-2 \\
-1
\end{array}\right] , \left[\begin{array}{r}
-1 \\
3 \\
3 \\
2
\end{array}\right] , \left[\begin{array}{r}
-4 \\
2 \\
2 \\
1
\end{array}\right] , \text{ and } \left[\begin{array}{r}
3 \\
-4 \\
-3 \\
-5
\end{array}\right]\) span or don't span \(\mathbb{R}^4\).



  
\end{exerciseStatement}

\begin{exerciseStatement}
    Explain why the vectors \(\left[\begin{array}{r}
1 \\
0 \\
2 \\
3
\end{array}\right] , \left[\begin{array}{r}
-4 \\
-3 \\
3 \\
0
\end{array}\right] , \left[\begin{array}{r}
3 \\
2 \\
3 \\
2
\end{array}\right] , \left[\begin{array}{r}
-1 \\
-2 \\
1 \\
4
\end{array}\right] , \text{ and } \left[\begin{array}{r}
-3 \\
-4 \\
3 \\
-5
\end{array}\right]\) span or don't span \(\mathbb{R}^4\).



  
\end{exerciseStatement}

\begin{exerciseStatement}
    Explain why the vectors \(\left[\begin{array}{r}
-5 \\
-3 \\
1 \\
4
\end{array}\right] , \left[\begin{array}{r}
-2 \\
1 \\
-1 \\
-3
\end{array}\right] , \left[\begin{array}{r}
-1 \\
-1 \\
3 \\
4
\end{array}\right] , \text{ and } \left[\begin{array}{r}
3 \\
-2 \\
-4 \\
-4
\end{array}\right]\) span or don't span \(\mathbb{R}^4\).



  
\end{exerciseStatement}

\begin{exerciseStatement}
    Explain why the vectors \(\left[\begin{array}{r}
-1 \\
-4 \\
-5 \\
4
\end{array}\right] , \left[\begin{array}{r}
4 \\
0 \\
2 \\
1
\end{array}\right] , \left[\begin{array}{r}
3 \\
-3 \\
-2 \\
2
\end{array}\right] , \text{ and } \left[\begin{array}{r}
-3 \\
3 \\
1 \\
-2
\end{array}\right]\) span or don't span \(\mathbb{R}^4\).



  
\end{exerciseStatement}

\begin{exerciseStatement}
    Explain why the vectors \(\left[\begin{array}{r}
-4 \\
-4 \\
-2 \\
3
\end{array}\right] , \left[\begin{array}{r}
-4 \\
0 \\
-5 \\
1
\end{array}\right] , \left[\begin{array}{r}
-4 \\
8 \\
-11 \\
-3
\end{array}\right] , \text{ and } \left[\begin{array}{r}
20 \\
-4 \\
28 \\
-3
\end{array}\right]\) span or don't span \(\mathbb{R}^4\).



  
\end{exerciseStatement}

\begin{exerciseStatement}
    Explain why the vectors \(\left[\begin{array}{r}
3 \\
-2 \\
0 \\
3
\end{array}\right] , \left[\begin{array}{r}
-2 \\
-3 \\
-1 \\
-5
\end{array}\right] , \left[\begin{array}{r}
4 \\
4 \\
0 \\
2
\end{array}\right] , \left[\begin{array}{r}
2 \\
2 \\
-5 \\
-2
\end{array}\right] , \text{ and } \left[\begin{array}{r}
4 \\
0 \\
-4 \\
-2
\end{array}\right]\) span or don't span \(\mathbb{R}^4\).



  
\end{exerciseStatement}

\startStandard{V5}

\begin{exerciseStatement}
    Explain why the vectors \(\left[\begin{array}{r}
0 \\
-6 \\
-3 \\
0 \\
-4
\end{array}\right] , \left[\begin{array}{r}
2 \\
-6 \\
4 \\
0 \\
4
\end{array}\right] , \left[\begin{array}{r}
3 \\
-5 \\
4 \\
4 \\
-1
\end{array}\right] , \text{ and } \left[\begin{array}{r}
3 \\
-1 \\
-2 \\
4 \\
-2
\end{array}\right]\) are linearly dependent or linearly independent.


  
\end{exerciseStatement}

\begin{exerciseStatement}
    Explain why the vectors \(\left[\begin{array}{r}
1 \\
-1 \\
-6 \\
5 \\
2
\end{array}\right] , \left[\begin{array}{r}
-5 \\
-5 \\
-5 \\
-4 \\
-1
\end{array}\right] , \left[\begin{array}{r}
-4 \\
-3 \\
-2 \\
5 \\
0
\end{array}\right] , \left[\begin{array}{r}
12 \\
8 \\
-2 \\
18 \\
6
\end{array}\right] , \text{ and } \left[\begin{array}{r}
19 \\
22 \\
34 \\
-22 \\
-4
\end{array}\right]\) are linearly dependent or linearly independent.


  
\end{exerciseStatement}

\begin{exerciseStatement}
    Explain why the vectors \(\left[\begin{array}{r}
-1 \\
-5 \\
-2 \\
-3 \\
2
\end{array}\right] , \left[\begin{array}{r}
5 \\
4 \\
-3 \\
2 \\
-1
\end{array}\right] , \left[\begin{array}{r}
0 \\
3 \\
-6 \\
-5 \\
3
\end{array}\right] , \left[\begin{array}{r}
-5 \\
1 \\
4 \\
-3 \\
4
\end{array}\right] , \text{ and } \left[\begin{array}{r}
-1 \\
-4 \\
5 \\
2 \\
0
\end{array}\right]\) are linearly dependent or linearly independent.


  
\end{exerciseStatement}

\begin{exerciseStatement}
    Explain why the vectors \(\left[\begin{array}{r}
4 \\
3 \\
-1 \\
3 \\
3
\end{array}\right] , \left[\begin{array}{r}
2 \\
-4 \\
5 \\
5 \\
3
\end{array}\right] , \left[\begin{array}{r}
-6 \\
-10 \\
7 \\
-1 \\
-3
\end{array}\right] , \text{ and } \left[\begin{array}{r}
-10 \\
0 \\
-4 \\
-14 \\
0
\end{array}\right]\) are linearly dependent or linearly independent.


  
\end{exerciseStatement}

\begin{exerciseStatement}
    Explain why the vectors \(\left[\begin{array}{r}
-3 \\
-5 \\
0 \\
4 \\
-6
\end{array}\right] , \left[\begin{array}{r}
0 \\
-4 \\
-4 \\
-1 \\
1
\end{array}\right] , \left[\begin{array}{r}
4 \\
1 \\
2 \\
-4 \\
-4
\end{array}\right] , \left[\begin{array}{r}
-2 \\
-1 \\
-1 \\
-4 \\
4
\end{array}\right] , \text{ and } \left[\begin{array}{r}
3 \\
-3 \\
-3 \\
-1 \\
1
\end{array}\right]\) are linearly dependent or linearly independent.


  
\end{exerciseStatement}

\begin{exerciseStatement}
    Explain why the vectors \(\left[\begin{array}{r}
0 \\
-4 \\
-2 \\
0 \\
-3
\end{array}\right] , \left[\begin{array}{r}
-1 \\
-3 \\
-3 \\
4 \\
1
\end{array}\right] , \left[\begin{array}{r}
-6 \\
-4 \\
-1 \\
-4 \\
4
\end{array}\right] , \text{ and } \left[\begin{array}{r}
3 \\
-3 \\
3 \\
-4 \\
4
\end{array}\right]\) are linearly dependent or linearly independent.


  
\end{exerciseStatement}

\begin{exerciseStatement}
    Explain why the vectors \(\left[\begin{array}{r}
4 \\
-2 \\
-3 \\
2 \\
-2
\end{array}\right] , \left[\begin{array}{r}
2 \\
4 \\
-6 \\
1 \\
-6
\end{array}\right] , \left[\begin{array}{r}
-3 \\
1 \\
0 \\
5 \\
1
\end{array}\right] , \left[\begin{array}{r}
4 \\
5 \\
3 \\
-6 \\
0
\end{array}\right] , \text{ and } \left[\begin{array}{r}
3 \\
-2 \\
-5 \\
4 \\
5
\end{array}\right]\) are linearly dependent or linearly independent.


  
\end{exerciseStatement}

\begin{exerciseStatement}
    Explain why the vectors \(\left[\begin{array}{r}
3 \\
0 \\
5 \\
3 \\
0
\end{array}\right] , \left[\begin{array}{r}
3 \\
0 \\
4 \\
-4 \\
-3
\end{array}\right] , \left[\begin{array}{r}
-3 \\
4 \\
-1 \\
5 \\
-2
\end{array}\right] , \left[\begin{array}{r}
-3 \\
1 \\
-6 \\
4 \\
4
\end{array}\right] , \text{ and } \left[\begin{array}{r}
2 \\
-6 \\
-5 \\
3 \\
-6
\end{array}\right]\) are linearly dependent or linearly independent.


  
\end{exerciseStatement}

\begin{exerciseStatement}
    Explain why the vectors \(\left[\begin{array}{r}
4 \\
1 \\
-5 \\
5 \\
5
\end{array}\right] , \left[\begin{array}{r}
1 \\
1 \\
-5 \\
2 \\
2
\end{array}\right] , \left[\begin{array}{r}
-5 \\
-6 \\
-3 \\
-3 \\
3
\end{array}\right] , \left[\begin{array}{r}
1 \\
-2 \\
10 \\
-1 \\
-1
\end{array}\right] , \text{ and } \left[\begin{array}{r}
3 \\
-4 \\
-5 \\
-3 \\
-4
\end{array}\right]\) are linearly dependent or linearly independent.


  
\end{exerciseStatement}

\begin{exerciseStatement}
    Explain why the vectors \(\left[\begin{array}{r}
-3 \\
0 \\
-3 \\
-4 \\
-1
\end{array}\right] , \left[\begin{array}{r}
0 \\
5 \\
3 \\
1 \\
0
\end{array}\right] , \left[\begin{array}{r}
0 \\
0 \\
-2 \\
0 \\
4
\end{array}\right] , \left[\begin{array}{r}
1 \\
4 \\
5 \\
1 \\
-6
\end{array}\right] , \text{ and } \left[\begin{array}{r}
0 \\
5 \\
-1 \\
1 \\
8
\end{array}\right]\) are linearly dependent or linearly independent.


  
\end{exerciseStatement}

\begin{exerciseStatement}
    Explain why the vectors \(\left[\begin{array}{r}
-2 \\
-1 \\
-6 \\
4 \\
-5
\end{array}\right] , \left[\begin{array}{r}
-4 \\
-1 \\
2 \\
-1 \\
-6
\end{array}\right] , \left[\begin{array}{r}
5 \\
-1 \\
4 \\
2 \\
3
\end{array}\right] , \left[\begin{array}{r}
4 \\
4 \\
3 \\
2 \\
3
\end{array}\right] , \text{ and } \left[\begin{array}{r}
2 \\
-5 \\
3 \\
-6 \\
-4
\end{array}\right]\) are linearly dependent or linearly independent.


  
\end{exerciseStatement}

\begin{exerciseStatement}
    Explain why the vectors \(\left[\begin{array}{r}
-1 \\
-5 \\
-5 \\
-3 \\
-5
\end{array}\right] , \left[\begin{array}{r}
-3 \\
-2 \\
-2 \\
-6 \\
1
\end{array}\right] , \left[\begin{array}{r}
-6 \\
-4 \\
-1 \\
-2 \\
-2
\end{array}\right] , \text{ and } \left[\begin{array}{r}
-7 \\
-9 \\
-3 \\
5 \\
-11
\end{array}\right]\) are linearly dependent or linearly independent.


  
\end{exerciseStatement}

\begin{exerciseStatement}
    Explain why the vectors \(\left[\begin{array}{r}
1 \\
-1 \\
3 \\
-2 \\
4
\end{array}\right] , \left[\begin{array}{r}
-3 \\
4 \\
2 \\
4 \\
2
\end{array}\right] , \left[\begin{array}{r}
7 \\
-9 \\
-1 \\
-10 \\
0
\end{array}\right] , \text{ and } \left[\begin{array}{r}
12 \\
-21 \\
-11 \\
-14 \\
-12
\end{array}\right]\) are linearly dependent or linearly independent.


  
\end{exerciseStatement}

\begin{exerciseStatement}
    Explain why the vectors \(\left[\begin{array}{r}
4 \\
-5 \\
2 \\
-5 \\
-2
\end{array}\right] , \left[\begin{array}{r}
-6 \\
-1 \\
1 \\
0 \\
-5
\end{array}\right] , \left[\begin{array}{r}
1 \\
-4 \\
5 \\
3 \\
-5
\end{array}\right] , \text{ and } \left[\begin{array}{r}
6 \\
18 \\
-9 \\
15 \\
21
\end{array}\right]\) are linearly dependent or linearly independent.


  
\end{exerciseStatement}

\begin{exerciseStatement}
    Explain why the vectors \(\left[\begin{array}{r}
-5 \\
4 \\
-4 \\
-1 \\
1
\end{array}\right] , \left[\begin{array}{r}
-3 \\
3 \\
2 \\
-4 \\
1
\end{array}\right] , \left[\begin{array}{r}
3 \\
-2 \\
-3 \\
-4 \\
4
\end{array}\right] , \text{ and } \left[\begin{array}{r}
16 \\
-13 \\
3 \\
2 \\
1
\end{array}\right]\) are linearly dependent or linearly independent.


  
\end{exerciseStatement}

\begin{exerciseStatement}
    Explain why the vectors \(\left[\begin{array}{r}
2 \\
-4 \\
2 \\
1 \\
-4
\end{array}\right] , \left[\begin{array}{r}
2 \\
-4 \\
1 \\
-3 \\
-6
\end{array}\right] , \left[\begin{array}{r}
2 \\
-3 \\
-2 \\
-1 \\
-1
\end{array}\right] , \left[\begin{array}{r}
-2 \\
2 \\
4 \\
-5 \\
-6
\end{array}\right] , \text{ and } \left[\begin{array}{r}
-4 \\
2 \\
-3 \\
2 \\
4
\end{array}\right]\) are linearly dependent or linearly independent.


  
\end{exerciseStatement}

\begin{exerciseStatement}
    Explain why the vectors \(\left[\begin{array}{r}
0 \\
-6 \\
3 \\
-6 \\
4
\end{array}\right] , \left[\begin{array}{r}
-6 \\
-5 \\
0 \\
-3 \\
-6
\end{array}\right] , \left[\begin{array}{r}
5 \\
0 \\
3 \\
-4 \\
1
\end{array}\right] , \text{ and } \left[\begin{array}{r}
-6 \\
7 \\
-6 \\
9 \\
-14
\end{array}\right]\) are linearly dependent or linearly independent.


  
\end{exerciseStatement}

\begin{exerciseStatement}
    Explain why the vectors \(\left[\begin{array}{r}
-1 \\
-1 \\
4 \\
-1 \\
2
\end{array}\right] , \left[\begin{array}{r}
-4 \\
4 \\
-6 \\
-4 \\
4
\end{array}\right] , \left[\begin{array}{r}
5 \\
-2 \\
-4 \\
-2 \\
3
\end{array}\right] , \text{ and } \left[\begin{array}{r}
-19 \\
11 \\
0 \\
-5 \\
4
\end{array}\right]\) are linearly dependent or linearly independent.


  
\end{exerciseStatement}

\begin{exerciseStatement}
    Explain why the vectors \(\left[\begin{array}{r}
-2 \\
0 \\
-3 \\
5 \\
-1
\end{array}\right] , \left[\begin{array}{r}
2 \\
-5 \\
-5 \\
4 \\
-2
\end{array}\right] , \left[\begin{array}{r}
1 \\
-1 \\
0 \\
-3 \\
5
\end{array}\right] , \left[\begin{array}{r}
6 \\
0 \\
9 \\
-15 \\
3
\end{array}\right] , \text{ and } \left[\begin{array}{r}
-6 \\
5 \\
2 \\
-4 \\
-1
\end{array}\right]\) are linearly dependent or linearly independent.


  
\end{exerciseStatement}

\begin{exerciseStatement}
    Explain why the vectors \(\left[\begin{array}{r}
5 \\
5 \\
0 \\
-3 \\
-3
\end{array}\right] , \left[\begin{array}{r}
1 \\
-2 \\
-1 \\
-1 \\
4
\end{array}\right] , \left[\begin{array}{r}
4 \\
2 \\
-5 \\
2 \\
2
\end{array}\right] , \text{ and } \left[\begin{array}{r}
2 \\
4 \\
-5 \\
-3 \\
-6
\end{array}\right]\) are linearly dependent or linearly independent.


  
\end{exerciseStatement}

\begin{exerciseStatement}
    Explain why the vectors \(\left[\begin{array}{r}
1 \\
1 \\
3 \\
3 \\
-5
\end{array}\right] , \left[\begin{array}{r}
-4 \\
4 \\
1 \\
4 \\
2
\end{array}\right] , \left[\begin{array}{r}
4 \\
2 \\
-5 \\
-3 \\
0
\end{array}\right] , \text{ and } \left[\begin{array}{r}
3 \\
-5 \\
-4 \\
-7 \\
3
\end{array}\right]\) are linearly dependent or linearly independent.


  
\end{exerciseStatement}

\begin{exerciseStatement}
    Explain why the vectors \(\left[\begin{array}{r}
-6 \\
-4 \\
1 \\
5 \\
-3
\end{array}\right] , \left[\begin{array}{r}
1 \\
-1 \\
-4 \\
-1 \\
-1
\end{array}\right] , \left[\begin{array}{r}
4 \\
5 \\
3 \\
-2 \\
-5
\end{array}\right] , \text{ and } \left[\begin{array}{r}
15 \\
6 \\
-10 \\
-14 \\
13
\end{array}\right]\) are linearly dependent or linearly independent.


  
\end{exerciseStatement}

\begin{exerciseStatement}
    Explain why the vectors \(\left[\begin{array}{r}
-1 \\
-5 \\
-5 \\
5 \\
5
\end{array}\right] , \left[\begin{array}{r}
0 \\
2 \\
1 \\
3 \\
-3
\end{array}\right] , \left[\begin{array}{r}
-3 \\
3 \\
-4 \\
4 \\
2
\end{array}\right] , \text{ and } \left[\begin{array}{r}
-2 \\
2 \\
-3 \\
-3 \\
-3
\end{array}\right]\) are linearly dependent or linearly independent.


  
\end{exerciseStatement}

\begin{exerciseStatement}
    Explain why the vectors \(\left[\begin{array}{r}
-5 \\
-5 \\
-2 \\
4 \\
-5
\end{array}\right] , \left[\begin{array}{r}
0 \\
-6 \\
1 \\
4 \\
-3
\end{array}\right] , \left[\begin{array}{r}
0 \\
-12 \\
2 \\
8 \\
-6
\end{array}\right] , \text{ and } \left[\begin{array}{r}
17 \\
23 \\
-15 \\
-18 \\
23
\end{array}\right]\) are linearly dependent or linearly independent.


  
\end{exerciseStatement}

\begin{exerciseStatement}
    Explain why the vectors \(\left[\begin{array}{r}
4 \\
-3 \\
0 \\
4 \\
-2
\end{array}\right] , \left[\begin{array}{r}
-3 \\
-1 \\
-6 \\
4 \\
5
\end{array}\right] , \left[\begin{array}{r}
0 \\
2 \\
3 \\
3 \\
-6
\end{array}\right] , \text{ and } \left[\begin{array}{r}
-3 \\
4 \\
0 \\
-5 \\
-3
\end{array}\right]\) are linearly dependent or linearly independent.


  
\end{exerciseStatement}

\startStandard{V6}

\begin{exerciseStatement}
    Explain why the vectors \(\left[\begin{array}{r}
0 \\
-5 \\
-2
\end{array}\right] , \left[\begin{array}{r}
0 \\
-4 \\
2
\end{array}\right] , \text{ and } \left[\begin{array}{r}
-5 \\
3 \\
0
\end{array}\right]\) are or are not a basis of \(\mathbb{R}^3\)


  
\end{exerciseStatement}

\begin{exerciseStatement}
    Explain why the vectors \(\left[\begin{array}{r}
0 \\
-1 \\
-5 \\
4 \\
2
\end{array}\right] , \left[\begin{array}{r}
-4 \\
-4 \\
-5 \\
-3 \\
-1
\end{array}\right] , \left[\begin{array}{r}
8 \\
11 \\
25 \\
-6 \\
-4
\end{array}\right] , \left[\begin{array}{r}
-3 \\
1 \\
-3 \\
-5 \\
-3
\end{array}\right] , \text{ and } \left[\begin{array}{r}
-2 \\
-9 \\
1 \\
0 \\
2
\end{array}\right]\) are or are not a basis of \(\mathbb{R}^5\)


  
\end{exerciseStatement}

\begin{exerciseStatement}
    Explain why the vectors \(\left[\begin{array}{r}
-1 \\
-4 \\
-2 \\
-3 \\
1
\end{array}\right] , \left[\begin{array}{r}
4 \\
3 \\
-2 \\
2 \\
-1
\end{array}\right] , \left[\begin{array}{r}
0 \\
3 \\
-5 \\
-4 \\
3
\end{array}\right] , \left[\begin{array}{r}
-4 \\
1 \\
3 \\
-3 \\
3
\end{array}\right] , \text{ and } \left[\begin{array}{r}
-1 \\
-4 \\
4 \\
2 \\
0
\end{array}\right]\) are or are not a basis of \(\mathbb{R}^5\)


  
\end{exerciseStatement}

\begin{exerciseStatement}
    Explain why the vectors \(\left[\begin{array}{r}
3 \\
3 \\
-1 \\
2
\end{array}\right] , \left[\begin{array}{r}
2 \\
1 \\
-4 \\
4
\end{array}\right] , \left[\begin{array}{r}
-3 \\
-3 \\
1 \\
-2
\end{array}\right] , \text{ and } \left[\begin{array}{r}
2 \\
3 \\
-3 \\
-4
\end{array}\right]\) are or are not a basis of \(\mathbb{R}^4\)


  
\end{exerciseStatement}

\begin{exerciseStatement}
    Explain why the vectors \(\left[\begin{array}{r}
-2 \\
-4 \\
0 \\
3
\end{array}\right] , \left[\begin{array}{r}
-5 \\
0 \\
-3 \\
-4
\end{array}\right] , \left[\begin{array}{r}
-1 \\
1 \\
3 \\
0
\end{array}\right] , \text{ and } \left[\begin{array}{r}
1 \\
-3 \\
-4 \\
-2
\end{array}\right]\) are or are not a basis of \(\mathbb{R}^4\)


  
\end{exerciseStatement}

\begin{exerciseStatement}
    Explain why the vectors \(\left[\begin{array}{r}
0 \\
-3 \\
-2
\end{array}\right] , \left[\begin{array}{r}
0 \\
-3 \\
-1
\end{array}\right] , \text{ and } \left[\begin{array}{r}
-2 \\
-2 \\
3
\end{array}\right]\) are or are not a basis of \(\mathbb{R}^3\)


  
\end{exerciseStatement}

\begin{exerciseStatement}
    Explain why the vectors \(\left[\begin{array}{r}
3 \\
-2 \\
-3 \\
1 \\
-1
\end{array}\right] , \left[\begin{array}{r}
2 \\
3 \\
-5 \\
1 \\
-5
\end{array}\right] , \left[\begin{array}{r}
-3 \\
1 \\
0 \\
4 \\
1
\end{array}\right] , \left[\begin{array}{r}
4 \\
4 \\
3 \\
-5 \\
0
\end{array}\right] , \text{ and } \left[\begin{array}{r}
2 \\
-2 \\
-4 \\
3 \\
4
\end{array}\right]\) are or are not a basis of \(\mathbb{R}^5\)


  
\end{exerciseStatement}

\begin{exerciseStatement}
    Explain why the vectors \(\left[\begin{array}{r}
2 \\
0 \\
4 \\
2 \\
0
\end{array}\right] , \left[\begin{array}{r}
2 \\
0 \\
3 \\
-3 \\
-3
\end{array}\right] , \left[\begin{array}{r}
-3 \\
3 \\
-1 \\
4 \\
-1
\end{array}\right] , \left[\begin{array}{r}
-3 \\
1 \\
-5 \\
3 \\
4
\end{array}\right] , \text{ and } \left[\begin{array}{r}
1 \\
-5 \\
-4 \\
2 \\
-5
\end{array}\right]\) are or are not a basis of \(\mathbb{R}^5\)


  
\end{exerciseStatement}

\begin{exerciseStatement}
    Explain why the vectors \(\left[\begin{array}{r}
4 \\
0 \\
-4 \\
4 \\
4
\end{array}\right] , \left[\begin{array}{r}
0 \\
1 \\
-4 \\
2 \\
2
\end{array}\right] , \left[\begin{array}{r}
-4 \\
-5 \\
-3 \\
-3 \\
2
\end{array}\right] , \left[\begin{array}{r}
-8 \\
5 \\
15 \\
-9 \\
-14
\end{array}\right] , \text{ and } \left[\begin{array}{r}
3 \\
-3 \\
-4 \\
-2 \\
-3
\end{array}\right]\) are or are not a basis of \(\mathbb{R}^5\)


  
\end{exerciseStatement}

\begin{exerciseStatement}
    Explain why the vectors \(\left[\begin{array}{r}
-3 \\
0 \\
-2 \\
-3
\end{array}\right] , \left[\begin{array}{r}
-1 \\
0 \\
4 \\
2
\end{array}\right] , \left[\begin{array}{r}
1 \\
0 \\
0 \\
0
\end{array}\right] , \text{ and } \left[\begin{array}{r}
-2 \\
0 \\
3 \\
1
\end{array}\right]\) are or are not a basis of \(\mathbb{R}^4\)


  
\end{exerciseStatement}

\begin{exerciseStatement}
    Explain why the vectors \(\left[\begin{array}{r}
-1 \\
-1 \\
-5 \\
4 \\
-4
\end{array}\right] , \left[\begin{array}{r}
-3 \\
-1 \\
2 \\
-1 \\
-5
\end{array}\right] , \left[\begin{array}{r}
4 \\
-1 \\
3 \\
2 \\
2
\end{array}\right] , \left[\begin{array}{r}
3 \\
3 \\
3 \\
2 \\
3
\end{array}\right] , \text{ and } \left[\begin{array}{r}
2 \\
-4 \\
2 \\
-5 \\
-3
\end{array}\right]\) are or are not a basis of \(\mathbb{R}^5\)


  
\end{exerciseStatement}

\begin{exerciseStatement}
    Explain why the vectors \(\left[\begin{array}{r}
-1 \\
-4 \\
-4 \\
-3
\end{array}\right] , \left[\begin{array}{r}
-4 \\
-2 \\
-1 \\
-2
\end{array}\right] , \left[\begin{array}{r}
-5 \\
1 \\
-5 \\
-3
\end{array}\right] , \text{ and } \left[\begin{array}{r}
-1 \\
-2 \\
-2 \\
1
\end{array}\right]\) are or are not a basis of \(\mathbb{R}^4\)


  
\end{exerciseStatement}

\begin{exerciseStatement}
    Explain why the vectors \(\left[\begin{array}{r}
0 \\
-1 \\
3 \\
-1
\end{array}\right] , \left[\begin{array}{r}
3 \\
-2 \\
3 \\
2
\end{array}\right] , \left[\begin{array}{r}
3 \\
2 \\
3 \\
-5
\end{array}\right] , \text{ and } \left[\begin{array}{r}
2 \\
-3 \\
3 \\
1
\end{array}\right]\) are or are not a basis of \(\mathbb{R}^4\)


  
\end{exerciseStatement}

\begin{exerciseStatement}
    Explain why the vectors \(\left[\begin{array}{r}
3 \\
-4 \\
2
\end{array}\right] , \left[\begin{array}{r}
-4 \\
-2 \\
-5
\end{array}\right] , \text{ and } \left[\begin{array}{r}
-14 \\
4 \\
-14
\end{array}\right]\) are or are not a basis of \(\mathbb{R}^3\)


  
\end{exerciseStatement}

\begin{exerciseStatement}
    Explain why the vectors \(\left[\begin{array}{r}
-4 \\
3 \\
-3
\end{array}\right] , \left[\begin{array}{r}
-1 \\
0 \\
-2
\end{array}\right] , \text{ and } \left[\begin{array}{r}
2 \\
1 \\
-4
\end{array}\right]\) are or are not a basis of \(\mathbb{R}^3\)


  
\end{exerciseStatement}

\begin{exerciseStatement}
    Explain why the vectors \(\left[\begin{array}{r}
1 \\
-4 \\
1 \\
1 \\
-4
\end{array}\right] , \left[\begin{array}{r}
2 \\
-3 \\
1 \\
-2 \\
-5
\end{array}\right] , \left[\begin{array}{r}
2 \\
-3 \\
-2 \\
-1 \\
-1
\end{array}\right] , \left[\begin{array}{r}
0 \\
-1 \\
3 \\
4 \\
-2
\end{array}\right] , \text{ and } \left[\begin{array}{r}
-4 \\
12 \\
-6 \\
-4 \\
15
\end{array}\right]\) are or are not a basis of \(\mathbb{R}^5\)


  
\end{exerciseStatement}

\begin{exerciseStatement}
    Explain why the vectors \(\left[\begin{array}{r}
0 \\
-5 \\
2 \\
-5
\end{array}\right] , \left[\begin{array}{r}
3 \\
-5 \\
-4 \\
0
\end{array}\right] , \left[\begin{array}{r}
-3 \\
-5 \\
4 \\
0
\end{array}\right] , \text{ and } \left[\begin{array}{r}
2 \\
-4 \\
1 \\
-1
\end{array}\right]\) are or are not a basis of \(\mathbb{R}^4\)


  
\end{exerciseStatement}

\begin{exerciseStatement}
    Explain why the vectors \(\left[\begin{array}{r}
-1 \\
-1 \\
3 \\
-1
\end{array}\right] , \left[\begin{array}{r}
2 \\
-3 \\
3 \\
-5
\end{array}\right] , \left[\begin{array}{r}
-3 \\
3 \\
4 \\
-2
\end{array}\right] , \text{ and } \left[\begin{array}{r}
-3 \\
-2 \\
2 \\
-5
\end{array}\right]\) are or are not a basis of \(\mathbb{R}^4\)


  
\end{exerciseStatement}

\begin{exerciseStatement}
    Explain why the vectors \(\left[\begin{array}{r}
-2 \\
0 \\
-3 \\
4 \\
-1
\end{array}\right] , \left[\begin{array}{r}
1 \\
-4 \\
-4 \\
3 \\
-1
\end{array}\right] , \left[\begin{array}{r}
1 \\
-1 \\
0 \\
-3 \\
4
\end{array}\right] , \left[\begin{array}{r}
-3 \\
2 \\
2 \\
-3 \\
0
\end{array}\right] , \text{ and } \left[\begin{array}{r}
-3 \\
12 \\
12 \\
-9 \\
3
\end{array}\right]\) are or are not a basis of \(\mathbb{R}^5\)


  
\end{exerciseStatement}

\begin{exerciseStatement}
    Explain why the vectors \(\left[\begin{array}{r}
4 \\
4 \\
0 \\
-3
\end{array}\right] , \left[\begin{array}{r}
-3 \\
1 \\
-2 \\
-1
\end{array}\right] , \left[\begin{array}{r}
-1 \\
3 \\
3 \\
2
\end{array}\right] , \text{ and } \left[\begin{array}{r}
-4 \\
2 \\
2 \\
1
\end{array}\right]\) are or are not a basis of \(\mathbb{R}^4\)


  
\end{exerciseStatement}

\begin{exerciseStatement}
    Explain why the vectors \(\left[\begin{array}{r}
1 \\
0 \\
2 \\
3
\end{array}\right] , \left[\begin{array}{r}
-4 \\
-3 \\
3 \\
0
\end{array}\right] , \left[\begin{array}{r}
3 \\
2 \\
3 \\
2
\end{array}\right] , \text{ and } \left[\begin{array}{r}
-2 \\
1 \\
-12 \\
-11
\end{array}\right]\) are or are not a basis of \(\mathbb{R}^4\)


  
\end{exerciseStatement}

\begin{exerciseStatement}
    Explain why the vectors \(\left[\begin{array}{r}
-5 \\
-3 \\
1
\end{array}\right] , \left[\begin{array}{r}
4 \\
-2 \\
1
\end{array}\right] , \text{ and } \left[\begin{array}{r}
-6 \\
-8 \\
3
\end{array}\right]\) are or are not a basis of \(\mathbb{R}^3\)


  
\end{exerciseStatement}

\begin{exerciseStatement}
    Explain why the vectors \(\left[\begin{array}{r}
-1 \\
-4 \\
-5
\end{array}\right] , \left[\begin{array}{r}
4 \\
4 \\
0
\end{array}\right] , \text{ and } \left[\begin{array}{r}
2 \\
1 \\
3
\end{array}\right]\) are or are not a basis of \(\mathbb{R}^3\)


  
\end{exerciseStatement}

\begin{exerciseStatement}
    Explain why the vectors \(\left[\begin{array}{r}
-4 \\
-4 \\
-2
\end{array}\right] , \left[\begin{array}{r}
3 \\
-4 \\
0
\end{array}\right] , \text{ and } \left[\begin{array}{r}
-5 \\
1 \\
3
\end{array}\right]\) are or are not a basis of \(\mathbb{R}^3\)


  
\end{exerciseStatement}

\begin{exerciseStatement}
    Explain why the vectors \(\left[\begin{array}{r}
3 \\
-2 \\
0 \\
3
\end{array}\right] , \left[\begin{array}{r}
-2 \\
-3 \\
-1 \\
-5
\end{array}\right] , \left[\begin{array}{r}
4 \\
4 \\
0 \\
2
\end{array}\right] , \text{ and } \left[\begin{array}{r}
2 \\
2 \\
-5 \\
-2
\end{array}\right]\) are or are not a basis of \(\mathbb{R}^4\)


  
\end{exerciseStatement}

\startStandard{V7}

\begin{exerciseStatement}
    Find a basis for the subspace
\[W=\mathrm{span}\ \left\{\left[\begin{array}{r}
0 \\
-4 \\
-2 \\
0
\end{array}\right] , \left[\begin{array}{r}
-3 \\
1 \\
-4 \\
3
\end{array}\right] , \left[\begin{array}{r}
0 \\
3 \\
2 \\
-3
\end{array}\right] , \left[\begin{array}{r}
2 \\
2 \\
-1 \\
2
\end{array}\right]\right\}.\]
 Be sure to explain why your result is a basis.


  
\end{exerciseStatement}

\begin{exerciseStatement}
    Find a basis for the subspace
\[W=\mathrm{span}\ \left\{\left[\begin{array}{r}
0 \\
-1 \\
-4 \\
3
\end{array}\right] , \left[\begin{array}{r}
1 \\
-4 \\
-3 \\
-4
\end{array}\right] , \left[\begin{array}{r}
1 \\
1 \\
17 \\
-19
\end{array}\right] , \left[\begin{array}{r}
-2 \\
3 \\
0 \\
-3
\end{array}\right] , \left[\begin{array}{r}
1 \\
-2 \\
-4 \\
-3
\end{array}\right] , \left[\begin{array}{r}
-2 \\
1 \\
-1 \\
1
\end{array}\right]\right\}.\]
 Be sure to explain why your result is a basis.


  
\end{exerciseStatement}

\begin{exerciseStatement}
    Find a basis for the subspace
\[W=\mathrm{span}\ \left\{\left[\begin{array}{r}
-1 \\
-4 \\
-1 \\
-2
\end{array}\right] , \left[\begin{array}{r}
1 \\
3 \\
2 \\
-2
\end{array}\right] , \left[\begin{array}{r}
0 \\
0 \\
0 \\
0
\end{array}\right] , \left[\begin{array}{r}
-4 \\
-3 \\
2 \\
-4
\end{array}\right] , \left[\begin{array}{r}
5 \\
18 \\
7 \\
2
\end{array}\right] , \left[\begin{array}{r}
-1 \\
-3 \\
3 \\
1
\end{array}\right]\right\}.\]
 Be sure to explain why your result is a basis.


  
\end{exerciseStatement}

\begin{exerciseStatement}
    Find a basis for the subspace
\[W=\mathrm{span}\ \left\{\left[\begin{array}{r}
2 \\
2 \\
-1 \\
2
\end{array}\right] , \left[\begin{array}{r}
2 \\
1 \\
-3 \\
3
\end{array}\right] , \left[\begin{array}{r}
3 \\
2 \\
0 \\
-2
\end{array}\right] , \left[\begin{array}{r}
-10 \\
-8 \\
2 \\
0
\end{array}\right] , \left[\begin{array}{r}
-3 \\
-2 \\
7 \\
-10
\end{array}\right]\right\}.\]
 Be sure to explain why your result is a basis.


  
\end{exerciseStatement}

\begin{exerciseStatement}
    Find a basis for the subspace
\[W=\mathrm{span}\ \left\{\left[\begin{array}{r}
-2 \\
-3 \\
0 \\
3
\end{array}\right] , \left[\begin{array}{r}
-4 \\
0 \\
-3 \\
-3
\end{array}\right] , \left[\begin{array}{r}
16 \\
6 \\
9 \\
3
\end{array}\right] , \left[\begin{array}{r}
1 \\
-3 \\
-3 \\
-1
\end{array}\right] , \left[\begin{array}{r}
-14 \\
-18 \\
-12 \\
4
\end{array}\right]\right\}.\]
 Be sure to explain why your result is a basis.


  
\end{exerciseStatement}

\begin{exerciseStatement}
    Find a basis for the subspace
\[W=\mathrm{span}\ \left\{\left[\begin{array}{r}
0 \\
-3 \\
-2 \\
0
\end{array}\right] , \left[\begin{array}{r}
-2 \\
-1 \\
-2 \\
-2
\end{array}\right] , \left[\begin{array}{r}
2 \\
1 \\
-4 \\
-3
\end{array}\right] , \left[\begin{array}{r}
-4 \\
4 \\
0 \\
-4
\end{array}\right]\right\}.\]
 Be sure to explain why your result is a basis.


  
\end{exerciseStatement}

\begin{exerciseStatement}
    Find a basis for the subspace
\[W=\mathrm{span}\ \left\{\left[\begin{array}{r}
3 \\
-2 \\
-2 \\
1
\end{array}\right] , \left[\begin{array}{r}
-1 \\
1 \\
3 \\
-4
\end{array}\right] , \left[\begin{array}{r}
2 \\
-1 \\
1 \\
-3
\end{array}\right] , \left[\begin{array}{r}
0 \\
3 \\
1 \\
3
\end{array}\right] , \left[\begin{array}{r}
-6 \\
3 \\
-3 \\
9
\end{array}\right] , \left[\begin{array}{r}
-2 \\
5 \\
7 \\
-5
\end{array}\right]\right\}.\]
 Be sure to explain why your result is a basis.


  
\end{exerciseStatement}

\begin{exerciseStatement}
    Find a basis for the subspace
\[W=\mathrm{span}\ \left\{\left[\begin{array}{r}
2 \\
0 \\
3 \\
2
\end{array}\right] , \left[\begin{array}{r}
0 \\
2 \\
0 \\
3
\end{array}\right] , \left[\begin{array}{r}
-3 \\
-2 \\
-2 \\
2
\end{array}\right] , \left[\begin{array}{r}
0 \\
4 \\
-5 \\
-10
\end{array}\right] , \left[\begin{array}{r}
0 \\
-4 \\
2 \\
3
\end{array}\right] , \left[\begin{array}{r}
1 \\
-4 \\
-3 \\
2
\end{array}\right]\right\}.\]
 Be sure to explain why your result is a basis.


  
\end{exerciseStatement}

\begin{exerciseStatement}
    Find a basis for the subspace
\[W=\mathrm{span}\ \left\{\left[\begin{array}{r}
3 \\
0 \\
-3 \\
3
\end{array}\right] , \left[\begin{array}{r}
3 \\
0 \\
1 \\
-4
\end{array}\right] , \left[\begin{array}{r}
1 \\
1 \\
-3 \\
-4
\end{array}\right] , \left[\begin{array}{r}
-2 \\
-2 \\
2 \\
-4
\end{array}\right] , \left[\begin{array}{r}
0 \\
0 \\
-3 \\
1
\end{array}\right] , \left[\begin{array}{r}
-6 \\
3 \\
8 \\
3
\end{array}\right]\right\}.\]
 Be sure to explain why your result is a basis.


  
\end{exerciseStatement}

\begin{exerciseStatement}
    Find a basis for the subspace
\[W=\mathrm{span}\ \left\{\left[\begin{array}{r}
-2 \\
0 \\
-2 \\
-3
\end{array}\right] , \left[\begin{array}{r}
-1 \\
0 \\
3 \\
2
\end{array}\right] , \left[\begin{array}{r}
4 \\
0 \\
-12 \\
-8
\end{array}\right] , \left[\begin{array}{r}
-2 \\
0 \\
3 \\
1
\end{array}\right] , \left[\begin{array}{r}
1 \\
0 \\
10 \\
9
\end{array}\right]\right\}.\]
 Be sure to explain why your result is a basis.


  
\end{exerciseStatement}

\begin{exerciseStatement}
    Find a basis for the subspace
\[W=\mathrm{span}\ \left\{\left[\begin{array}{r}
-1 \\
-1 \\
-4 \\
3
\end{array}\right] , \left[\begin{array}{r}
-3 \\
-3 \\
-1 \\
1
\end{array}\right] , \left[\begin{array}{r}
-1 \\
-4 \\
3 \\
-1
\end{array}\right] , \left[\begin{array}{r}
6 \\
15 \\
-8 \\
2
\end{array}\right] , \left[\begin{array}{r}
6 \\
6 \\
2 \\
-2
\end{array}\right] , \left[\begin{array}{r}
-20 \\
-41 \\
13 \\
0
\end{array}\right]\right\}.\]
 Be sure to explain why your result is a basis.


  
\end{exerciseStatement}

\begin{exerciseStatement}
    Find a basis for the subspace
\[W=\mathrm{span}\ \left\{\left[\begin{array}{r}
-1 \\
-3 \\
-3 \\
-2
\end{array}\right] , \left[\begin{array}{r}
-3 \\
-2 \\
-1 \\
-2
\end{array}\right] , \left[\begin{array}{r}
-4 \\
1 \\
-4 \\
-3
\end{array}\right] , \left[\begin{array}{r}
-3 \\
3 \\
-9 \\
-4
\end{array}\right] , \left[\begin{array}{r}
2 \\
-1 \\
3 \\
-4
\end{array}\right]\right\}.\]
 Be sure to explain why your result is a basis.


  
\end{exerciseStatement}

\begin{exerciseStatement}
    Find a basis for the subspace
\[W=\mathrm{span}\ \left\{\left[\begin{array}{r}
0 \\
-1 \\
2 \\
-1
\end{array}\right] , \left[\begin{array}{r}
3 \\
-2 \\
3 \\
1
\end{array}\right] , \left[\begin{array}{r}
3 \\
1 \\
2 \\
-4
\end{array}\right] , \left[\begin{array}{r}
-3 \\
11 \\
-11 \\
-8
\end{array}\right] , \left[\begin{array}{r}
-3 \\
-1 \\
-7 \\
12
\end{array}\right]\right\}.\]
 Be sure to explain why your result is a basis.


  
\end{exerciseStatement}

\begin{exerciseStatement}
    Find a basis for the subspace
\[W=\mathrm{span}\ \left\{\left[\begin{array}{r}
3 \\
-4 \\
1 \\
-3
\end{array}\right] , \left[\begin{array}{r}
-2 \\
-4 \\
-1 \\
1
\end{array}\right] , \left[\begin{array}{r}
-17 \\
16 \\
-6 \\
16
\end{array}\right] , \left[\begin{array}{r}
3 \\
2 \\
-3 \\
-4
\end{array}\right]\right\}.\]
 Be sure to explain why your result is a basis.


  
\end{exerciseStatement}

\begin{exerciseStatement}
    Find a basis for the subspace
\[W=\mathrm{span}\ \left\{\left[\begin{array}{r}
-3 \\
3 \\
-3 \\
-1
\end{array}\right] , \left[\begin{array}{r}
0 \\
-2 \\
2 \\
1
\end{array}\right] , \left[\begin{array}{r}
-12 \\
6 \\
-6 \\
-1
\end{array}\right] , \left[\begin{array}{r}
30 \\
-16 \\
16 \\
3
\end{array}\right]\right\}.\]
 Be sure to explain why your result is a basis.


  
\end{exerciseStatement}

\begin{exerciseStatement}
    Find a basis for the subspace
\[W=\mathrm{span}\ \left\{\left[\begin{array}{r}
1 \\
-3 \\
1 \\
1
\end{array}\right] , \left[\begin{array}{r}
-3 \\
1 \\
-3 \\
1
\end{array}\right] , \left[\begin{array}{r}
11 \\
-1 \\
11 \\
-5
\end{array}\right] , \left[\begin{array}{r}
-2 \\
-1 \\
-1 \\
0
\end{array}\right] , \left[\begin{array}{r}
-14 \\
4 \\
-12 \\
2
\end{array}\right] , \left[\begin{array}{r}
-18 \\
-1 \\
-17 \\
8
\end{array}\right]\right\}.\]
 Be sure to explain why your result is a basis.


  
\end{exerciseStatement}

\begin{exerciseStatement}
    Find a basis for the subspace
\[W=\mathrm{span}\ \left\{\left[\begin{array}{r}
0 \\
-4 \\
2 \\
-4
\end{array}\right] , \left[\begin{array}{r}
2 \\
-4 \\
-4 \\
0
\end{array}\right] , \left[\begin{array}{r}
-2 \\
-4 \\
3 \\
0
\end{array}\right] , \left[\begin{array}{r}
4 \\
8 \\
-6 \\
0
\end{array}\right] , \left[\begin{array}{r}
-10 \\
-8 \\
19 \\
-4
\end{array}\right]\right\}.\]
 Be sure to explain why your result is a basis.


  
\end{exerciseStatement}

\begin{exerciseStatement}
    Find a basis for the subspace
\[W=\mathrm{span}\ \left\{\left[\begin{array}{r}
-1 \\
-1 \\
2 \\
-1
\end{array}\right] , \left[\begin{array}{r}
1 \\
-3 \\
2 \\
-4
\end{array}\right] , \left[\begin{array}{r}
-3 \\
3 \\
3 \\
-2
\end{array}\right] , \left[\begin{array}{r}
-3 \\
-2 \\
2 \\
-4
\end{array}\right] , \left[\begin{array}{r}
-4 \\
-6 \\
-2 \\
2
\end{array}\right]\right\}.\]
 Be sure to explain why your result is a basis.


  
\end{exerciseStatement}

\begin{exerciseStatement}
    Find a basis for the subspace
\[W=\mathrm{span}\ \left\{\left[\begin{array}{r}
-2 \\
0 \\
-2 \\
3
\end{array}\right] , \left[\begin{array}{r}
-1 \\
1 \\
-3 \\
-4
\end{array}\right] , \left[\begin{array}{r}
2 \\
-1 \\
0 \\
-1
\end{array}\right] , \left[\begin{array}{r}
0 \\
-2 \\
3 \\
-2
\end{array}\right] , \left[\begin{array}{r}
1 \\
2 \\
-3 \\
0
\end{array}\right] , \left[\begin{array}{r}
-4 \\
3 \\
1 \\
-3
\end{array}\right]\right\}.\]
 Be sure to explain why your result is a basis.


  
\end{exerciseStatement}

\begin{exerciseStatement}
    Find a basis for the subspace
\[W=\mathrm{span}\ \left\{\left[\begin{array}{r}
3 \\
3 \\
0 \\
-2
\end{array}\right] , \left[\begin{array}{r}
-2 \\
1 \\
-1 \\
-1
\end{array}\right] , \left[\begin{array}{r}
7 \\
10 \\
-1 \\
-7
\end{array}\right] , \left[\begin{array}{r}
-1 \\
5 \\
-2 \\
-4
\end{array}\right] , \left[\begin{array}{r}
-6 \\
-6 \\
0 \\
4
\end{array}\right]\right\}.\]
 Be sure to explain why your result is a basis.


  
\end{exerciseStatement}

\begin{exerciseStatement}
    Find a basis for the subspace
\[W=\mathrm{span}\ \left\{\left[\begin{array}{r}
1 \\
0 \\
2 \\
2
\end{array}\right] , \left[\begin{array}{r}
-4 \\
-3 \\
3 \\
0
\end{array}\right] , \left[\begin{array}{r}
2 \\
1 \\
2 \\
1
\end{array}\right] , \left[\begin{array}{r}
-3 \\
-2 \\
0 \\
-3
\end{array}\right] , \left[\begin{array}{r}
11 \\
8 \\
-4 \\
0
\end{array}\right]\right\}.\]
 Be sure to explain why your result is a basis.


  
\end{exerciseStatement}

\begin{exerciseStatement}
    Find a basis for the subspace
\[W=\mathrm{span}\ \left\{\left[\begin{array}{r}
-4 \\
-3 \\
0 \\
3
\end{array}\right] , \left[\begin{array}{r}
-2 \\
1 \\
-1 \\
-3
\end{array}\right] , \left[\begin{array}{r}
24 \\
13 \\
2 \\
-9
\end{array}\right] , \left[\begin{array}{r}
2 \\
-2 \\
-3 \\
-3
\end{array}\right]\right\}.\]
 Be sure to explain why your result is a basis.


  
\end{exerciseStatement}

\begin{exerciseStatement}
    Find a basis for the subspace
\[W=\mathrm{span}\ \left\{\left[\begin{array}{r}
-1 \\
-3 \\
-4 \\
3
\end{array}\right] , \left[\begin{array}{r}
3 \\
0 \\
1 \\
0
\end{array}\right] , \left[\begin{array}{r}
-6 \\
9 \\
9 \\
-9
\end{array}\right] , \left[\begin{array}{r}
-3 \\
3 \\
1 \\
-2
\end{array}\right]\right\}.\]
 Be sure to explain why your result is a basis.


  
\end{exerciseStatement}

\begin{exerciseStatement}
    Find a basis for the subspace
\[W=\mathrm{span}\ \left\{\left[\begin{array}{r}
-3 \\
-3 \\
-2 \\
3
\end{array}\right] , \left[\begin{array}{r}
-3 \\
0 \\
-4 \\
0
\end{array}\right] , \left[\begin{array}{r}
3 \\
-2 \\
-3 \\
-1
\end{array}\right] , \left[\begin{array}{r}
-3 \\
8 \\
-1 \\
-5
\end{array}\right]\right\}.\]
 Be sure to explain why your result is a basis.


  
\end{exerciseStatement}

\begin{exerciseStatement}
    Find a basis for the subspace
\[W=\mathrm{span}\ \left\{\left[\begin{array}{r}
3 \\
-2 \\
0 \\
2
\end{array}\right] , \left[\begin{array}{r}
-2 \\
-2 \\
-1 \\
-4
\end{array}\right] , \left[\begin{array}{r}
9 \\
4 \\
3 \\
14
\end{array}\right] , \left[\begin{array}{r}
2 \\
2 \\
-4 \\
-2
\end{array}\right] , \left[\begin{array}{r}
2 \\
12 \\
14 \\
24
\end{array}\right]\right\}.\]
 Be sure to explain why your result is a basis.


  
\end{exerciseStatement}

\startStandard{V8}

\begin{exerciseStatement}
    Explain how to find the dimension of
\[W=\mathrm{span}\ \left\{\left[\begin{array}{r}
0 \\
-4 \\
-2 \\
0 \\
-3
\end{array}\right] , \left[\begin{array}{r}
1 \\
-4 \\
3 \\
0 \\
3
\end{array}\right] , \left[\begin{array}{r}
1 \\
-8 \\
1 \\
0 \\
0
\end{array}\right] , \left[\begin{array}{r}
2 \\
-8 \\
6 \\
0 \\
6
\end{array}\right]\right\}.\]



  
\end{exerciseStatement}

\begin{exerciseStatement}
    Explain how to find the dimension of
\[W=\mathrm{span}\ \left\{\left[\begin{array}{r}
0 \\
-1 \\
-4 \\
3 \\
1
\end{array}\right] , \left[\begin{array}{r}
-4 \\
-3 \\
-4 \\
-3 \\
-1
\end{array}\right] , \left[\begin{array}{r}
-12 \\
-8 \\
-8 \\
-12 \\
-4
\end{array}\right] , \left[\begin{array}{r}
-3 \\
1 \\
-2 \\
-4 \\
-3
\end{array}\right] , \left[\begin{array}{r}
-16 \\
-12 \\
-16 \\
-12 \\
-4
\end{array}\right] , \left[\begin{array}{r}
-4 \\
1 \\
-4 \\
-3 \\
-4
\end{array}\right]\right\}.\]



  
\end{exerciseStatement}

\begin{exerciseStatement}
    Explain how to find the dimension of
\[W=\mathrm{span}\ \left\{\left[\begin{array}{r}
-1 \\
-4 \\
-1 \\
-2 \\
1
\end{array}\right] , \left[\begin{array}{r}
3 \\
2 \\
-2 \\
1 \\
-1
\end{array}\right] , \left[\begin{array}{r}
5 \\
10 \\
0 \\
5 \\
-3
\end{array}\right] , \left[\begin{array}{r}
-4 \\
1 \\
2 \\
-2 \\
2
\end{array}\right] , \left[\begin{array}{r}
-9 \\
-20 \\
4 \\
-5 \\
3
\end{array}\right] , \left[\begin{array}{r}
0 \\
17 \\
6 \\
6 \\
-2
\end{array}\right]\right\}.\]



  
\end{exerciseStatement}

\begin{exerciseStatement}
    Explain how to find the dimension of
\[W=\mathrm{span}\ \left\{\left[\begin{array}{r}
2 \\
2 \\
-1 \\
2 \\
2
\end{array}\right] , \left[\begin{array}{r}
1 \\
-3 \\
3 \\
3 \\
2
\end{array}\right] , \left[\begin{array}{r}
-8 \\
8 \\
-10 \\
-16 \\
-12
\end{array}\right] , \left[\begin{array}{r}
-11 \\
17 \\
-19 \\
-25 \\
-18
\end{array}\right] , \left[\begin{array}{r}
-2 \\
3 \\
-3 \\
0 \\
-2
\end{array}\right]\right\}.\]



  
\end{exerciseStatement}

\begin{exerciseStatement}
    Explain how to find the dimension of
\[W=\mathrm{span}\ \left\{\left[\begin{array}{r}
-2 \\
-3 \\
0 \\
3 \\
-4
\end{array}\right] , \left[\begin{array}{r}
0 \\
-3 \\
-3 \\
-1 \\
1
\end{array}\right] , \left[\begin{array}{r}
2 \\
6 \\
3 \\
-2 \\
3
\end{array}\right] , \left[\begin{array}{r}
0 \\
-3 \\
-3 \\
-1 \\
1
\end{array}\right] , \left[\begin{array}{r}
2 \\
-2 \\
-2 \\
-1 \\
1
\end{array}\right]\right\}.\]



  
\end{exerciseStatement}

\begin{exerciseStatement}
    Explain how to find the dimension of
\[W=\mathrm{span}\ \left\{\left[\begin{array}{r}
0 \\
-3 \\
-2 \\
0 \\
-2
\end{array}\right] , \left[\begin{array}{r}
-1 \\
-2 \\
-2 \\
2 \\
1
\end{array}\right] , \left[\begin{array}{r}
-3 \\
6 \\
2 \\
6 \\
11
\end{array}\right] , \left[\begin{array}{r}
2 \\
-2 \\
2 \\
-3 \\
3
\end{array}\right]\right\}.\]



  
\end{exerciseStatement}

\begin{exerciseStatement}
    Explain how to find the dimension of
\[W=\mathrm{span}\ \left\{\left[\begin{array}{r}
3 \\
-2 \\
-2 \\
1 \\
-1
\end{array}\right] , \left[\begin{array}{r}
1 \\
3 \\
-4 \\
1 \\
-4
\end{array}\right] , \left[\begin{array}{r}
-14 \\
13 \\
6 \\
-4 \\
1
\end{array}\right] , \left[\begin{array}{r}
-16 \\
18 \\
4 \\
-4 \\
-2
\end{array}\right] , \left[\begin{array}{r}
-26 \\
10 \\
24 \\
-10 \\
16
\end{array}\right] , \left[\begin{array}{r}
1 \\
1 \\
-3 \\
3 \\
0
\end{array}\right]\right\}.\]



  
\end{exerciseStatement}

\begin{exerciseStatement}
    Explain how to find the dimension of
\[W=\mathrm{span}\ \left\{\left[\begin{array}{r}
2 \\
0 \\
3 \\
2 \\
0
\end{array}\right] , \left[\begin{array}{r}
2 \\
0 \\
3 \\
-3 \\
-2
\end{array}\right] , \left[\begin{array}{r}
-2 \\
2 \\
-1 \\
3 \\
-1
\end{array}\right] , \left[\begin{array}{r}
6 \\
-6 \\
3 \\
-19 \\
-1
\end{array}\right] , \left[\begin{array}{r}
-6 \\
2 \\
-7 \\
-6 \\
-3
\end{array}\right] , \left[\begin{array}{r}
-20 \\
16 \\
-14 \\
50 \\
4
\end{array}\right]\right\}.\]



  
\end{exerciseStatement}

\begin{exerciseStatement}
    Explain how to find the dimension of
\[W=\mathrm{span}\ \left\{\left[\begin{array}{r}
3 \\
0 \\
-3 \\
3 \\
3
\end{array}\right] , \left[\begin{array}{r}
0 \\
1 \\
-4 \\
1 \\
1
\end{array}\right] , \left[\begin{array}{r}
-3 \\
-4 \\
-2 \\
-2 \\
2
\end{array}\right] , \left[\begin{array}{r}
-4 \\
0 \\
0 \\
-3 \\
1
\end{array}\right] , \left[\begin{array}{r}
2 \\
-3 \\
-3 \\
-2 \\
-3
\end{array}\right] , \left[\begin{array}{r}
-1 \\
9 \\
3 \\
-1 \\
-5
\end{array}\right]\right\}.\]



  
\end{exerciseStatement}

\begin{exerciseStatement}
    Explain how to find the dimension of
\[W=\mathrm{span}\ \left\{\left[\begin{array}{r}
-2 \\
0 \\
-2 \\
-3 \\
-1
\end{array}\right] , \left[\begin{array}{r}
0 \\
3 \\
2 \\
1 \\
0
\end{array}\right] , \left[\begin{array}{r}
0 \\
0 \\
-2 \\
0 \\
3
\end{array}\right] , \left[\begin{array}{r}
0 \\
3 \\
-2 \\
1 \\
6
\end{array}\right] , \left[\begin{array}{r}
1 \\
0 \\
-3 \\
-4 \\
1
\end{array}\right]\right\}.\]



  
\end{exerciseStatement}

\begin{exerciseStatement}
    Explain how to find the dimension of
\[W=\mathrm{span}\ \left\{\left[\begin{array}{r}
-1 \\
-1 \\
-4 \\
3 \\
-3
\end{array}\right] , \left[\begin{array}{r}
-3 \\
-1 \\
1 \\
-1 \\
-4
\end{array}\right] , \left[\begin{array}{r}
17 \\
9 \\
16 \\
-11 \\
31
\end{array}\right] , \left[\begin{array}{r}
2 \\
2 \\
2 \\
1 \\
2
\end{array}\right] , \left[\begin{array}{r}
6 \\
2 \\
10 \\
-12 \\
16
\end{array}\right] , \left[\begin{array}{r}
25 \\
13 \\
16 \\
-8 \\
41
\end{array}\right]\right\}.\]



  
\end{exerciseStatement}

\begin{exerciseStatement}
    Explain how to find the dimension of
\[W=\mathrm{span}\ \left\{\left[\begin{array}{r}
-1 \\
-3 \\
-3 \\
-2 \\
-3
\end{array}\right] , \left[\begin{array}{r}
-2 \\
-1 \\
-2 \\
-4 \\
1
\end{array}\right] , \left[\begin{array}{r}
-4 \\
-3 \\
-1 \\
-1 \\
-2
\end{array}\right] , \left[\begin{array}{r}
1 \\
2 \\
-1 \\
3 \\
-4
\end{array}\right] , \left[\begin{array}{r}
-4 \\
2 \\
1 \\
-2 \\
2
\end{array}\right]\right\}.\]



  
\end{exerciseStatement}

\begin{exerciseStatement}
    Explain how to find the dimension of
\[W=\mathrm{span}\ \left\{\left[\begin{array}{r}
0 \\
-1 \\
2 \\
-1 \\
3
\end{array}\right] , \left[\begin{array}{r}
-2 \\
3 \\
1 \\
3 \\
1
\end{array}\right] , \left[\begin{array}{r}
-4 \\
10 \\
-6 \\
10 \\
-10
\end{array}\right] , \left[\begin{array}{r}
1 \\
2 \\
2 \\
-3 \\
-1
\end{array}\right] , \left[\begin{array}{r}
7 \\
-27 \\
1 \\
-12 \\
16
\end{array}\right]\right\}.\]



  
\end{exerciseStatement}

\begin{exerciseStatement}
    Explain how to find the dimension of
\[W=\mathrm{span}\ \left\{\left[\begin{array}{r}
3 \\
-4 \\
1 \\
-3 \\
-2
\end{array}\right] , \left[\begin{array}{r}
-4 \\
-1 \\
1 \\
0 \\
-4
\end{array}\right] , \left[\begin{array}{r}
1 \\
-3 \\
3 \\
2 \\
-3
\end{array}\right] , \left[\begin{array}{r}
-15 \\
4 \\
5 \\
13 \\
-8
\end{array}\right]\right\}.\]



  
\end{exerciseStatement}

\begin{exerciseStatement}
    Explain how to find the dimension of
\[W=\mathrm{span}\ \left\{\left[\begin{array}{r}
-3 \\
3 \\
-3 \\
-1 \\
0
\end{array}\right] , \left[\begin{array}{r}
-2 \\
2 \\
1 \\
-3 \\
0
\end{array}\right] , \left[\begin{array}{r}
2 \\
-2 \\
-2 \\
-3 \\
2
\end{array}\right] , \left[\begin{array}{r}
2 \\
3 \\
3 \\
3 \\
1
\end{array}\right]\right\}.\]



  
\end{exerciseStatement}

\begin{exerciseStatement}
    Explain how to find the dimension of
\[W=\mathrm{span}\ \left\{\left[\begin{array}{r}
1 \\
-3 \\
1 \\
1 \\
-3
\end{array}\right] , \left[\begin{array}{r}
1 \\
-3 \\
1 \\
-2 \\
-4
\end{array}\right] , \left[\begin{array}{r}
1 \\
-2 \\
-2 \\
-1 \\
-1
\end{array}\right] , \left[\begin{array}{r}
0 \\
-1 \\
2 \\
3 \\
-2
\end{array}\right] , \left[\begin{array}{r}
-1 \\
3 \\
1 \\
-3 \\
3
\end{array}\right] , \left[\begin{array}{r}
-4 \\
13 \\
-7 \\
0 \\
16
\end{array}\right]\right\}.\]



  
\end{exerciseStatement}

\begin{exerciseStatement}
    Explain how to find the dimension of
\[W=\mathrm{span}\ \left\{\left[\begin{array}{r}
0 \\
-4 \\
2 \\
-4 \\
2
\end{array}\right] , \left[\begin{array}{r}
-4 \\
-4 \\
0 \\
-2 \\
-4
\end{array}\right] , \left[\begin{array}{r}
8 \\
-4 \\
6 \\
-8 \\
14
\end{array}\right] , \left[\begin{array}{r}
4 \\
8 \\
-2 \\
6 \\
2
\end{array}\right] , \left[\begin{array}{r}
-2 \\
0 \\
0 \\
0 \\
-3
\end{array}\right]\right\}.\]



  
\end{exerciseStatement}

\begin{exerciseStatement}
    Explain how to find the dimension of
\[W=\mathrm{span}\ \left\{\left[\begin{array}{r}
-1 \\
-1 \\
2 \\
-1 \\
1
\end{array}\right] , \left[\begin{array}{r}
-3 \\
2 \\
-4 \\
-3 \\
3
\end{array}\right] , \left[\begin{array}{r}
3 \\
-2 \\
-3 \\
-2 \\
2
\end{array}\right] , \left[\begin{array}{r}
8 \\
-2 \\
4 \\
8 \\
-8
\end{array}\right] , \left[\begin{array}{r}
-4 \\
-4 \\
2 \\
2 \\
-2
\end{array}\right]\right\}.\]



  
\end{exerciseStatement}

\begin{exerciseStatement}
    Explain how to find the dimension of
\[W=\mathrm{span}\ \left\{\left[\begin{array}{r}
-2 \\
0 \\
-2 \\
3 \\
-1
\end{array}\right] , \left[\begin{array}{r}
1 \\
-3 \\
-4 \\
2 \\
-1
\end{array}\right] , \left[\begin{array}{r}
0 \\
-1 \\
0 \\
-2 \\
3
\end{array}\right] , \left[\begin{array}{r}
3 \\
11 \\
18 \\
-11 \\
0
\end{array}\right] , \left[\begin{array}{r}
-4 \\
3 \\
1 \\
-3 \\
-1
\end{array}\right] , \left[\begin{array}{r}
-4 \\
-3 \\
-4 \\
0 \\
0
\end{array}\right]\right\}.\]



  
\end{exerciseStatement}

\begin{exerciseStatement}
    Explain how to find the dimension of
\[W=\mathrm{span}\ \left\{\left[\begin{array}{r}
3 \\
3 \\
0 \\
-2 \\
-2
\end{array}\right] , \left[\begin{array}{r}
1 \\
-1 \\
-1 \\
-1 \\
3
\end{array}\right] , \left[\begin{array}{r}
3 \\
3 \\
0 \\
-2 \\
-2
\end{array}\right] , \left[\begin{array}{r}
-2 \\
2 \\
2 \\
2 \\
-6
\end{array}\right] , \left[\begin{array}{r}
3 \\
2 \\
1 \\
0 \\
2
\end{array}\right]\right\}.\]



  
\end{exerciseStatement}

\begin{exerciseStatement}
    Explain how to find the dimension of
\[W=\mathrm{span}\ \left\{\left[\begin{array}{r}
1 \\
0 \\
2 \\
2 \\
-4
\end{array}\right] , \left[\begin{array}{r}
-3 \\
3 \\
0 \\
2 \\
1
\end{array}\right] , \left[\begin{array}{r}
4 \\
-3 \\
2 \\
0 \\
-5
\end{array}\right] , \left[\begin{array}{r}
-3 \\
-3 \\
-3 \\
2 \\
-4
\end{array}\right] , \left[\begin{array}{r}
-19 \\
9 \\
-11 \\
2 \\
16
\end{array}\right]\right\}.\]



  
\end{exerciseStatement}

\begin{exerciseStatement}
    Explain how to find the dimension of
\[W=\mathrm{span}\ \left\{\left[\begin{array}{r}
-4 \\
-3 \\
0 \\
3 \\
-2
\end{array}\right] , \left[\begin{array}{r}
1 \\
-1 \\
-3 \\
-1 \\
-1
\end{array}\right] , \left[\begin{array}{r}
2 \\
3 \\
2 \\
-2 \\
-3
\end{array}\right] , \left[\begin{array}{r}
11 \\
5 \\
-5 \\
-8 \\
8
\end{array}\right]\right\}.\]



  
\end{exerciseStatement}

\begin{exerciseStatement}
    Explain how to find the dimension of
\[W=\mathrm{span}\ \left\{\left[\begin{array}{r}
-1 \\
-3 \\
-4 \\
3 \\
3
\end{array}\right] , \left[\begin{array}{r}
0 \\
1 \\
0 \\
2 \\
-2
\end{array}\right] , \left[\begin{array}{r}
-1 \\
-5 \\
-4 \\
-1 \\
7
\end{array}\right] , \left[\begin{array}{r}
1 \\
3 \\
4 \\
-3 \\
-3
\end{array}\right]\right\}.\]



  
\end{exerciseStatement}

\begin{exerciseStatement}
    Explain how to find the dimension of
\[W=\mathrm{span}\ \left\{\left[\begin{array}{r}
-3 \\
-3 \\
-2 \\
3 \\
-3
\end{array}\right] , \left[\begin{array}{r}
0 \\
-4 \\
0 \\
3 \\
-2
\end{array}\right] , \left[\begin{array}{r}
-3 \\
-1 \\
3 \\
1 \\
-3
\end{array}\right] , \left[\begin{array}{r}
1 \\
-1 \\
2 \\
-2 \\
3
\end{array}\right]\right\}.\]



  
\end{exerciseStatement}

\begin{exerciseStatement}
    Explain how to find the dimension of
\[W=\mathrm{span}\ \left\{\left[\begin{array}{r}
3 \\
-2 \\
0 \\
2 \\
-2
\end{array}\right] , \left[\begin{array}{r}
-2 \\
-1 \\
-4 \\
3 \\
3
\end{array}\right] , \left[\begin{array}{r}
0 \\
1 \\
2 \\
2 \\
-4
\end{array}\right] , \left[\begin{array}{r}
-2 \\
3 \\
0 \\
-4 \\
-2
\end{array}\right] , \left[\begin{array}{r}
0 \\
-7 \\
-4 \\
4 \\
18
\end{array}\right]\right\}.\]



  
\end{exerciseStatement}

\startStandard{V9}

\begin{exerciseStatement}
    Find a basis for the subspace of \(\mathcal{P}^3\)
\[W=\mathrm{span}\ \left\{-3 \, x^{3} - 6 \, x^{2} - 5 , -6 \, x^{3} + 2 \, x^{2} - 4 \, x , 4 \, x^{3} - 2 \, x^{2} - x + 3 , 0 , 3 \, x^{3} + 4 \, x^{2} + 4\right\}.\]
 Be sure to explain why your result is a basis.


  
\end{exerciseStatement}

\begin{exerciseStatement}
    Find a basis for the subspace of \(M_{2,2}\)
\[W=\mathrm{span}\ \left\{\left[\begin{array}{cc}
5 & 2 \\
-5 & -5
\end{array}\right] , \left[\begin{array}{cc}
-3 & -2 \\
5 & 0
\end{array}\right] , \left[\begin{array}{cc}
-8 & -5 \\
0 & 2
\end{array}\right] , \left[\begin{array}{cc}
-5 & -4 \\
-1 & -4
\end{array}\right] , \left[\begin{array}{cc}
3 & 1 \\
-1 & -6
\end{array}\right]\right\}.\]
 Be sure to explain why your result is a basis.


  
\end{exerciseStatement}

\begin{exerciseStatement}
    Find a basis for the subspace of \(\mathcal{P}^3\)
\[W=\mathrm{span}\ \left\{-6 \, x^{3} - 5 \, x^{2} - 3 \, x + 1 , -2 \, x^{3} - 5 \, x^{2} - x + 5 , -2 \, x^{3} - 5 \, x^{2} - x + 5 , 4 \, x^{3} + 5 \, x^{2} + 2 \, x - 3 , 24 \, x^{3} + 25 \, x^{2} + 12 \, x - 11\right\}.\]
 Be sure to explain why your result is a basis.


  
\end{exerciseStatement}

\begin{exerciseStatement}
    Find a basis for the subspace of \(\mathcal{P}^3\)
\[W=\mathrm{span}\ \left\{-7 \, x^{3} + x^{2} + 2 \, x + 7 , 9 \, x^{3} - 7 \, x^{2} - 10 \, x - 5 , -4 \, x^{3} + 2 \, x^{2} + 3 \, x + 3 , -3 \, x^{3} + 4 \, x^{2} + 3 \, x - 3 , -x^{3} + 3 \, x^{2} + 4 \, x - 1\right\}.\]
 Be sure to explain why your result is a basis.


  
\end{exerciseStatement}

\begin{exerciseStatement}
    Find a basis for the subspace of \(\mathcal{P}^3\)
\[W=\mathrm{span}\ \left\{8 \, x^{3} - 5 \, x^{2} + 9 \, x - 7 , -5 \, x^{2} - 3 \, x + 1 , -4 \, x^{3} - 6 \, x + 4 , -4 \, x^{3} - x^{2} - x - 2 , -4 \, x^{3} - 4 \, x^{2} + 2 \, x + 1\right\}.\]
 Be sure to explain why your result is a basis.


  
\end{exerciseStatement}

\begin{exerciseStatement}
    Find a basis for the subspace of \(M_{2,2}\)
\[W=\mathrm{span}\ \left\{\left[\begin{array}{cc}
0 & 9 \\
3 & 9
\end{array}\right] , \left[\begin{array}{cc}
6 & -18 \\
-2 & -16
\end{array}\right] , \left[\begin{array}{cc}
-6 & 0 \\
-4 & -2
\end{array}\right] , \left[\begin{array}{cc}
0 & -3 \\
-1 & -3
\end{array}\right] , \left[\begin{array}{cc}
0 & 3 \\
1 & 3
\end{array}\right]\right\}.\]
 Be sure to explain why your result is a basis.


  
\end{exerciseStatement}

\begin{exerciseStatement}
    Find a basis for the subspace of \(M_{2,2}\)
\[W=\mathrm{span}\ \left\{\left[\begin{array}{cc}
10 & 6 \\
-2 & -2
\end{array}\right] , \left[\begin{array}{cc}
2 & 6 \\
-4 & -7
\end{array}\right] , \left[\begin{array}{cc}
2 & -2 \\
2 & 4
\end{array}\right] , \left[\begin{array}{cc}
1 & 0 \\
5 & 1
\end{array}\right] , \left[\begin{array}{cc}
4 & 4 \\
-2 & -3
\end{array}\right]\right\}.\]
 Be sure to explain why your result is a basis.


  
\end{exerciseStatement}

\begin{exerciseStatement}
    Find a basis for the subspace of \(\mathcal{P}^3\)
\[W=\mathrm{span}\ \left\{-2 \, x^{3} + 5 \, x^{2} - x + 4 , -30 \, x^{3} + 30 \, x^{2} - 18 \, x + 12 , 5 \, x^{3} + 3 \, x + 3 , 3 \, x^{2} + 3 , 5 \, x^{3} - 9 \, x^{2} + 3 \, x - 6\right\}.\]
 Be sure to explain why your result is a basis.


  
\end{exerciseStatement}

\begin{exerciseStatement}
    Find a basis for the subspace of \(M_{2,2}\)
\[W=\mathrm{span}\ \left\{\left[\begin{array}{cc}
-6 & -6 \\
-1 & -7
\end{array}\right] , \left[\begin{array}{cc}
-6 & -3 \\
-3 & 3
\end{array}\right] , \left[\begin{array}{cc}
4 & 4 \\
1 & -5
\end{array}\right] , \left[\begin{array}{cc}
-5 & 1 \\
0 & -4
\end{array}\right] , \left[\begin{array}{cc}
5 & 5 \\
1 & 1
\end{array}\right]\right\}.\]
 Be sure to explain why your result is a basis.


  
\end{exerciseStatement}

\begin{exerciseStatement}
    Find a basis for the subspace of \(M_{2,2}\)
\[W=\mathrm{span}\ \left\{\left[\begin{array}{cc}
3 & 1 \\
0 & 0
\end{array}\right] , \left[\begin{array}{cc}
-4 & -1 \\
0 & 5
\end{array}\right] , \left[\begin{array}{cc}
1 & -3 \\
0 & -3
\end{array}\right] , \left[\begin{array}{cc}
-6 & 5 \\
0 & 11
\end{array}\right] , \left[\begin{array}{cc}
3 & 0 \\
0 & -15
\end{array}\right]\right\}.\]
 Be sure to explain why your result is a basis.


  
\end{exerciseStatement}

\begin{exerciseStatement}
    Find a basis for the subspace of \(\mathcal{P}^3\)
\[W=\mathrm{span}\ \left\{-x^{3} - 4 \, x^{2} - 5 \, x + 4 , 9 \, x^{3} + 13 \, x^{2} + 17 \, x - 17 , 17 \, x^{3} + 22 \, x^{2} + 29 \, x - 30 , -6 \, x^{3} - x^{2} - 2 \, x + 5 , -3 \, x^{3} + 11 \, x^{2} + 13 \, x - 7\right\}.\]
 Be sure to explain why your result is a basis.


  
\end{exerciseStatement}

\begin{exerciseStatement}
    Find a basis for the subspace of \(\mathcal{P}^3\)
\[W=\mathrm{span}\ \left\{-2 \, x^{3} - 2 \, x^{2} - x - 4 , -6 \, x^{3} + x^{2} - 6 \, x - 2 , -13 \, x^{3} + 3 \, x^{2} - 3 \, x + 1 , -5 \, x^{3} - 5 \, x^{2} - x - 1 , -2 \, x^{3} - 3 \, x^{2} - 5 \, x - 3\right\}.\]
 Be sure to explain why your result is a basis.


  
\end{exerciseStatement}

\begin{exerciseStatement}
    Find a basis for the subspace of \(\mathcal{P}^3\)
\[W=\mathrm{span}\ \left\{4 \, x^{3} - 3 \, x^{2} + 4 \, x - 2 , -15 \, x^{3} + 16 \, x^{2} - 7 \, x + 13 , 3 \, x^{3} + 2 \, x^{2} + 4 \, x + 2 , 3 \, x^{3} - x^{2} + x - 1 , 3 \, x^{3} - 4 \, x^{2} + 2 \, x - 6\right\}.\]
 Be sure to explain why your result is a basis.


  
\end{exerciseStatement}

\begin{exerciseStatement}
    Find a basis for the subspace of \(M_{2,2}\)
\[W=\mathrm{span}\ \left\{\left[\begin{array}{cc}
-5 & -2 \\
-6 & -1
\end{array}\right] , \left[\begin{array}{cc}
-4 & 5 \\
3 & -5
\end{array}\right] , \left[\begin{array}{cc}
-4 & 4 \\
-5 & 2
\end{array}\right] , \left[\begin{array}{cc}
1 & 0 \\
-5 & 1
\end{array}\right] , \left[\begin{array}{cc}
4 & -16 \\
-7 & -6
\end{array}\right]\right\}.\]
 Be sure to explain why your result is a basis.


  
\end{exerciseStatement}

\begin{exerciseStatement}
    Find a basis for the subspace of \(\mathcal{P}^3\)
\[W=\mathrm{span}\ \left\{4 \, x^{3} - 4 \, x^{2} - 3 \, x - 2 , -4 \, x^{3} + 4 \, x^{2} - 5 \, x - 6 , 5 \, x^{3} - 5 \, x^{2} + 9 \, x + 13 , 3 \, x^{3} - 3 \, x^{2} + x - 1 , 4 \, x^{3} + 5 \, x^{2} + 5 \, x + 3\right\}.\]
 Be sure to explain why your result is a basis.


  
\end{exerciseStatement}

\begin{exerciseStatement}
    Find a basis for the subspace of \(M_{2,2}\)
\[W=\mathrm{span}\ \left\{\left[\begin{array}{cc}
-3 & -2 \\
-1 & -1
\end{array}\right] , \left[\begin{array}{cc}
-1 & -14 \\
-2 & -4
\end{array}\right] , \left[\begin{array}{cc}
1 & -4 \\
2 & -4
\end{array}\right] , \left[\begin{array}{cc}
2 & 2 \\
-4 & 2
\end{array}\right] , \left[\begin{array}{cc}
1 & -3 \\
-6 & 2
\end{array}\right]\right\}.\]
 Be sure to explain why your result is a basis.


  
\end{exerciseStatement}

\begin{exerciseStatement}
    Find a basis for the subspace of \(M_{2,2}\)
\[W=\mathrm{span}\ \left\{\left[\begin{array}{cc}
0 & -3 \\
-6 & 5
\end{array}\right] , \left[\begin{array}{cc}
-6 & 4 \\
-6 & -5
\end{array}\right] , \left[\begin{array}{cc}
-1 & 0 \\
-6 & 3
\end{array}\right] , \left[\begin{array}{cc}
0 & 6 \\
12 & -10
\end{array}\right] , \left[\begin{array}{cc}
0 & 3 \\
-4 & 1
\end{array}\right]\right\}.\]
 Be sure to explain why your result is a basis.


  
\end{exerciseStatement}

\begin{exerciseStatement}
    Find a basis for the subspace of \(\mathcal{P}^3\)
\[W=\mathrm{span}\ \left\{4 \, x^{3} - 4 \, x^{2} + 2 \, x - 1 , 5 \, x^{3} + 4 \, x^{2} - 4 \, x - 6 , 4 \, x^{3} - x^{2} - x - 1 , x^{3} + 2 \, x^{2} + 4 \, x - 6 , 3 \, x^{3} - 2 \, x^{2} - 4 \, x - 2\right\}.\]
 Be sure to explain why your result is a basis.


  
\end{exerciseStatement}

\begin{exerciseStatement}
    Find a basis for the subspace of \(\mathcal{P}^3\)
\[W=\mathrm{span}\ \left\{14 \, x^{3} - 14 \, x^{2} - 7 \, x , -4 \, x^{3} + 3 \, x^{2} + 2 \, x - 3 , -5 \, x^{3} + 2 \, x^{2} - x + 5 , -x^{3} + 4 \, x^{2} + 4 \, x - 5 , -3 \, x^{3} - 2 \, x + 5\right\}.\]
 Be sure to explain why your result is a basis.


  
\end{exerciseStatement}

\begin{exerciseStatement}
    Find a basis for the subspace of \(M_{2,2}\)
\[W=\mathrm{span}\ \left\{\left[\begin{array}{cc}
-1 & -1 \\
4 & 4
\end{array}\right] , \left[\begin{array}{cc}
-3 & -3 \\
1 & -2
\end{array}\right] , \left[\begin{array}{cc}
-4 & 10 \\
10 & 0
\end{array}\right] , \left[\begin{array}{cc}
-2 & 5 \\
5 & 0
\end{array}\right] , \left[\begin{array}{cc}
-4 & 10 \\
10 & 0
\end{array}\right]\right\}.\]
 Be sure to explain why your result is a basis.


  
\end{exerciseStatement}

\begin{exerciseStatement}
    Find a basis for the subspace of \(\mathcal{P}^3\)
\[W=\mathrm{span}\ \left\{4 \, x^{3} + 2 \, x^{2} + 4 \, x + 1 , 4 \, x^{3} - 5 \, x^{2} - 4 \, x - 4 , 3 \, x^{3} + x^{2} + x - 2 , 4 \, x^{3} - 4 \, x^{2} - 5 \, x + 3 , -3 \, x^{2} - 5 \, x + 2\right\}.\]
 Be sure to explain why your result is a basis.


  
\end{exerciseStatement}

\begin{exerciseStatement}
    Find a basis for the subspace of \(M_{2,2}\)
\[W=\mathrm{span}\ \left\{\left[\begin{array}{cc}
-4 & -6 \\
-4 & 1
\end{array}\right] , \left[\begin{array}{cc}
5 & 3 \\
-2 & -5
\end{array}\right] , \left[\begin{array}{cc}
-4 & -1 \\
-1 & 4
\end{array}\right] , \left[\begin{array}{cc}
21 & 16 \\
14 & -8
\end{array}\right] , \left[\begin{array}{cc}
5 & -3 \\
1 & -1
\end{array}\right]\right\}.\]
 Be sure to explain why your result is a basis.


  
\end{exerciseStatement}

\begin{exerciseStatement}
    Find a basis for the subspace of \(M_{2,2}\)
\[W=\mathrm{span}\ \left\{\left[\begin{array}{cc}
5 & 5 \\
0 & 2
\end{array}\right] , \left[\begin{array}{cc}
-3 & -1 \\
-5 & -5
\end{array}\right] , \left[\begin{array}{cc}
1 & 3 \\
-3 & -3
\end{array}\right] , \left[\begin{array}{cc}
-11 & -5 \\
-11 & -15
\end{array}\right] , \left[\begin{array}{cc}
-2 & 2 \\
-3 & -3
\end{array}\right]\right\}.\]
 Be sure to explain why your result is a basis.


  
\end{exerciseStatement}

\begin{exerciseStatement}
    Find a basis for the subspace of \(\mathcal{P}^3\)
\[W=\mathrm{span}\ \left\{20 \, x^{3} + 5 \, x^{2} + 20 \, x - 7 , -6 \, x^{3} - 5 \, x + 4 , -3 \, x^{3} + 4 \, x^{2} - x + 1 , -4 \, x^{3} + 2 \, x^{2} + 5 \, x - 2 , -2 \, x^{3} - 5 \, x^{2} - 5 \, x - 5\right\}.\]
 Be sure to explain why your result is a basis.


  
\end{exerciseStatement}

\begin{exerciseStatement}
    Find a basis for the subspace of \(\mathcal{P}^3\)
\[W=\mathrm{span}\ \left\{3 \, x^{3} + 23 \, x^{2} - 2 \, x - 12 , 2 \, x^{3} - 4 \, x^{2} - 4 \, x + 4 , -3 \, x^{2} + 4 \, x - 2 , 5 \, x^{2} + 4 \, x - 6 , -x^{3} - 3 \, x^{2} - 2 \, x + 4\right\}.\]
 Be sure to explain why your result is a basis.


  
\end{exerciseStatement}

\startStandard{V10}

\begin{exerciseStatement}
    Find a basis for the solution space of the homogeneous system
\begin{align*}
 -4 x_ 1 -4 x_ 2 &= 0  \\
  -10 x_ 1 + 2 x_ 2 -6 x_ 3 &= 0  \\
  -12 x_ 1 -6 x_ 2 -3 x_ 3 &= 0  \\
  4 x_ 1 + 4 x_ 2 &= 0  \\
 \end{align*}



  
\end{exerciseStatement}

\begin{exerciseStatement}
    Find a basis for the solution space of the homogeneous system
\begin{align*}
 -1 x_ 1 + 1 x_ 2 -4 x_ 3 -8 x_ 4 + 2 x_ 5 &= 0  \\
  -4 x_ 1 -1 x_ 2 + 10 x_ 3 + 17 x_ 4 -5 x_ 5 &= 0  \\
  -3 x_ 1 -6 x_ 2 + 10 x_ 3 + 27 x_ 4 -5 x_ 5 &= 0  \\
  -2 x_ 1 + 5 x_ 2 + 10 x_ 3 + 5 x_ 4 -5 x_ 5 &= 0  \\
 \end{align*}



  
\end{exerciseStatement}

\begin{exerciseStatement}
    Find a basis for the solution space of the homogeneous system
\begin{align*}
 -3 x_ 1 -1 x_ 2 + 1 x_ 3 + 2 x_ 4 + 3 x_ 5 &= 0  \\
  -5 x_ 2 + 5 x_ 3 + 5 x_ 4 + 15 x_ 5 &= 0  \\
  -6 x_ 1 -2 x_ 2 + 2 x_ 3 + 4 x_ 4 + 6 x_ 5 &= 0  \\
  18 x_ 1 -3 x_ 2 + 3 x_ 3 -3 x_ 4 + 9 x_ 5 &= 0  \\
 \end{align*}



  
\end{exerciseStatement}

\begin{exerciseStatement}
    Find a basis for the solution space of the homogeneous system
\begin{align*}
 3 x_ 1 + 4 x_ 2 + 4 x_ 3 + 3 x_ 4 &= 0  \\
  4 x_ 1 + 3 x_ 2 + 3 x_ 3 + 2 x_ 4 &= 0  \\
  -3 x_ 1 -1 x_ 2 -1 x_ 3 -4 x_ 4 &= 0  \\
  -5 x_ 1 + 3 x_ 2 + 3 x_ 3 + 5 x_ 4 &= 0  \\
 \end{align*}



  
\end{exerciseStatement}

\begin{exerciseStatement}
    Find a basis for the solution space of the homogeneous system
\begin{align*}
 -4 x_ 1 -6 x_ 2 + 1 x_ 3 -3 x_ 4 &= 0  \\
  -2 x_ 1 + 4 x_ 3 -5 x_ 4 &= 0  \\
  -1 x_ 1 -4 x_ 2 + 1 x_ 3 &= 0  \\
  -1 x_ 1 -4 x_ 2 + 2 x_ 3 + 4 x_ 4 &= 0  \\
 \end{align*}



  
\end{exerciseStatement}

\begin{exerciseStatement}
    Find a basis for the solution space of the homogeneous system
\begin{align*}
 -3 x_ 1 + 9 x_ 2 &= 0  \\
  -1 x_ 1 + 3 x_ 2 -4 x_ 3 &= 0  \\
  -3 x_ 1 + 9 x_ 2 -2 x_ 3 &= 0  \\
  -3 x_ 1 + 9 x_ 2 &= 0  \\
 \end{align*}



  
\end{exerciseStatement}

\begin{exerciseStatement}
    Find a basis for the solution space of the homogeneous system
\begin{align*}
 -4 x_ 1 -8 x_ 2 -2 x_ 3 + 4 x_ 4 + 4 x_ 5 &= 0  \\
  4 x_ 1 + 2 x_ 2 + 2 x_ 3 -2 x_ 4 &= 0  \\
  8 x_ 1 + 1 x_ 2 + 4 x_ 3 -3 x_ 4 + 2 x_ 5 &= 0  \\
  -12 x_ 1 + 6 x_ 2 -6 x_ 3 + 2 x_ 4 -8 x_ 5 &= 0  \\
 \end{align*}



  
\end{exerciseStatement}

\begin{exerciseStatement}
    Find a basis for the solution space of the homogeneous system
\begin{align*}
 -3 x_ 2 + 3 x_ 3 -3 x_ 4 -9 x_ 5 &= 0  \\
  3 x_ 1 -6 x_ 2 -3 x_ 4 + 6 x_ 5 &= 0  \\
  -5 x_ 2 + 5 x_ 3 + 4 x_ 4 -15 x_ 5 &= 0  \\
  4 x_ 1 -11 x_ 2 + 3 x_ 3 -1 x_ 4 -1 x_ 5 &= 0  \\
 \end{align*}



  
\end{exerciseStatement}

\begin{exerciseStatement}
    Find a basis for the solution space of the homogeneous system
\begin{align*}
 5 x_ 1 -11 x_ 2 + 4 x_ 3 -10 x_ 4 -3 x_ 5 &= 0  \\
  1 x_ 1 -2 x_ 2 + 1 x_ 3 -2 x_ 4 -3 x_ 5 &= 0  \\
  1 x_ 1 -8 x_ 2 -5 x_ 3 -2 x_ 4 + 3 x_ 5 &= 0  \\
  -5 x_ 1 + 20 x_ 2 + 5 x_ 3 + 10 x_ 4 -5 x_ 5 &= 0  \\
 \end{align*}



  
\end{exerciseStatement}

\begin{exerciseStatement}
    Find a basis for the solution space of the homogeneous system
\begin{align*}
 -3 x_ 1 -1 x_ 3 + 3 x_ 4 &= 0  \\
  &= 0  \\
  -3 x_ 1 + 5 x_ 3 + 3 x_ 4 &= 0  \\
  -4 x_ 1 + 3 x_ 3 + 4 x_ 4 &= 0  \\
 \end{align*}



  
\end{exerciseStatement}

\begin{exerciseStatement}
    Find a basis for the solution space of the homogeneous system
\begin{align*}
 -12 x_ 1 -5 x_ 2 -2 x_ 3 -6 x_ 4 -7 x_ 5 &= 0  \\
  -9 x_ 1 -4 x_ 2 -1 x_ 3 + 5 x_ 4 -5 x_ 5 &= 0  \\
  -8 x_ 1 -1 x_ 2 -6 x_ 3 -1 x_ 4 -7 x_ 5 &= 0  \\
  8 x_ 1 + 2 x_ 2 + 4 x_ 3 + 4 x_ 4 + 6 x_ 5 &= 0  \\
 \end{align*}



  
\end{exerciseStatement}

\begin{exerciseStatement}
    Find a basis for the solution space of the homogeneous system
\begin{align*}
 -5 x_ 1 + 1 x_ 2 -1 x_ 3 -2 x_ 4 &= 0  \\
  -3 x_ 1 -6 x_ 2 -5 x_ 3 + 1 x_ 4 &= 0  \\
  -2 x_ 1 -4 x_ 2 -5 x_ 3 + 4 x_ 4 &= 0  \\
  -2 x_ 1 -1 x_ 2 -3 x_ 3 -2 x_ 4 &= 0  \\
 \end{align*}



  
\end{exerciseStatement}

\begin{exerciseStatement}
    Find a basis for the solution space of the homogeneous system
\begin{align*}
 4 x_ 1 + 1 x_ 2 + 5 x_ 3 + 2 x_ 4 &= 0  \\
  -3 x_ 1 -1 x_ 2 -4 x_ 3 -4 x_ 4 &= 0  \\
  4 x_ 1 + 3 x_ 2 + 7 x_ 3 + 3 x_ 4 &= 0  \\
  2 x_ 1 -2 x_ 2 + 2 x_ 4 &= 0  \\
 \end{align*}



  
\end{exerciseStatement}

\begin{exerciseStatement}
    Find a basis for the solution space of the homogeneous system
\begin{align*}
 -12 x_ 1 -2 x_ 2 + 4 x_ 3 &= 0  \\
  15 x_ 1 -6 x_ 2 -5 x_ 3 &= 0  \\
  -6 x_ 1 -1 x_ 2 + 2 x_ 3 &= 0  \\
  15 x_ 1 + 1 x_ 2 -5 x_ 3 &= 0  \\
 \end{align*}



  
\end{exerciseStatement}

\begin{exerciseStatement}
    Find a basis for the solution space of the homogeneous system
\begin{align*}
 1 x_ 1 + 1 x_ 2 -5 x_ 3 &= 0  \\
  3 x_ 1 -3 x_ 2 + 4 x_ 3 &= 0  \\
  -2 x_ 1 + 3 x_ 2 -4 x_ 3 &= 0  \\
  -3 x_ 1 + 2 x_ 2 -1 x_ 3 &= 0  \\
 \end{align*}



  
\end{exerciseStatement}

\begin{exerciseStatement}
    Find a basis for the solution space of the homogeneous system
\begin{align*}
 2 x_ 1 + 6 x_ 2 -6 x_ 3 -4 x_ 4 -1 x_ 5 &= 0  \\
  -4 x_ 1 + 2 x_ 3 + 2 x_ 4 &= 0  \\
  2 x_ 1 + 6 x_ 2 -3 x_ 3 -4 x_ 4 -1 x_ 5 &= 0  \\
  1 x_ 1 -3 x_ 2 -2 x_ 3 + 1 x_ 4 + 3 x_ 5 &= 0  \\
 \end{align*}



  
\end{exerciseStatement}

\begin{exerciseStatement}
    Find a basis for the solution space of the homogeneous system
\begin{align*}
 4 x_ 1 -6 x_ 2 + 1 x_ 4 &= 0  \\
  -6 x_ 1 + 5 x_ 2 -6 x_ 3 -1 x_ 4 &= 0  \\
  -5 x_ 1 + 3 x_ 3 + 2 x_ 4 &= 0  \\
  + 3 x_ 2 -6 x_ 3 -2 x_ 4 &= 0  \\
 \end{align*}



  
\end{exerciseStatement}

\begin{exerciseStatement}
    Find a basis for the solution space of the homogeneous system
\begin{align*}
 2 x_ 1 + 3 x_ 2 -1 x_ 3 + 4 x_ 4 &= 0  \\
  -4 x_ 1 -6 x_ 2 -1 x_ 3 + 5 x_ 4 &= 0  \\
  4 x_ 1 + 4 x_ 2 + 4 x_ 3 -2 x_ 4 &= 0  \\
  -6 x_ 1 + 2 x_ 2 -1 x_ 3 -4 x_ 4 &= 0  \\
 \end{align*}



  
\end{exerciseStatement}

\begin{exerciseStatement}
    Find a basis for the solution space of the homogeneous system
\begin{align*}
 -1 x_ 2 -2 x_ 4 + 2 x_ 5 &= 0  \\
  -3 x_ 1 + 2 x_ 2 -4 x_ 3 &= 0  \\
  5 x_ 1 -5 x_ 2 + 7 x_ 3 -3 x_ 4 + 3 x_ 5 &= 0  \\
  -3 x_ 1 -5 x_ 2 + 15 x_ 3 + 5 x_ 4 -5 x_ 5 &= 0  \\
 \end{align*}



  
\end{exerciseStatement}

\begin{exerciseStatement}
    Find a basis for the solution space of the homogeneous system
\begin{align*}
 -3 x_ 1 + 4 x_ 2 + 5 x_ 3 + 2 x_ 4 &= 0  \\
  1 x_ 1 + 4 x_ 2 + 5 x_ 3 + 6 x_ 4 &= 0  \\
  -2 x_ 1 + 2 x_ 2 -2 x_ 4 &= 0  \\
  -1 x_ 1 -5 x_ 2 -3 x_ 3 -4 x_ 4 &= 0  \\
 \end{align*}



  
\end{exerciseStatement}

\begin{exerciseStatement}
    Find a basis for the solution space of the homogeneous system
\begin{align*}
 -5 x_ 1 + 7 x_ 2 + 1 x_ 3 -9 x_ 4 &= 0  \\
  -4 x_ 1 + 5 x_ 2 + 1 x_ 3 -7 x_ 4 &= 0  \\
  4 x_ 1 -17 x_ 2 + 3 x_ 3 + 11 x_ 4 &= 0  \\
  1 x_ 1 -11 x_ 2 + 3 x_ 3 + 5 x_ 4 &= 0  \\
 \end{align*}



  
\end{exerciseStatement}

\begin{exerciseStatement}
    Find a basis for the solution space of the homogeneous system
\begin{align*}
 -1 x_ 1 -6 x_ 2 -3 x_ 3 &= 0  \\
  4 x_ 1 -4 x_ 2 + 1 x_ 3 &= 0  \\
  5 x_ 1 + 1 x_ 2 -1 x_ 3 &= 0  \\
  3 x_ 1 + 5 x_ 2 -4 x_ 3 &= 0  \\
 \end{align*}



  
\end{exerciseStatement}

\begin{exerciseStatement}
    Find a basis for the solution space of the homogeneous system
\begin{align*}
 5 x_ 1 -1 x_ 2 -8 x_ 3 &= 0  \\
  -5 x_ 2 + 10 x_ 3 &= 0  \\
  2 x_ 1 -5 x_ 2 + 6 x_ 3 &= 0  \\
  1 x_ 1 + 5 x_ 2 -12 x_ 3 &= 0  \\
 \end{align*}



  
\end{exerciseStatement}

\begin{exerciseStatement}
    Find a basis for the solution space of the homogeneous system
\begin{align*}
 -5 x_ 1 + 20 x_ 2 -5 x_ 3 &= 0  \\
  -5 x_ 1 + 10 x_ 2 &= 0  \\
  -2 x_ 1 + 16 x_ 2 -6 x_ 3 &= 0  \\
  4 x_ 1 -10 x_ 2 + 1 x_ 3 &= 0  \\
 \end{align*}



  
\end{exerciseStatement}

\begin{exerciseStatement}
    Find a basis for the solution space of the homogeneous system
\begin{align*}
 8 x_ 1 -2 x_ 2 + 2 x_ 3 + 4 x_ 4 &= 0  \\
  -6 x_ 1 -3 x_ 2 -6 x_ 3 -3 x_ 4 &= 0  \\
  -1 x_ 2 -1 x_ 3 &= 0  \\
  8 x_ 1 -6 x_ 2 -2 x_ 3 + 4 x_ 4 &= 0  \\
 \end{align*}



  
\end{exerciseStatement}

\end{document}
